\nonstopmode{}
\documentclass[letterpaper]{book}
\usepackage[times,inconsolata,hyper]{Rd}
\usepackage{makeidx}
\usepackage[utf8,latin1]{inputenc}
% \usepackage{graphicx} % @USE GRAPHICX@
\makeindex{}
\begin{document}
\chapter*{}
\begin{center}
{\textbf{\huge Package `BioCro'}}
\par\bigskip{\large \today}
\end{center}
\begin{description}
\raggedright{}
\item[Version]\AsIs{0.92}
\item[Title]\AsIs{BioCro Crop and Agroecosystem Simulator}
\item[Description]\AsIs{Simulation of C4 Crops and Coppice Trees}
\item[Author]\AsIs{Fernando Miguez }\email{femiguez@iastate.edu}\AsIs{,
Deepak Jaiswal }\email{djaiswal@illinois.edu}\AsIs{,
Dan Wang }\email{dwng@illinois.edu}\AsIs{,
David LeBauer }\email{dlebauer@illinois.edu}\AsIs{}
\item[Maintainer]\AsIs{David LeBauer }\email{dlebauer@illinois.edu}\AsIs{}
\item[Depends]\AsIs{R(>= 2.5.0), lattice, data.table}
\item[Suggests]\AsIs{coda, knitr}
\item[Imports]\AsIs{stats, lattice, data.table}
\item[VignetteBuilder]\AsIs{knitr}
\item[License]\AsIs{FreeBSD + file LICENSE}
\item[Collate]\AsIs{'BioCro.R' 'CanA.R' 'Century.R' 'CheckLeapYear.R'
'MCMCBioGro.R' 'OpBioGro.R' 'OpEC4photo.R' 'Opc3photo.R'
'Opc4photo.R' 'RUEmod.R' 'RUEmodMY.R' 'Rmiscanmod.R'
'RssBioGro.R' 'SatWatVap.R' 'SoilEvapo.R' 'biocro-package.R'
'c3CanA.R' 'c3photo.R' 'c4photo.R' 'constrOpBioGro.R'
'eC4photo.R' 'idbp.R' 'lightME.R' 'plotAC.R' 'plotAQ.R'
'showSoilType.R' 'soilML.R' 'sunML.R' 'valid\_dbp.R'
'waterStress.R' 'weach.R' 'weach365.R' 'weach366.R'
'weachNEW.R' 'weach\_imn.R' 'idbpm.R' 'MaizeGro.R'
'OpMaizeGro.R' 'RssMaizeGro.R' 'caneGro.R' 'willowGro.R'
'CropGro.R' 'cropcent.R' 'weachDT.R'}
\end{description}
\Rdcontents{\R{} topics documented:}
\inputencoding{utf8}
\HeaderA{aci}{A/Ci curves}{aci}
\keyword{datasets}{aci}
%
\begin{Description}\relax
Four A/Ci (assimilation vs. intercellular CO2) curves.
\end{Description}
%
\begin{Format}
A data frame with 32 observations on the following 7 variables.
\begin{description}
 \item[list('ID')] Identification for each curve. a numeric
vector\item[list('Photo')] Assimilation. a numeric vector
\item[list('PARi')] Incident Photosynthetic Active Radiation. a numeric
vector\item[list('Tleaf')] Temperature of the leaf. a numeric vector
\item[list('RH\_S')] Realtive humidity. a numeric vector
\item[list('Ci')] Intercellular CO2. a numeric vector
\item[list('CO2\_R')] Reference CO2. a numeric vector
\end{description}
\end{Format}
%
\begin{Details}\relax
Measurements taken on Miscanthus x giganteus.
\end{Details}
%
\begin{Source}\relax
Measurements taken by Dandan Wang.
\end{Source}
%
\begin{References}\relax
Dandan Wang
\end{References}
%
\begin{Examples}
\begin{ExampleCode}
data(aci)
plotAC(aci)
\end{ExampleCode}
\end{Examples}
\inputencoding{utf8}
\HeaderA{annualDB}{Miscanthus dry biomass data.}{annualDB}
\aliasA{annualDB2}{annualDB}{annualDB2}
\keyword{datasets}{annualDB}
%
\begin{Description}\relax
The first column is the thermal time. The second, third,
fourth and fifth columns are miscanthus stem, leaf, root
and rhizome dry biomass in Mg ha\eqn{^{-1}}{} (root is
missing). The sixth column is the leaf area index. The
\code{annualDB.c} version is altered so that root biomass
is not missing and LAI is smaller. The purpose of this
last modification is for testing some functions.
\end{Description}
%
\begin{Format}
data frame of dimensions 5 by 6.
\end{Format}
%
\begin{Source}\relax
Clive Beale and Stephen Long. (1997) Seasonal dynamics of
nutrient accumulation and partitioning in the perennial
C4 grasses Miscanthus x giganteus and Spartina
cynosuroides. Biomass and Bioenergy. 419-428.
\end{Source}
\inputencoding{utf8}
\HeaderA{aq}{A/Q curves}{aq}
\keyword{datasets}{aq}
%
\begin{Description}\relax
Example of A/Q curves which serves as a template for
using the \code{\LinkA{Opc4photo}{Opc4photo}} and
\code{\LinkA{mOpc4photo}{mOpc4photo}} functions.
\end{Description}
%
\begin{Format}
A data frame with 64 observations on the following 6 variables.
\begin{description}
 \item[list('ID')] a numeric vector\item[list('trt')] a factor
with levels \code{mxg} \code{swg}\item[list('A')] a numeric vector.
Assimilation\item[list('PARi')] a numeric vector. Photosynthetic Active
Radiation (incident).\item[list('Tleaf')] a numeric vector. Temperature
of the leaf.\item[list('RH\_S')] a numeric vector. Relative humidity
(fraction).
\end{description}
\end{Format}
%
\begin{Details}\relax
\code{swg} stand for switchgrass (Panicum virgatum)
\code{mxg} stands for miscanthus (Miscanthus x gignateus)


above \textasciitilde{}\textasciitilde{}
\end{Details}
%
\begin{Source}\relax
Data based on measurements made by Dandan Wang. 
reference to a publication or URL from which the data
were obtained \textasciitilde{}\textasciitilde{}
\end{Source}
%
\begin{References}\relax
Dandan Wang 
\textasciitilde{}\textasciitilde{}
\end{References}
%
\begin{Examples}
\begin{ExampleCode}
data(aq)
plotAQ(aq)
\end{ExampleCode}
\end{Examples}
\inputencoding{utf8}
\HeaderA{ardBuck}{Arden Buck Equation from Buck Research Manual (1996) http://cires.colorado.edu/\textasciitilde{}voemel/vp.html}{ardBuck}
%
\begin{Description}\relax
Arden Buck Equation from Buck Research Manual (1996)
http://cires.colorado.edu/\textasciitilde{}voemel/vp.html
\end{Description}
%
\begin{Usage}
\begin{verbatim}
ardBuck(Tcelsius)
\end{verbatim}
\end{Usage}
%
\begin{Arguments}
\begin{ldescription}
\item[\code{Tcelsius}] the temperature in degrees C
\end{ldescription}
\end{Arguments}
\inputencoding{utf8}
\HeaderA{BioGro}{Biomass crops growth simulation}{BioGro}
\aliasA{canopyParms}{BioGro}{canopyParms}
\aliasA{centuryParms}{BioGro}{centuryParms}
\aliasA{nitroParms}{BioGro}{nitroParms}
\aliasA{phenoParms}{BioGro}{phenoParms}
\aliasA{photoParms}{BioGro}{photoParms}
\aliasA{print.BioGro}{BioGro}{print.BioGro}
\aliasA{seneParms}{BioGro}{seneParms}
\aliasA{showSoilType}{BioGro}{showSoilType}
\aliasA{soilParms}{BioGro}{soilParms}
\aliasA{SoilType}{BioGro}{SoilType}
\keyword{models}{BioGro}
%
\begin{Description}\relax
Simulates dry biomass growth during an entire growing
season.  It represents an integration of the
photosynthesis function \code{\LinkA{c4photo}{c4photo}}, canopy
evapo/transpiration \code{\LinkA{CanA}{CanA}}, the multilayer
canopy model \code{\LinkA{sunML}{sunML}} and a dry biomass
partitioning calendar and senescence. It also considers,
carbon and nitrogen cycles and water and nitrogen
limitations.
\end{Description}
%
\begin{Usage}
\begin{verbatim}
  BioGro(WetDat, day1 = NULL, dayn = NULL, timestep = 1,
    lat = 40, iRhizome = 7, irtl = 1e-04,
    canopyControl = list(), seneControl = list(),
    photoControl = list(), phenoControl = list(),
    soilControl = list(), nitroControl = list(),
    centuryControl = list())
\end{verbatim}
\end{Usage}
%
\begin{Arguments}
\begin{ldescription}
\item[\code{WetDat}] weather data as produced by the
\code{\LinkA{weach}{weach}} function.

\item[\code{day1}] first day of the growing season, (1--365).

\item[\code{dayn}] last day of the growing season, (1--365, but
larger than \code{day1}). See details.

\item[\code{timestep}] Simulation timestep, the default of 1
requires houlry weather data. A value of 3 would require
weather data every 3 hours.  This number should be a
divisor of 24.

\item[\code{lat}] latitude, default 40.

\item[\code{iRhizome}] initial dry biomass of the Rhizome (Mg
\eqn{ha^{-1}}{}).

\item[\code{irtl}] Initial rhizome proportion that becomes leaf.
This should not typically be changed, but it can be used
to indirectly control the effect of planting density.

\item[\code{canopyControl}] List that controls aspects of the
canopy simulation. It should be supplied through the
\code{canopyParms} function.

\code{Sp} (specific leaf area) here the units are ha
\eqn{Mg^{-1}}{}.  If you have data in \eqn{m^2}{} of leaf per
kg of dry matter (e.g. 15) then divide by 10 before
inputting this coefficient.

\code{nlayers} (number of layers of the canopy) Maximum
50. To increase the number of layers (more than 50) the
\code{C} source code needs to be changed slightly.

\code{kd} (extinction coefficient for diffuse light)
between 0 and 1.

\code{mResp} (maintenance respiration) a vector of length
2 with the first component for leaf and stem and the
second component for rhizome and root.

\item[\code{seneControl}] List that controls aspects of
senescence simulation. It should be supplied through the
\code{seneParms} function.

\code{senLeaf} Thermal time at which leaf senescence will
start.

\code{senStem} Thermal time at which stem senescence will
start.

\code{senRoot} Thermal time at which root senescence will
start.

\code{senRhizome} Thermal time at which rhizome
senescence will start.

\item[\code{photoControl}] List that controls aspects of
photosynthesis simulation. It should be supplied through
the \code{photoParms} function.

\code{vmax} Vmax passed to the \code{\LinkA{c4photo}{c4photo}}
function.

\code{alpha} alpha parameter passed to the
\code{\LinkA{c4photo}{c4photo}} function.

\code{kparm} kparm parameter passed to the
\code{\LinkA{c4photo}{c4photo}} function.

\code{theta} theta parameter passed to the
\code{\LinkA{c4photo}{c4photo}} function.

\code{beta} beta parameter passed to the
\code{\LinkA{c4photo}{c4photo}} function.

\code{Rd} Rd parameter passed to the
\code{\LinkA{c4photo}{c4photo}} function.

\code{Catm} Catm parameter passed to the
\code{\LinkA{c4photo}{c4photo}} function.

\code{b0} b0 parameter passed to the
\code{\LinkA{c4photo}{c4photo}} function.

\code{b1} b1 parameter passed to the
\code{\LinkA{c4photo}{c4photo}} function.

\item[\code{phenoControl}] List that controls aspects of the
crop phenology. It should be supplied through the
\code{phenoParms} function.

\code{tp1-tp6} thermal times which determine the time
elapsed between phenological stages. Between 0 and tp1 is
the juvenile stage. etc.

\code{kLeaf1-6} proportion of the carbon that is
allocated to leaf for phenological stages 1 through 6.

\code{kStem1-6} proportion of the carbon that is
allocated to stem for phenological stages 1 through 6.

\code{kRoot1-6} proportion of the carbon that is
allocated to root for phenological stages 1 through 6.

\code{kRhizome1-6} proportion of the carbon that is
allocated to rhizome for phenological stages 1 through 6.

\code{kGrain1-6} proportion of the carbon that is
allocated to grain for phenological stages 1 through 6.
At the moment only the last stage (i.e. 6 or
post-flowering) is allowed to be larger than zero. An
error will be returned if kGrain1-5 are different from
zero.

\item[\code{soilControl}] List that controls aspects of the soil
environment. It should be supplied through the
\code{soilParms} function.

\code{FieldC} Field capacity. This can be used to
override the defaults possible from the soil types (see
\code{\LinkA{showSoilType}{showSoilType}}).

\code{WiltP} Wilting point.  This can be used to override
the defaults possible from the soil types (see
\code{\LinkA{showSoilType}{showSoilType}}).

\code{phi1} Parameter which controls the spread of the
logistic function. See \code{\LinkA{wtrstr}{wtrstr}} for more
details.

\code{phi2} Parameter which controls the reduction of the
leaf area growth due to water stress. See
\code{\LinkA{wtrstr}{wtrstr}} for more details.

\code{soilDepth} Maximum depth of the soil that the roots
have access to (i.e. rooting depth).

\code{iWatCont} Initial water content of the soil the
first day of the growing season. It can be a single value
or a vector for the number of layers specified.

\code{soilType} Soil type, default is 6 (a more typical
soil would be 3). To see details use the function
\code{\LinkA{showSoilType}{showSoilType}}.

\code{soilLayer} Integer between 1 and 50. The default is
5. If only one soil layer is used the behavior can be
quite different.

\code{soilDepths} Intervals for the soil layers.

\code{wsFun} one of 'logistic','linear','exp' or 'none'.
Controls the method for the relationship between soil
water content and water stress factor.

\code{scsf} stomatal conductance sensitivity factor
(default = 1). This is an empirical coefficient that
needs to be adjusted for different species.

\code{rfl} Root factor lambda. A Poisson distribution is
used to simulate the distribution of roots in the soil
profile and this parameter can be used to change the
lambda parameter of the Poisson.

\code{rsec} Radiation soil evaporation coefficient.
Empirical coefficient used in the incidence of direct
radiation on soil evaporation.

\code{rsdf} Root soil depth factor. Empirical coefficient
used in calculating the depth of roots as a function of
root biomass.

\item[\code{nitroControl}] List that controls aspects of the
nitrogen environment. It should be supplied through the
\code{nitrolParms} function.

\code{iLeafN} initial value of leaf nitrogen (g m-2).

\code{kLN} coefficient of decrease in leaf nitrogen
during the growing season. The equation is LN = iLeafN *
(Stem + Leaf)\textasciicircum{}-kLN .

\code{Vmax.b1} slope which determines the effect of leaf
nitrogen on Vmax.

\code{alpha.b1} slope which controls the effect of leaf
nitrogen on alpha.

\item[\code{centuryControl}] List that controls aspects of the
Century model for carbon and nitrogen dynamics in the
soil. It should be supplied through the
\code{centuryParms} function.

\code{SC1-9} Soil carbon pools in the soil.  SC1:
Structural surface litter.  SC2: Metabolic surface
litter.  SC3: Structural root litter.  SC4: Metabolic
root litter.  SC5: Surface microbe.  SC6: Soil microbe.
SC7: Slow carbon.  SC8: Passive carbon.  SC9: Leached
carbon.

\code{LeafL.Ln} Leaf litter lignin content.

\code{StemL.Ln} Stem litter lignin content.

\code{RootL.Ln} Root litter lignin content.

\code{RhizomeL.Ln} Rhizome litter lignin content.

\code{LeafL.N} Leaf litter nitrogen content.

\code{StemL.N} Stem litter nitrogen content.

\code{RootL.N} Root litter nitrogen content.

\code{RhizomeL.N} Rhizome litter nitrogen content.

\code{Nfert} Nitrogen from a fertilizer source.

\code{iMinN} Initial value for the mineral nitrogen pool.

\code{Litter} Initial values of litter (leaf, stem, root,
rhizome).

\code{timestep} currently either week (default) or day.
\end{ldescription}
\end{Arguments}
%
\begin{Value}
a \code{\LinkA{list}{list}} structure with components
\end{Value}
%
\begin{Examples}
\begin{ExampleCode}
## Not run: 
data(weather05)

res0 <- BioGro(weather05)

plot(res0)

## Looking at the soil model

res1 <- BioGro(weather05, soilControl = soilParms(soilLayers = 6))
plot(res1, plot.kind='SW') ## Without hydraulic distribution
res2 <- BioGro(weather05, soilControl = soilParms(soilLayers = 6, hydrDist=TRUE))
plot(res2, plot.kind='SW') ## With hydraulic distribution


## Example of user defined soil parameters.
## The effect of phi2 on yield and soil water content

ll.0 <- soilParms(FieldC=0.37,WiltP=0.2,phi2=1)
ll.1 <- soilParms(FieldC=0.37,WiltP=0.2,phi2=2)
ll.2 <- soilParms(FieldC=0.37,WiltP=0.2,phi2=3)
ll.3 <- soilParms(FieldC=0.37,WiltP=0.2,phi2=4)

ans.0 <- BioGro(weather05,soilControl=ll.0)
ans.1 <- BioGro(weather05,soilControl=ll.1)
ans.2 <- BioGro(weather05,soilControl=ll.2)
ans.3 <-BioGro(weather05,soilControl=ll.3)

xyplot(ans.0$SoilWatCont +
       ans.1$SoilWatCont +
       ans.2$SoilWatCont +
       ans.3$SoilWatCont ~ ans.0$DayofYear,
       type='l',
       ylab='Soil water Content (fraction)',
       xlab='DOY')

## Compare LAI

xyplot(ans.0$LAI +
       ans.1$LAI +
       ans.2$LAI +
       ans.3$LAI ~ ans.0$DayofYear,
       type='l',
       ylab='Leaf Area Index',
       xlab='DOY')




## End(Not run)
\end{ExampleCode}
\end{Examples}
\inputencoding{utf8}
\HeaderA{c3CanA}{Simulates canopy assimilation for C3 canopies}{c3CanA}
\keyword{models}{c3CanA}
%
\begin{Description}\relax
It represents an integration of the photosynthesis
function \code{\LinkA{c3photo}{c3photo}}, canopy
evapo/transpiration and the multilayer canopy model
\code{\LinkA{sunML}{sunML}}.
\end{Description}
%
\begin{Usage}
\begin{verbatim}
  c3CanA(lai, doy, hr, solar, temp, rh, windspeed,
    lat = 40, nlayers = 8, kd = 0.1, heightFactor = 3,
    c3photoControl = list(), lnControl = list(),
    StomWS = 1)
\end{verbatim}
\end{Usage}
%
\begin{Arguments}
\begin{ldescription}
\item[\code{lai}] leaf area index.

\item[\code{doy}] day of the year, (1--365).

\item[\code{hr}] hour of the day, (0--23).

\item[\code{solar}] solar radiation (\eqn{\mu}{} mol
\eqn{m^{-2}}{} \eqn{s^{-1}}{}).

\item[\code{temp}] temperature (Celsius).

\item[\code{rh}] relative humidity (0--1).

\item[\code{windspeed}] wind speed (m \eqn{s^{-1}}{}).

\item[\code{lat}] latitude.

\item[\code{nlayers}] number of layers in the simulation of the
canopy (max allowed is 50).

\item[\code{kd}] Ligth extinction coefficient for diffuse
light.

\item[\code{heightFactor}] Height Factor. Divide LAI by this
number to get the height of a crop.

\item[\code{c3photoControl}] list that sets the photosynthesis
parameters for c3 plants through the c3photoParms
function

\item[\code{lnControl}] list that sets the leaf nitrogen
parameters.

LeafN: Initial value of leaf nitrogen (g m-2).

kpLN: coefficient of decrease in leaf nitrogen during the
growing season. The equation is LN = iLeafN * exp(-kLN *
TTc).

lnFun: controls whether there is a decline in leaf
nitrogen with the depth of the canopy. 'none' means no
decline, 'linear' means a linear decline controlled by
the following two parameters.

lnb0: Intercept of the linear decline of leaf nitrogen in
the depth of the canopy.

lnb1: Slope of the linear decline of leaf nitrogen in the
depth of the canopy. The equation is 'vmax = leafN\_lay *
lnb1 + lnb0'.
\end{ldescription}
\end{Arguments}
%
\begin{Details}\relax
The photosynthesis function is modeled after the version
in WIMOVAC.  This is based on the well known Farquar
model.
\end{Details}
%
\begin{Value}
\code{\LinkA{list}{list}}

returns a list with several elements

CanopyAssim: hourly canopy assimilation (\eqn{Mg
  ha^{-1}}{} \eqn{hr^{-1}}{})

CanopyTrans: hourly canopy transpiration (\eqn{Mg
  ha^{-1}}{} \eqn{hr^{-1}}{})

CanopyCond: hourly canopy conductance (units ?)

TranEpen: hourly canopy transpiration according to Penman
(\eqn{Mg ha^{-1}}{} \eqn{hr^{-1}}{})

TranEpen: hourly canopy transpiration according to
Priestly (\eqn{Mg ha^{-1}}{} \eqn{hr^{-1}}{})

LayMat: hourly by Layer matrix containing details of the
calculations by layer (each layer is a row).  col1:
Direct Irradiance col2: Diffuse Irradiance col3: Leaf
area in the sun col4: Leaf area in the shade col5:
Transpiration of leaf area in the sun col6: Transpiration
of leaf area in the shade col7: Assimilation of leaf area
in the sun col8: Assimilation of leaf area in the shade
col9: Difference in temperature between the leaf and the
air (i.e. TLeaf - TAir) for leaves in sun.  col10:
Difference in temperature between the leaf and the air
(i.e. TLeaf - TAir) for leaves in shade.  col11: Stomatal
conductance for leaves in the sun col12: Stomatal
conductance for leaves in the shade col13: Nitrogen
concentration in the leaf (g m\textasciicircum{}-2) col14: Vmax value as
depending on leaf nitrogen
\end{Value}
%
\begin{Author}\relax
Fernando E. Miguez
\end{Author}
%
\begin{References}\relax
Farquhar model 
site here \textasciitilde{}
\end{References}
%
\begin{Examples}
\begin{ExampleCode}
data(doy124)
tmp <- numeric(24)

for(i in 1:24){
   lai <- doy124[i,1]
   doy <- doy124[i,3]
   hr  <- doy124[i,4]
 solar <- doy124[i,5]
  temp <- doy124[i,6]
    rh <- doy124[i,7]
    ws <- doy124[i,8]

  tmp[i] <- c3CanA(lai,doy,hr,solar,temp,rh,ws)$CanopyAssim

}

plot(c(0:23),tmp,
            type='l',lwd=2,
            xlab='Hour',
            ylab=expression(paste('Canopy assimilation (Mg  ',
            ha^-2,' ',h^-1,')')))
\end{ExampleCode}
\end{Examples}
\inputencoding{utf8}
\HeaderA{c3photo}{Simulates C3 photosynthesis}{c3photo}
\keyword{models}{c3photo}
%
\begin{Description}\relax
Simulates coupled assimilation and stomatal conductance
based on Farquhar and Ball-Berry.
\end{Description}
%
\begin{Usage}
\begin{verbatim}
  c3photo(Qp, Tl, RH, vcmax = 100, jmax = 180, Rd = 1.1,
    Catm = 380, O2 = 210, b0 = 0.08, b1 = 5, theta = 0.7,
    StomWS = 1, ws = c("gs", "vmax"))
\end{verbatim}
\end{Usage}
%
\begin{Arguments}
\begin{ldescription}
\item[\code{Qp}] Quantum flux

\item[\code{Tl}] Leaf temperature

\item[\code{RH}] Relative humidity (fraction -- 0-1)

\item[\code{vcmax}] Maximum rate of carboxylation

\item[\code{jmax}] Maximum rate of electron transport

\item[\code{Rd}] Leaf dark respiration

\item[\code{Catm}] Atmospheric carbon dioxide.

\item[\code{O2}] Atmospheric Oxygen concentration (mmol/mol)

\item[\code{b0}] Intercept for the Ball-Berry model

\item[\code{b1}] Slope for the Ball-Berry model

\item[\code{theta}] Curvature parameter
\end{ldescription}
\end{Arguments}
%
\begin{Value}
A list
\end{Value}
%
\begin{Note}\relax
\textasciitilde{}\textasciitilde{}further notes\textasciitilde{}\textasciitilde{} \#\# Additional notes about assumptions
\end{Note}
%
\begin{Author}\relax
Fernando E. Miguez
\end{Author}
%
\begin{References}\relax
Farquhar (1980) Ball-Berry (1987)
\end{References}
%
\begin{SeeAlso}\relax
See Also \code{\LinkA{Opc3photo}{Opc3photo}}
\end{SeeAlso}
%
\begin{Examples}
\begin{ExampleCode}
## Testing the c3photo function
## First example: looking at the effect of changing Vcmax
Qps <- seq(0,2000,10)
Tls <- seq(0,55,5)
rhs <- c(0.7)
dat1 <- data.frame(expand.grid(Qp=Qps,Tl=Tls,RH=rhs))

res1 <- c3photo(dat1$Qp,dat1$Tl,dat1$RH) ## default Vcmax = 100
res2 <- c3photo(dat1$Qp,dat1$Tl,dat1$RH,vcmax=120)

## Plot comparing alpha 0.04 vs. 0.06 for a range of conditions
xyplot(res1$Assim + res2$Assim ~ Qp | factor(Tl) , data = dat1,
            type='l',col=c('blue','green'),lwd=2,
            ylab=expression(paste('Assimilation (',
                 mu,mol,' ',m^-2,' ',s^-1,')')),
             xlab=expression(paste('Quantum flux (',
                 mu,mol,' ',m^-2,' ',s^-1,')')),
            key=list(text=list(c('Vcmax 100','Vcmax 120')),
              lines=TRUE,col=c('blue','green'),lwd=2))

## Second example: looking at the effect of changing Jmax
## Plot comparing Jmax 300 vs. 100 for a range of conditions

res1 <- c3photo(dat1$Qp,dat1$Tl,dat1$RH) ## default Jmax = 300
res2 <- c3photo(dat1$Qp,dat1$Tl,dat1$RH,jmax=100)

xyplot(res1$Assim + res2$Assim ~ Qp | factor(Tl) , data = dat1,
           type='l',col=c('blue','green'),lwd=2,
            ylab=expression(paste('Assimilation (',
                 mu,mol,' ',m^-2,' ',s^-1,')')),
             xlab=expression(paste('Quantum flux (',
                 mu,mol,' ',m^-2,' ',s^-1,')')),
            key=list(text=list(c('Jmax 300','Jmax 100')),
              lines=TRUE,col=c('blue','green'),lwd=2))

## A/Ci curve

Ca <- seq(20,1000,length.out=50)
dat2 <- data.frame(Qp=rep(700,50), Tl=rep(25,50), rh=rep(0.7,50))
res1 <- c3photo(dat2$Qp, dat2$Tl, dat2$rh, Catm = Ca)
res2 <- c3photo(dat2$Qp, dat2$Tl, dat2$rh, Catm = Ca, vcmax = 70)

xyplot(res1$Assim ~ res1$Ci,
           lwd=2,
           panel = function(x,y,...){
                   panel.xyplot(x,y,type='l',col='blue',...)
                   panel.xyplot(res2$Ci,res2$Assim, type='l', col =
           'green',...)
           },
            ylab=expression(paste('Assimilation (',
                 mu,mol,' ',m^-2,' ',s^-1,')')))
\end{ExampleCode}
\end{Examples}
\inputencoding{utf8}
\HeaderA{c4photo}{Coupled photosynthesis-stomatal conductance simulation}{c4photo}
\keyword{models}{c4photo}
%
\begin{Description}\relax
The mathematical model is based on Collatz et al (1992)
(see References). Stomatal conductance is based on code
provided by Joe Berry.
\end{Description}
%
\begin{Usage}
\begin{verbatim}
  c4photo(Qp, Tl, RH, vmax = 39, alpha = 0.04, kparm = 0.7,
    theta = 0.83, beta = 0.93, Rd = 0.8, UPPERTEMP = 37.5,
    LOWERTEMP = 3, Catm = 380, b0 = 0.08, b1 = 3,
    StomWS = 1, ws = c("gs", "vmax"))
\end{verbatim}
\end{Usage}
%
\begin{Arguments}
\begin{ldescription}
\item[\code{Qp}] quantum flux (direct light), (\eqn{\mu}{}
mol \eqn{m^{-2}}{} \eqn{s^{-1}}{}).

\item[\code{Tl}] temperature of the leaf (Celsius).

\item[\code{RH}] relative humidity (as a fraction, i.e. 0-1).

\item[\code{vmax}] maximum carboxylation of Rubisco according to
the Collatz model.

\item[\code{alpha}] alpha parameter according to the Collatz
model. Initial slope of the response to Irradiance.

\item[\code{kparm}] k parameter according to the Collatz model.
Initial slope of the response to CO2.

\item[\code{theta}] theta parameter according to the Collatz
model. Curvature for light response.

\item[\code{beta}] beta parameter according to the Collatz
model. Curvature for response to CO2.

\item[\code{Rd}] Rd parameter according to the Collatz model.
Dark respiration.

\item[\code{UPPERTEMP}] 

\item[\code{LOWERTEMP}] 

\item[\code{Catm}] Atmospheric CO2 in ppm (or
\eqn{\mu}{}mol/mol).

\item[\code{b0}] intercept for the Ball-Berry stomatal
conductance model.

\item[\code{b1}] slope for the Ball-Berry stomatal conductance
model.

\item[\code{StomWS}] coefficient which controls the effect of
water stress on stomatal conductance and assimilation.

\item[\code{ws}] option to control whether the water stress
factor is applied to stomatal conductance ('gs') or to
Vmax ('vmax').
\end{ldescription}
\end{Arguments}
%
\begin{Value}
a \code{\LinkA{list}{list}} structure with components
\end{Value}
%
\begin{References}\relax
G. Collatz, M. Ribas-Carbo, J. Berry. (1992).  Coupled
photosynthesis-stomatal conductance model for leaves of
C4 plants. \emph{Australian Journal of Plant Physiology}
519--538.
\end{References}
%
\begin{SeeAlso}\relax
\code{\LinkA{eC4photo}{eC4photo}}
\end{SeeAlso}
%
\begin{Examples}
\begin{ExampleCode}
## Not run: 
     ## First example: looking at the effect of changing alpha
      Qps <- seq(0,2000,10)
      Tls <- seq(0,55,5)
      rhs <- c(0.7)
      dat1 <- data.frame(expand.grid(Qp=Qps,Tl=Tls,RH=rhs))
      res1 <- c4photo(dat1$Qp,dat1$Tl,dat1$RH) ## default alpha = 0.04
      res2 <- c4photo(dat1$Qp,dat1$Tl,dat1$RH,alpha=0.06)

     ## Plot comparing alpha 0.04 vs. 0.06 for a range of conditions
     xyplot(res1$Assim + res2$Assim ~ Qp | factor(Tl) , data = dat1,
            type='l',col=c('blue','green'),lwd=2,
            ylab=expression(paste('Assimilation (',
                 mu,mol,' ',m^-2,' ',s^-1,')')),
             xlab=expression(paste('Quantum flux (',
                 mu,mol,' ',m^-2,' ',s^-1,')')),
            key=list(text=list(c('alpha 0.04','alpha 0.06')),
              lines=TRUE,col=c('blue','green'),lwd=2))

     ## Second example: looking at the effect of changing vmax
     ## Plot comparing Vmax 39 vs. 50 for a range of conditions

      res1 <- c4photo(dat1$Qp,dat1$Tl,dat1$RH) ## default Vmax = 39
      res2 <- c4photo(dat1$Qp,dat1$Tl,dat1$RH,vmax=50)

     xyplot(res1$Assim + res2$Assim ~ Qp | factor(Tl) , data = dat1,
            type='l',col=c('blue','green'),lwd=2,
            ylab=expression(paste('Assimilation (',
                 mu,mol,' ',m^-2,' ',s^-1,')')),
             xlab=expression(paste('Quantum flux (',
                 mu,mol,' ',m^-2,' ',s^-1,')')),
            key=list(text=list(c('Vmax 39','Vmax 50')),
              lines=TRUE,col=c('blue','green'),lwd=2))

     ## Small effect of low RH on  Assim
      Qps <- seq(0,2000,10)
      Tls <- seq(0,55,5)
      rhs <- c(0.2,0.9)
      dat1 <- data.frame(expand.grid(Qp=Qps,Tl=Tls,RH=rhs))
      res1 <- c4photo(dat1$Qp,dat1$Tl,dat1$RH)
     # plot for Assimilation and two RH
      xyplot(res1$Assim ~ Qp | factor(Tl) , data = dat1,
             groups=RH, type='l',
             col=c('blue','green'),lwd=2,
             ylab=expression(paste('Assimilation (',
                 mu,mol,' ',m^-2,' ',s^-1,')')),
             xlab=expression(paste('Quantum flux (',
                 mu,mol,' ',m^-2,' ',s^-1,')')),
             key=list(text=list(c('RH 20%','RH 90%')),
                        lines=TRUE,col=c('blue','green'),
                        lwd=2))

    ## Effect of the previous runs on Stomatal conductance

    xyplot(res1$Gs ~ Qp | factor(Tl) , data = dat1,
           type='l', groups=RH,
           col=c('blue','green'),lwd=2,
           ylab=expression(paste('Stomatal Conductance (',
                           mu,mol,' ',m^-2,' ',s^-1,')')),
           xlab=expression(paste('Quantum flux (',
                           mu,mol,' ',m^-2,' ',s^-1,')')),
           key=list(text=list(c('RH 20%','RH 90%')),
                     lines=TRUE,col=c('blue','green'),
                     lwd=2))


## A Ci curve for the Collatz model
## Assuming constant values of Qp, Temp, and RH
## Notice the effect of the different kparm
## The loop is needed because the length of Ca
## should be the same as Qp

Ca <- seq(15,400,5)

res1 <- numeric(length(Ca))
res2 <- numeric(length(Ca))
for(i in 1:length(Ca)){
  res1[i] <- c4photo(1500,25,0.7,Catm=Ca[i])$Assim
  res2[i] <- c4photo(1500,25,0.7,Catm=Ca[i],kparm=0.8)$Assim
}

xyplot(res1 + res2 ~ Ca ,type='l',lwd=2,
       col=c('blue','green'),
     xlab=expression(paste(CO[2],' (ppm)')),
     ylab=expression(paste('Assimilation (',
         mu,mol,' ',m^-2,' ',s^-1,')')),
     key=list(text=list(c('kparm 0.7','kparm 0.8')),
                        lines=TRUE,col=c('blue','green'),
                        lwd=2))

## Effect of Reduction in Assimilation due to
## water stress

Qps <- seq(0,2000,10)
Tls <- seq(0,55,5)
rhs <- c(0.7)
dat1 <- data.frame(expand.grid(Qp=Qps,Tl=Tls,RH=rhs))
res1 <- c4photo(dat1$Qp,dat1$Tl,dat1$RH) ## default StomWS = 1 No stress
res2 <- c4photo(dat1$Qp,dat1$Tl,dat1$RH,StomWS=0.5)

## Plot comparing StomWS = 1 vs. 0.5 for a range of conditions
xyplot(res1$Assim + res2$Assim ~ Qp | factor(Tl) , data = dat1,
       type='l',col=c('blue','green'),lwd=2,
       ylab=expression(paste('Assimilation (',
           mu,mol,' ',m^-2,' ',s^-1,')')),
       xlab=expression(paste('Quantum flux (',
            mu,mol,' ',m^-2,' ',s^-1,')')),
       key=list(text=list(c('StomWS 1','StomWS 0.5')),
           lines=TRUE,col=c('blue','green'),lwd=2))


## Effect on Stomatal Conductance
## Plot comparing StomWS = 1 vs. 0.5 for a range of conditions
xyplot(res1$Gs + res2$Gs ~ Qp | factor(Tl) , data = dat1,
        type='l',col=c('blue','green'),lwd=2,
        ylab=expression(paste('Stomatal Conductance (mmol ',
          m^-2,' ',s^-1,')')),
        xlab=expression(paste('Quantum flux (',
          mu,mol,' ',m^-2,' ',s^-1,')')),
        key=list(text=list(c('StomWS 1','StomWS 0.5')),
           lines=TRUE,col=c('blue','green'),lwd=2))

## End(Not run)
\end{ExampleCode}
\end{Examples}
\inputencoding{utf8}
\HeaderA{CanA}{Simulates canopy assimilation}{CanA}
\aliasA{lnParms}{CanA}{lnParms}
\keyword{models}{CanA}
%
\begin{Description}\relax
It represents an integration of the photosynthesis
function \code{\LinkA{c4photo}{c4photo}}, canopy
evapo/transpiration and the multilayer canopy model
\code{\LinkA{sunML}{sunML}}.
\end{Description}
%
\begin{Usage}
\begin{verbatim}
  CanA(lai, doy, hr, solar, temp, rh, windspeed, lat = 40,
    nlayers = 8, kd = 0.1, StomataWS = 1, chi.l = 1,
    heightFactor = 3, photoControl = list(),
    lnControl = list(), units = c("kg/m2/hr", "Mg/ha/hr"))
\end{verbatim}
\end{Usage}
%
\begin{Arguments}
\begin{ldescription}
\item[\code{lai}] leaf area index.

\item[\code{doy}] day of the year, (1--365).

\item[\code{hr}] hour of the day, (0--23).

\item[\code{solar}] solar radiation (\eqn{\mu}{} mol
\eqn{m^{-2}}{} \eqn{s^{-1}}{}).

\item[\code{temp}] temperature (Celsius).

\item[\code{rh}] relative humidity (0--1).

\item[\code{windspeed}] wind speed (m \eqn{s^{-1}}{}).

\item[\code{lat}] latitude.

\item[\code{nlayers}] number of layers in the simulation of the
canopy (max allowed is 50).

\item[\code{kd}] Ligth extinction coefficient for diffuse
light.

\item[\code{StomataWS}] coefficient controlling the effect of
water stress on stomatal conductance and assimilation.

\item[\code{heightFactor}] Height Factor. Divide LAI by this
number to get the height of a crop.

\item[\code{photoControl}] list that sets the photosynthesis
parameters. See \code{\LinkA{BioGro}{BioGro}}.

\item[\code{lnControl}] list that sets the leaf nitrogen
parameters.

LeafN: Initial value of leaf nitrogen (g m-2).

kpLN: coefficient of decrease in leaf nitrogen during the
growing season. The equation is LN = iLeafN * exp(-kLN *
TTc).

lnFun: controls whether there is a decline in leaf
nitrogen with the depth of the canopy. 'none' means no
decline, 'linear' means a linear decline controlled by
the following two parameters.

lnb0: Intercept of the linear decline of leaf nitrogen in
the depth of the canopy.

lnb1: Slope of the linear decline of leaf nitrogen in the
depth of the canopy. The equation is 'vmax = leafN\_lay *
lnb1 + lnb0'.

\item[\code{units}] Whether to return units in kg/m2/hr or
Mg/ha/hr. This is typically run at hourly intervals, that
is why the hr is kept, but it could be used with data at
finer timesteps and then convert the results.
\end{ldescription}
\end{Arguments}
%
\begin{Value}
\code{\LinkA{list}{list}}

returns a list with several elements

CanopyAssim: hourly canopy assimilation (\eqn{kg
  m^{-2}}{} \eqn{hr^{-1}}{}) or canopy
assimilation (\eqn{Mg ha^{-1}}{} \eqn{hr^{-1}}{})

CanopyTrans: hourly canopy transpiration (\eqn{kg
  m^{-2}}{} \eqn{hr^{-1}}{}) or canopy
transpiration (\eqn{Mg ha^{-1}}{} \eqn{hr^{-1}}{})

CanopyCond: hourly canopy conductance (units ?)

TranEpen: hourly canopy transpiration according to Penman
(\eqn{kg m^{-2}}{} \eqn{hr^{-1}}{}) or canopy
transpiration according to Penman (\eqn{Mg ha^{-1}}{}
\eqn{hr^{-1}}{})

TranEpen: hourly canopy transpiration according to
Priestly (\eqn{kg m^{-2}}{} \eqn{hr^{-1}}{})
canopy transpiration according to Priestly (\eqn{Mg
  ha^{-1}}{} \eqn{hr^{-1}}{})

LayMat: hourly by Layer matrix containing details of the
calculations by layer (each layer is a row).  col1:
Direct Irradiance col2: Diffuse Irradiance col3: Leaf
area in the sun col4: Leaf area in the shade col5:
Transpiration of leaf area in the sun col6: Transpiration
of leaf area in the shade col7: Assimilation of leaf area
in the sun col8: Assimilation of leaf area in the shade
col9: Difference in temperature between the leaf and the
air (i.e. TLeaf - TAir) for leaves in sun.  col10:
Difference in temperature between the leaf and the air
(i.e. TLeaf - TAir) for leaves in shade.  col11: Stomatal
conductance for leaves in the sun col12: Stomatal
conductance for leaves in the shade col13: Nitrogen
concentration in the leaf (g m\textasciicircum{}-2) col14: Vmax value as
depending on leaf nitrogen
\end{Value}
%
\begin{Examples}
\begin{ExampleCode}
## Not run: 
data(doy124)
tmp <- numeric(24)

for(i in 1:24){
   lai <- doy124[i,1]
   doy <- doy124[i,3]
   hr  <- doy124[i,4]
 solar <- doy124[i,5]
  temp <- doy124[i,6]
    rh <- doy124[i,7]
    ws <- doy124[i,8]

  tmp[i] <- CanA(lai,doy,hr,solar,temp,rh,ws)$CanopyAssim

}

plot(c(0:23),tmp,
            type='l',lwd=2,
            xlab='Hour',
            ylab=expression(paste('Canopy assimilation (kg  ',
            m^-2,' ',h^-1,')')))


## End(Not run)
\end{ExampleCode}
\end{Examples}
\inputencoding{utf8}
\HeaderA{caneGro}{Simulation of cane, Growth, LAI, Photosynthesis and phenology}{caneGro}
\keyword{models}{caneGro}
%
\begin{Description}\relax
It takes weather data as input (hourly timesteps) and
several parameters and it produces phenology,
photosynthesis, LAI, etc.
\end{Description}
%
\begin{Usage}
\begin{verbatim}
caneGro(WetDat, day1 = NULL, dayn = NULL, timestep = 1, lat = 40,
  iRhizome = 7, irtl = 1e-04, canopyControl = list(),
  seneControl = list(), photoControl = list(), phenoControl = list(),
  soilControl = list(), nitroControl = list(), canephenoControl = list(),
  centuryControl = list(), managementControl = list(),
  frostControl = list())
\end{verbatim}
\end{Usage}
%
\begin{Arguments}
\begin{ldescription}
\item[\code{WetDat}] weather data as produced by the
\code{\LinkA{weach}{weach}} function.

\item[\code{plant.day}] Planting date (format 0-365)

\item[\code{emerge.day}] Emergence date (format 0-365)

\item[\code{harvest.day}] Harvest date (format 0-365)

\item[\code{plant.density}] Planting density (plants per meter
squared, default = 7)

\item[\code{timestep}] Simulation timestep, the default of 1
requires houlry weather data. A value of 3 would require
weather data every 3 hours.  This number should be a
divisor of 24.

\item[\code{lat}] latitude, default 40.

\item[\code{canopyControl}] List that controls aspects of the
canopy simulation. It should be supplied through the
\code{canopyParms} function.

\code{Sp} (specific leaf area) here the units are ha
\eqn{Mg^{-1}}{}.  If you have data in \eqn{m^2}{} of leaf per
kg of dry matter (e.g. 15) then divide by 10 before
inputting this coefficient.

\code{SpD} decrease of specific leaf area. Empirical
parameter. Default 0. example value (1.7e-3).

\code{nlayers} (number of layers of the canopy) Maximum
50. To increase the number of layers (more than 50) the
\code{C} source code needs to be changed slightly.

\code{kd} (extinction coefficient for diffuse light)
between 0 and 1.

\code{mResp} (maintenance respiration) a vector of length
2 with the first component for leaf and stem and the
second component for rhizome and root.

\item[\code{caneSeneControl}] List that controls aspects of
senescence simulation. It should be supplied through the
\code{caneSeneParms} function.

\code{senLeaf} Thermal time at which leaf senescence will
start.

\code{senStem} Thermal time at which stem senescence will
start.

\code{senRoot} Thermal time at which root senescence will
start.

\item[\code{photoControl}] List that controls aspects of
photosynthesis simulation. It should be supplied through
the \code{canePhotoParms} function.

\code{vmax} Vmax passed to the \code{\LinkA{c4photo}{c4photo}}
function.

\code{alpha} alpha parameter passed to the
\code{\LinkA{c4photo}{c4photo}} function.

\code{kparm} kparm parameter passed to the
\code{\LinkA{c4photo}{c4photo}} function.

\code{theta} theta parameter passed to the
\code{\LinkA{c4photo}{c4photo}} function.

\code{beta} beta parameter passed to the
\code{\LinkA{c4photo}{c4photo}} function.

\code{Rd} Rd parameter passed to the
\code{\LinkA{c4photo}{c4photo}} function.

\code{UPPERTEMP} UPPERTEMP parameter passed to the
\code{\LinkA{c4photo}{c4photo}} function.

\code{LOWERTEMP} LOWERTEMP parameter passed to the
\code{\LinkA{c4photo}{c4photo}} function.

\code{Catm} Catm parameter passed to the
\code{\LinkA{c4photo}{c4photo}} function.

\code{b0} b0 parameter passed to the
\code{\LinkA{c4photo}{c4photo}} function.

\code{b1} b1 parameter passed to the
\code{\LinkA{c4photo}{c4photo}} function.

\item[\code{canePhenoControl}] argument used to pass parameters
related to phenology characteristics 
\code{canePhenoControl} here\textasciitilde{}\textasciitilde{}

\item[\code{soilControl}] 
here\textasciitilde{}\textasciitilde{}

\item[\code{nitroControl}] 
here\textasciitilde{}\textasciitilde{}

\item[\code{centuryControl}] 
here\textasciitilde{}\textasciitilde{}
\end{ldescription}
\end{Arguments}
%
\begin{Details}\relax
The phenology follows the 'Corn Growth and Development'
Iowa State Publication. 
than the description above \textasciitilde{}\textasciitilde{}
\end{Details}
%
\begin{Value}
It currently returns a list with the following components

\begin{ldescription}
\item[\code{DayofYear}] Day of the year (0-365)

\item[\code{Hour}] Hour of the day (0-23)

\item[\code{TTTc}] Accumulated thermal time

\item[\code{PhenoStage}] Phenological stage of the crop

\item[\code{CanopyAssim}] Hourly canopy assimilation, (Mg
\eqn{ha^-1}{} ground \eqn{hr^-1}{}).

\item[\code{CanopyTrans}] Hourly canopy transpiration, (Mg
\eqn{ha^-1}{} ground \eqn{hr^-1}{}).

\item[\code{LAI}] Leaf Area Index
\end{ldescription}
\end{Value}
%
\begin{Author}\relax
Fernando E Miguez
\end{Author}
%
\begin{SeeAlso}\relax
\code{\LinkA{BioGro}{BioGro}} 
\code{\LinkA{help}{help}}, \textasciitilde{}\textasciitilde{}\textasciitilde{}
\end{SeeAlso}
\inputencoding{utf8}
\HeaderA{Century}{This function implements the Century model from Parton.}{Century}
\aliasA{CenturyC}{Century}{CenturyC}
\keyword{models}{Century}
%
\begin{Description}\relax
Calculates flows of soil organic carbon and nitrogen
based on the Century model. There are two versions one
written in R (Century) and one in C (CenturyC) they
should give the same result. The C version only runs at
weekly timesteps.
\end{Description}
%
\begin{Usage}
\begin{verbatim}
  Century(LeafL, StemL, RootL, RhizL, smoist, stemp,
    precip, leachWater, centuryControl = list(),
    verbose = FALSE)
\end{verbatim}
\end{Usage}
%
\begin{Arguments}
\begin{ldescription}
\item[\code{LeafL}] Leaf litter.

\item[\code{StemL}] Stem litter.

\item[\code{RootL}] Root litter.

\item[\code{RhizL}] Rhizome litter.

\item[\code{smoist}] Soil moisture.

\item[\code{stemp}] Soil temperature.

\item[\code{precip}] Precipitation.

\item[\code{leachWater}] Leached water.

\item[\code{centuryControl}] See \code{\LinkA{centuryParms}{centuryParms}}.

\item[\code{verbose}] Only used in the R version for debugging.

\item[\code{soilType}] See \code{\LinkA{showSoilType}{showSoilType}}.
\end{ldescription}
\end{Arguments}
%
\begin{Details}\relax
Most of the details can be found in the papers by Parton
et al. (1987, 1988, 1993)
\end{Details}
%
\begin{Value}
A list with,
\end{Value}
%
\begin{Author}\relax
Fernando E. Miguez
\end{Author}
%
\begin{References}\relax
\textasciitilde{}put references to the literature/web site here \textasciitilde{}
\end{References}
%
\begin{Examples}
\begin{ExampleCode}
Century(0,0,0,0,0.3,5,2,2)$Resp
Century(0,0,0,0,0.3,5,2,2)$MinN
\end{ExampleCode}
\end{Examples}
\inputencoding{utf8}
\HeaderA{CenturyC}{C version of the Century function}{CenturyC}
%
\begin{Description}\relax
C version of the Century function
\end{Description}
%
\begin{Usage}
\begin{verbatim}
CenturyC(LeafL, StemL, RootL, RhizL, smoist, stemp, precip, leachWater,
  centuryControl = list(), soilType = 0)
\end{verbatim}
\end{Usage}
%
\begin{Arguments}
\begin{ldescription}
\item[\code{LeafL}] Leaf litter.

\item[\code{StemL}] Stem litter.

\item[\code{RootL}] Root litter.

\item[\code{RhizL}] Rhizome litter.

\item[\code{smoist}] Soil moisture.

\item[\code{stemp}] Soil temperature.

\item[\code{precip}] Precipitation.

\item[\code{leachWater}] Leached water.

\item[\code{centuryControl}] See \code{\LinkA{centuryParms}{centuryParms}}.

\item[\code{soilType}] See \code{\LinkA{showSoilType}{showSoilType}}.
\end{ldescription}
\end{Arguments}
\inputencoding{utf8}
\HeaderA{CheckLeapYear}{This fuction checks if year is a leap year. If yes, then returns 1 or 0.}{CheckLeapYear}
%
\begin{Description}\relax
This fuction checks if year is a leap year. If yes, then
returns 1 or 0.
\end{Description}
%
\begin{Usage}
\begin{verbatim}
CheckLeapYear(year)
\end{verbatim}
\end{Usage}
%
\begin{Arguments}
\begin{ldescription}
\item[\code{year}] the year in question
\end{ldescription}
\end{Arguments}
\inputencoding{utf8}
\HeaderA{cmi0506}{Weather data}{cmi0506}
\keyword{datasets}{cmi0506}
%
\begin{Description}\relax
selected weather data corresponding to the Champaign
weather station (IL, U.S.). It has two years: 2005 and
2006. Dimensions: 730 by 11. The columns correspond to
the input necessary for the \code{\LinkA{weach}{weach}} function.
\end{Description}
%
\begin{Format}
data frame of dimensions 730 by 11.
\end{Format}
%
\begin{Source}\relax
\url{http://www.sws.uiuc.edu/warm/}
\end{Source}
\inputencoding{utf8}
\HeaderA{cmiWet}{Weather data}{cmiWet}
\keyword{datasets}{cmiWet}
%
\begin{Description}\relax
Layer data for evapo/transpiration. Simulated data to
show the result of the EvapoTrans function.
\end{Description}
%
\begin{Format}
data frame of dimensions 384 by 9.
\end{Format}
%
\begin{Details}\relax
lfClass: leaf class, 'sun' or 'shade'.

layer: layer in the canopy, 1 to 8.

hour: hour of the day, (0--23).

Rad: direct light.

Itot: total radiation.

Temp: air temperature, (Celsius).

RH: relative humidity, (0--1).

WindSpeed: wind speed, (m \eqn{s^{-1}}{}).

LAI: leaf area index.
\end{Details}
%
\begin{Source}\relax
simulated
\end{Source}
\inputencoding{utf8}
\HeaderA{CropGro}{Biomass crops growth simulation}{CropGro}
\aliasA{BioGro}{CropGro}{BioGro}
\aliasA{canopyParms}{CropGro}{canopyParms}
\aliasA{centuryParms}{CropGro}{centuryParms}
\aliasA{nitroParms}{CropGro}{nitroParms}
\aliasA{phenoParms}{CropGro}{phenoParms}
\aliasA{photoParms}{CropGro}{photoParms}
\aliasA{print.BioGro}{CropGro}{print.BioGro}
\aliasA{seneParms}{CropGro}{seneParms}
\aliasA{showSoilType}{CropGro}{showSoilType}
\aliasA{soilParms}{CropGro}{soilParms}
\aliasA{SoilType}{CropGro}{SoilType}
\keyword{models}{CropGro}
%
\begin{Description}\relax
Simulates dry biomass growth during an entire growing
season.  It represents an integration of the
photosynthesis function \code{\LinkA{c4photo}{c4photo}}, canopy
evapo/transpiration \code{\LinkA{CanA}{CanA}}, the multilayer
canopy model \code{\LinkA{sunML}{sunML}} and a dry biomass
partitioning calendar and senescence. It also considers,
carbon and nitrogen cycles and water and nitrogen
limitations.
\end{Description}
%
\begin{Usage}
\begin{verbatim}
  CropGro(WetDat, day1 = NULL, dayn = NULL, timestep = 1,
    lat = 40, iRhizome = 7, irtl = 1e-04,
    canopyControl = list(), seneControl = list(),
    photoControl = list(), phenoControl = list(),
    soilControl = list(), nitroControl = list(),
    SOMpoolsControl = list(),
    SOMAssignParmsControl = list(),
    GetCropCentStateVarParmsControl = list(),
    GetSoilTextureParmsControl = list(),
    GetBioCroToCropcentParmsControl = list(),
    GetErosionParmsControl = list(),
    GetC13ParmsControl = list(),
    GetLeachingParmsControl = list(),
    GetSymbNFixationParmsControl = list(),
    centuryControl = list())
\end{verbatim}
\end{Usage}
%
\begin{Arguments}
\begin{ldescription}
\item[\code{WetDat}] weather data as produced by the
\code{\LinkA{weach}{weach}} function.

\item[\code{day1}] first day of the growing season, (1--365).

\item[\code{dayn}] last day of the growing season, (1--365, but
larger than \code{day1}). See details.

\item[\code{timestep}] Simulation timestep, the default of 1
requires houlry weather data. A value of 3 would require
weather data every 3 hours.  This number should be a
divisor of 24.

\item[\code{lat}] latitude, default 40.

\item[\code{iRhizome}] initial dry biomass of the Rhizome (Mg
\eqn{ha^{-1}}{}).

\item[\code{irtl}] Initial rhizome proportion that becomes leaf.
This should not typically be changed, but it can be used
to indirectly control the effect of planting density.

\item[\code{canopyControl}] List that controls aspects of the
canopy simulation. It should be supplied through the
\code{canopyParms} function.

\code{Sp} (specific leaf area) here the units are ha
\eqn{Mg^{-1}}{}.  If you have data in \eqn{m^2}{} of leaf per
kg of dry matter (e.g. 15) then divide by 10 before
inputting this coefficient.

\code{nlayers} (number of layers of the canopy) Maximum
50. To increase the number of layers (more than 50) the
\code{C} source code needs to be changed slightly.

\code{kd} (extinction coefficient for diffuse light)
between 0 and 1.

\code{mResp} (maintenance respiration) a vector of length
2 with the first component for leaf and stem and the
second component for rhizome and root.

\item[\code{seneControl}] List that controls aspects of
senescence simulation. It should be supplied through the
\code{seneParms} function.

\code{senLeaf} Thermal time at which leaf senescence will
start.

\code{senStem} Thermal time at which stem senescence will
start.

\code{senRoot} Thermal time at which root senescence will
start.

\code{senRhizome} Thermal time at which rhizome
senescence will start.

\item[\code{photoControl}] List that controls aspects of
photosynthesis simulation. It should be supplied through
the \code{photoParms} function.

\code{vmax} Vmax passed to the \code{\LinkA{c4photo}{c4photo}}
function.

\code{alpha} alpha parameter passed to the
\code{\LinkA{c4photo}{c4photo}} function.

\code{kparm} kparm parameter passed to the
\code{\LinkA{c4photo}{c4photo}} function.

\code{theta} theta parameter passed to the
\code{\LinkA{c4photo}{c4photo}} function.

\code{beta} beta parameter passed to the
\code{\LinkA{c4photo}{c4photo}} function.

\code{Rd} Rd parameter passed to the
\code{\LinkA{c4photo}{c4photo}} function.

\code{Catm} Catm parameter passed to the
\code{\LinkA{c4photo}{c4photo}} function.

\code{b0} b0 parameter passed to the
\code{\LinkA{c4photo}{c4photo}} function.

\code{b1} b1 parameter passed to the
\code{\LinkA{c4photo}{c4photo}} function.

\item[\code{phenoControl}] List that controls aspects of the
crop phenology. It should be supplied through the
\code{phenoParms} function.

\code{tp1-tp6} thermal times which determine the time
elapsed between phenological stages. Between 0 and tp1 is
the juvenile stage. etc.

\code{kLeaf1-6} proportion of the carbon that is
allocated to leaf for phenological stages 1 through 6.

\code{kStem1-6} proportion of the carbon that is
allocated to stem for phenological stages 1 through 6.

\code{kRoot1-6} proportion of the carbon that is
allocated to root for phenological stages 1 through 6.

\code{kRhizome1-6} proportion of the carbon that is
allocated to rhizome for phenological stages 1 through 6.

\code{kGrain1-6} proportion of the carbon that is
allocated to grain for phenological stages 1 through 6.
At the moment only the last stage (i.e. 6 or
post-flowering) is allowed to be larger than zero. An
error will be returned if kGrain1-5 are different from
zero.

\item[\code{soilControl}] List that controls aspects of the soil
environment. It should be supplied through the
\code{soilParms} function.

\code{FieldC} Field capacity. This can be used to
override the defaults possible from the soil types (see
\code{\LinkA{showSoilType}{showSoilType}}).

\code{WiltP} Wilting point.  This can be used to override
the defaults possible from the soil types (see
\code{\LinkA{showSoilType}{showSoilType}}).

\code{phi1} Parameter which controls the spread of the
logistic function. See \code{\LinkA{wtrstr}{wtrstr}} for more
details.

\code{phi2} Parameter which controls the reduction of the
leaf area growth due to water stress. See
\code{\LinkA{wtrstr}{wtrstr}} for more details.

\code{soilDepth} Maximum depth of the soil that the roots
have access to (i.e. rooting depth).

\code{iWatCont} Initial water content of the soil the
first day of the growing season. It can be a single value
or a vector for the number of layers specified.

\code{soilType} Soil type, default is 6 (a more typical
soil would be 3). To see details use the function
\code{\LinkA{showSoilType}{showSoilType}}.

\code{soilLayer} Integer between 1 and 50. The default is
5. If only one soil layer is used the behavior can be
quite different.

\code{soilDepths} Intervals for the soil layers.

\code{wsFun} one of 'logistic','linear','exp' or 'none'.
Controls the method for the relationship between soil
water content and water stress factor.

\code{scsf} stomatal conductance sensitivity factor
(default = 1). This is an empirical coefficient that
needs to be adjusted for different species.

\code{rfl} Root factor lambda. A Poisson distribution is
used to simulate the distribution of roots in the soil
profile and this parameter can be used to change the
lambda parameter of the Poisson.

\code{rsec} Radiation soil evaporation coefficient.
Empirical coefficient used in the incidence of direct
radiation on soil evaporation.

\code{rsdf} Root soil depth factor. Empirical coefficient
used in calculating the depth of roots as a function of
root biomass.

\item[\code{nitroControl}] List that controls aspects of the
nitrogen environment. It should be supplied through the
\code{nitrolParms} function.

\code{iLeafN} initial value of leaf nitrogen (g m-2).

\code{kLN} coefficient of decrease in leaf nitrogen
during the growing season. The equation is LN = iLeafN *
(Stem + Leaf)\textasciicircum{}-kLN .

\code{Vmax.b1} slope which determines the effect of leaf
nitrogen on Vmax.

\code{alpha.b1} slope which controls the effect of leaf
nitrogen on alpha.

\item[\code{centuryControl}] List that controls aspects of the
Century model for carbon and nitrogen dynamics in the
soil. It should be supplied through the
\code{centuryParms} function.

\code{SC1-9} Soil carbon pools in the soil.  SC1:
Structural surface litter.  SC2: Metabolic surface
litter.  SC3: Structural root litter.  SC4: Metabolic
root litter.  SC5: Surface microbe.  SC6: Soil microbe.
SC7: Slow carbon.  SC8: Passive carbon.  SC9: Leached
carbon.

\code{LeafL.Ln} Leaf litter lignin content.

\code{StemL.Ln} Stem litter lignin content.

\code{RootL.Ln} Root litter lignin content.

\code{RhizomeL.Ln} Rhizome litter lignin content.

\code{LeafL.N} Leaf litter nitrogen content.

\code{StemL.N} Stem litter nitrogen content.

\code{RootL.N} Root litter nitrogen content.

\code{RhizomeL.N} Rhizome litter nitrogen content.

\code{Nfert} Nitrogen from a fertilizer source.

\code{iMinN} Initial value for the mineral nitrogen pool.

\code{Litter} Initial values of litter (leaf, stem, root,
rhizome).

\code{timestep} currently either week (default) or day.
\end{ldescription}
\end{Arguments}
%
\begin{Value}
a \code{\LinkA{list}{list}} structure with components
\end{Value}
%
\begin{Examples}
\begin{ExampleCode}
## Not run: 
data(weather05)

res0 <- BioGro(weather05)

plot(res0)

## Looking at the soil model

res1 <- BioGro(weather05, soilControl = soilParms(soilLayers = 6))
plot(res1, plot.kind='SW') ## Without hydraulic distribution
res2 <- BioGro(weather05, soilControl = soilParms(soilLayers = 6, hydrDist=TRUE))
plot(res2, plot.kind='SW') ## With hydraulic distribution


## Example of user defined soil parameters.
## The effect of phi2 on yield and soil water content

ll.0 <- soilParms(FieldC=0.37,WiltP=0.2,phi2=1)
ll.1 <- soilParms(FieldC=0.37,WiltP=0.2,phi2=2)
ll.2 <- soilParms(FieldC=0.37,WiltP=0.2,phi2=3)
ll.3 <- soilParms(FieldC=0.37,WiltP=0.2,phi2=4)

ans.0 <- BioGro(weather05,soilControl=ll.0)
ans.1 <- BioGro(weather05,soilControl=ll.1)
ans.2 <- BioGro(weather05,soilControl=ll.2)
ans.3 <-BioGro(weather05,soilControl=ll.3)

xyplot(ans.0$SoilWatCont +
       ans.1$SoilWatCont +
       ans.2$SoilWatCont +
       ans.3$SoilWatCont ~ ans.0$DayofYear,
       type='l',
       ylab='Soil water Content (fraction)',
       xlab='DOY')

## Compare LAI

xyplot(ans.0$LAI +
       ans.1$LAI +
       ans.2$LAI +
       ans.3$LAI ~ ans.0$DayofYear,
       type='l',
       ylab='Leaf Area Index',
       xlab='DOY')




## End(Not run)
\end{ExampleCode}
\end{Examples}
\inputencoding{utf8}
\HeaderA{doy124}{Weather data}{doy124}
\keyword{datasets}{doy124}
%
\begin{Description}\relax
Example for a given day of the year to illustrate the
\code{\LinkA{CanA}{CanA}} function.
\end{Description}
%
\begin{Format}
data frame of dimensions 24 by 8.
\end{Format}
%
\begin{Details}\relax
LAI: leaf area index.

year: year.

doy: 124 in this case.

hour: hour of the day, (0--23).

solarR: direct solar radiation.

DailyTemp.C: hourly air temperature (Celsius).

RH: relative humidity, (0--1).

WindSpeed: 4.1 m \eqn{s^{-1}}{} average daily value in this
case.
\end{Details}
%
\begin{Source}\relax
simulated
\end{Source}
\inputencoding{utf8}
\HeaderA{eC4photo}{C4 photosynthesis simulation (von Caemmerer model)}{eC4photo}
\keyword{models}{eC4photo}
%
\begin{Description}\relax
Simulation of C4 photosynthesis based on the equations
proposed by von Caemmerer (2000).  At this point
assimilation and stomatal conductance are not coupled and
although, for example a lower relative humidity will
lower stomatal conductance it will not affect
assimilation.  Hopefully, this will be improved in the
future.
\end{Description}
%
\begin{Usage}
\begin{verbatim}
  eC4photo(Qp, airtemp, rh, ca = 380, oa = 210, vcmax = 60,
    vpmax = 120, vpr = 80, jmax = 400)
\end{verbatim}
\end{Usage}
%
\begin{Arguments}
\begin{ldescription}
\item[\code{Qp}] quantum flux (\eqn{\mu}{} mol
\eqn{m^{-2}}{} \eqn{s^{-1}}{}).

\item[\code{airtemp}] air temperature (Celsius).

\item[\code{rh}] relative humidity in proportion (e.g. 0.7).

\item[\code{ca}] atmospheric carbon dioxide concentration (ppm
or \eqn{\mu}{}bar) (e.g. 380).

\item[\code{oa}] atmospheric oxygen concentration (mbar) (e.g.
210).

\item[\code{vcmax}] Maximum rubisco activity (\eqn{\mu}{}
mol \eqn{m^{-2}}{} \eqn{s^{-1}}{}).

\item[\code{vpmax}] Maximum PEP carboxylase activity
(\eqn{\mu}{} mol \eqn{m^{-2}}{}
\eqn{s^{-1}}{}).

\item[\code{vpr}] PEP regeneration rate (\eqn{\mu}{} mol
\eqn{m^{-2}}{} \eqn{s^{-1}}{}).

\item[\code{jmax}] Maximal electron transport rate
(\eqn{\mu}{}mol electrons \eqn{m^{-2}}{}
\eqn{s^{-1}}{}).
\end{ldescription}
\end{Arguments}
%
\begin{Details}\relax
The equations are taken from von Caemmerer (2000) for the
assimilation part and stomatal conductance is based on
FORTRAN code by Joe Berry (translated to C).
\end{Details}
%
\begin{Value}
results of call to C function eC4photo\_sym

a \code{\LinkA{list}{list}} structure with components
\end{Value}
%
\begin{References}\relax
Susanne von Caemmerer (2000) Biochemical Models of Leaf
Photosynthesis. CSIRO Publishing. (In particular chapter
4).
\end{References}
%
\begin{Examples}
\begin{ExampleCode}
## Not run: 
## A simple example for the use of eC4photo
## This is the model based on von Caemmerer
## First we can compare the effect of varying
## Jmax. Notice that this is different from
## varying alpha in the Collatz model

Qps <- seq(0,2000,10)
Tls <- seq(0,55,5)
rhs <- c(0.7)
dat1 <- data.frame(expand.grid(Qp=Qps,Tl=Tls,RH=rhs))
res1 <- eC4photo(dat1$Qp,dat1$Tl,dat1$RH)
res2 <- eC4photo(dat1$Qp,dat1$Tl,dat1$RH,jmax=700)

## Plot comparing Jmax 400 vs. 700 for a range of conditions
xyplot(res1$Assim + res2$Assim ~ Qp | factor(Tl) , data = dat1,
            type='l',col=c('blue','green'),lwd=2,
            ylab=expression(paste('Assimilation (',
                 mu,mol,' ',m^-2,' ',s^-1,')')),
             xlab=expression(paste('Quantum flux (',
                 mu,mol,' ',m^-2,' ',s^-1,')')),
            key=list(text=list(c('Jmax 400','Jmax 700')),
              lines=TRUE,col=c('blue','green'),lwd=2))

## Second example is the effect of varying Vcmax

Qps <- seq(0,2000,10)
Tls <- seq(0,35,5)
rhs <- 0.7
vcmax <- seq(0,40,5)
dat1 <- data.frame(expand.grid(Qp=Qps,Tl=Tls,RH=rhs,vcmax=vcmax))
res1 <- numeric(nrow(dat1))
for(i in 1:nrow(dat1)){
res1[i] <- eC4photo(dat1$Qp[i],dat1$Tl[i],dat1$RH[i],vcmax=dat1$vcmax[i])$Assim
}

## Plot comparing different Vcmax
cols <- rev(heat.colors(9))
xyplot(res1 ~ Qp | factor(Tl) , data = dat1,col=cols,
            groups=vcmax,
            type='l',lwd=2,
            ylab=expression(paste('Assimilation (',
                 mu,mol,' ',m^-2,' ',s^-1,')')),
             xlab=expression(paste('Quantum flux (',
                 mu,mol,' ',m^-2,' ',s^-1,')')),
            key=list(text=list(as.character(vcmax)),
              lines=TRUE,col=cols,lwd=2))



## End(Not run)
\end{ExampleCode}
\end{Examples}
\inputencoding{utf8}
\HeaderA{eCanA}{Simulates canopy assimilation (von Caemmerer model)}{eCanA}
\keyword{models}{eCanA}
%
\begin{Description}\relax
It represents an integration of the photosynthesis
function \code{\LinkA{eC4photo}{eC4photo}}, canopy
evapo/transpiration and the multilayer canopy model
\code{\LinkA{sunML}{sunML}}.
\end{Description}
%
\begin{Usage}
\begin{verbatim}
  eCanA(LAI, doy, hour, solarR, AirTemp, RH, WindS, Vcmax,
    Vpmax, Vpr, Jmax, Ca = 380, Oa = 210, StomataWS = 1)
\end{verbatim}
\end{Usage}
%
\begin{Arguments}
\begin{ldescription}
\item[\code{LAI}] leaf area index.

\item[\code{doy}] day of the year, (1--365).

\item[\code{hour}] hour of the day, (0--23).

\item[\code{solarR}] solar radiation (\eqn{\mu mol \; m^{-2} \;
  s^{-1}}{}).

\item[\code{AirTemp}] temperature (Celsius).

\item[\code{RH}] relative humidity (0--1).

\item[\code{WindS}] wind speed (\eqn{m \; s^{-1}}{}).

\item[\code{Vcmax}] Maximum rubisco activity (\eqn{\mu mol \;
  m^{-2} \; s^{-1}}{}).

\item[\code{Vpmax}] Maximum PEP carboxylase activity (\eqn{\mu
  mol \; m^{-2} \; s^{-1}}{}).

\item[\code{Vpr}] PEP regeneration rate (\eqn{\mu mol \; m^{-2}
  \; s^{-1}}{}).

\item[\code{Jmax}] Maximal electron transport rate
(\eqn{\mu}{}mol electrons \eqn{m^{-2}}{}
\eqn{s^{-1}}{}).

\item[\code{Ca}] atmospheric carbon dioxide concentration (ppm
or \eqn{\mu}{}bar) (e.g. 380).

\item[\code{Oa}] atmospheric oxygen concentration (mbar) (e.g.
210).

\item[\code{StomataWS}] Effect of water stress on assimilation.
\end{ldescription}
\end{Arguments}
%
\begin{Value}
\code{\LinkA{numeric}{numeric}}

returns a single value which is hourly canopy
assimilation (mol \eqn{m^{-2}}{} ground
\eqn{hr^{-1}}{})
\end{Value}
%
\begin{Examples}
\begin{ExampleCode}
## Not run: 
data(doy124)
tmp1 <- numeric(24)
for(i in 1:24){
  lai <- doy124[i,1]
  doy <- doy124[i,3]
  hr  <- doy124[i,4]
  solar <- doy124[i,5]
  temp <- doy124[i,6]
  rh <- doy124[i,7]/100
  ws <- doy124[i,8]

  tmp1[i] <- CanA(lai,doy,hr,solar,temp,rh,ws)
}

plot(c(0:23),tmp1,
     type='l',lwd=2,
     xlab='Hour',
     ylab=expression(paste('Canopy assimilation (mol  ',
     m^-2,' ',s^-1,')')))

## End(Not run)
\end{ExampleCode}
\end{Examples}
\inputencoding{utf8}
\HeaderA{EngWea94i}{Weather data corresponding to a paper by Clive Beale (see source).}{EngWea94i}
\keyword{datasets}{EngWea94i}
%
\begin{Description}\relax
Weather data with the precipitation column giving
precipitation plus irrigation.
\end{Description}
%
\begin{Format}
A data frame with 8760 observations on the following 8 variables.
\begin{description}
 \item[list('year')] a numeric vector\item[list('doy')] a
numeric vector\item[list('hour')] a numeric vector
\item[list('solarR')] a numeric vector\item[list('DailyTemp.C')] a
numeric vector\item[list('RH')] a numeric vector
\item[list('WindSpeed')] a numeric vector\item[list('precip')] a numeric
vector
\end{description}
\end{Format}
%
\begin{Details}\relax
\textasciitilde{}\textasciitilde{} If necessary, more details than the above \textasciitilde{}\textasciitilde{}
\end{Details}
%
\begin{Source}\relax
\textasciitilde{}\textasciitilde{} reference to a publication or URL from which the data
were obtained \textasciitilde{}\textasciitilde{}
\end{Source}
%
\begin{References}\relax
\textasciitilde{}\textasciitilde{} possibly secondary sources and usages \textasciitilde{}\textasciitilde{}
\end{References}
%
\begin{Examples}
\begin{ExampleCode}
data(EngWea94i)
## maybe str(EngWea94i) ; plot(EngWea94i) ...
\end{ExampleCode}
\end{Examples}
\inputencoding{utf8}
\HeaderA{EngWea94rf}{Weather data corresponding to a paper by Clive Beale (see source).}{EngWea94rf}
\keyword{datasets}{EngWea94rf}
%
\begin{Description}\relax
Weather data with the precipitation column giving
precipitation without irrigation.
\end{Description}
%
\begin{Format}
A data frame with 8760 observations on the following 8 variables.
\begin{description}
 \item[list('year')] a numeric vector\item[list('doy')] a
numeric vector\item[list('hour')] a numeric vector
\item[list('solarR')] a numeric vector\item[list('DailyTemp.C')] a
numeric vector\item[list('RH')] a numeric vector
\item[list('WindSpeed')] a numeric vector\item[list('precip')] a numeric
vector
\end{description}
\end{Format}
%
\begin{Details}\relax
\textasciitilde{}\textasciitilde{} If necessary, more details than the description above
\textasciitilde{}\textasciitilde{}
\end{Details}
%
\begin{Source}\relax
\textasciitilde{}\textasciitilde{} reference to a publication or URL from which the data
were obtained \textasciitilde{}\textasciitilde{}
\end{Source}
%
\begin{References}\relax
\textasciitilde{}\textasciitilde{} possibly secondary sources and usages \textasciitilde{}\textasciitilde{}
\end{References}
%
\begin{Examples}
\begin{ExampleCode}
data(EngWea94rf)
## maybe str(EngWea94rf) ; plot(EngWea94rf) ...
\end{ExampleCode}
\end{Examples}
\inputencoding{utf8}
\HeaderA{flow}{Returns values based on if kno is less than, equal to, or greater than three.}{flow}
%
\begin{Description}\relax
Returns Values for SC, fC, Resp, Kf, and MinN to be used in
the Century function
\end{Description}
%
\begin{Usage}
\begin{verbatim}
flow(SC, CNratio, A, Lc, Tm, resp, kno, Ks, verbose = FALSE)
\end{verbatim}
\end{Usage}
%
\begin{Arguments}
\begin{ldescription}
\item[\code{SC}] Soil Carbon

\item[\code{CNratio}] ratio of carbon to nitrogen

\item[\code{A}] effects of teperature and moisture

\item[\code{Lc}] See \code{\LinkA{FmLcFun}{FmLcFun}}

\item[\code{TM}] effect of soil texture on active SOM turnover

\item[\code{resp}] respiration

\item[\code{kno}] an integer value which determines

\item[\code{Ks}] flow constant

\item[\code{verbose}] Only used in the R version for debugging
\end{ldescription}
\end{Arguments}
\inputencoding{utf8}
\HeaderA{FmLcFun}{Returns values for Fm and Lc}{FmLcFun}
%
\begin{Description}\relax
A basic function designed to define the value for Fm and Lc
which are used in the century function.
\end{Description}
%
\begin{Usage}
\begin{verbatim}
FmLcFun(Lig, Nit)
\end{verbatim}
\end{Usage}
%
\begin{Arguments}
\begin{ldescription}
\item[\code{Lig}] lignin

\item[\code{Nit}] nitrogen
\end{ldescription}
\end{Arguments}
\inputencoding{utf8}
\HeaderA{fnpsvp}{The Goff Gratch equation from Smithsonian Tables, 1984. http://cires.colorado.edu/\textasciitilde{}voemel/vp.html}{fnpsvp}
%
\begin{Description}\relax
The Goff Gratch equation from Smithsonian Tables, 1984.
http://cires.colorado.edu/\textasciitilde{}voemel/vp.html
\end{Description}
%
\begin{Usage}
\begin{verbatim}
fnpsvp(Tkelvin)
\end{verbatim}
\end{Usage}
%
\begin{Arguments}
\begin{ldescription}
\item[\code{Tkelvin}] the abolute temperature
\end{ldescription}
\end{Arguments}
\inputencoding{utf8}
\HeaderA{idbp}{Initial Dry Biomass Partitioning Coefficients}{idbp}
\keyword{utilities}{idbp}
%
\begin{Description}\relax
Atempts to guess good initial vales for dry biomass
coefficients that can be passed to \code{BioGro},
\code{OpBioGro}, \code{constrOpBioGro}, or
\code{MCMCBioGro}.  It is very fragile.
\end{Description}
%
\begin{Usage}
\begin{verbatim}
  idbp(data, phenoControl = list())
\end{verbatim}
\end{Usage}
%
\begin{Arguments}
\begin{ldescription}
\item[\code{data}] Should have at least five columns with:
ThermalT, Stem, Leaf, Root, Rhizome and Grain.

\item[\code{phenoControl}] list that supplies mainly in this
case the thrmal time periods that delimit the
phenological stages,
\end{ldescription}
\end{Arguments}
%
\begin{Details}\relax
This function will not accept missing values. It can be
quite fragile and it is rather inflexible in what it
expects in terms of data.
\end{Details}
%
\begin{Value}
It returns a vector of length 25 suitable for
\code{BioGro}, \code{OpBioGro}, \code{constrOpBioGro}, or
\code{MCMCBioGro}.
\end{Value}
%
\begin{Note}\relax
It is highly recommended that the results of this
function are tested with \code{\LinkA{valid\_dbp}{valid.Rul.dbp}}.
\end{Note}
%
\begin{Author}\relax
Fernando E. Miguez
\end{Author}
%
\begin{SeeAlso}\relax
\code{\LinkA{valid\_dbp}{valid.Rul.dbp}}
\end{SeeAlso}
%
\begin{Examples}
\begin{ExampleCode}
## See ?OpBioGro
\end{ExampleCode}
\end{Examples}
\inputencoding{utf8}
\HeaderA{idbpm}{idpm}{idbpm}
%
\begin{Description}\relax
This estimates initial dry biomass partitioning
coefficients based on data for an annual grass
\end{Description}
%
\begin{Usage}
\begin{verbatim}
  idbpm(data, MaizePhenoControl = list())
\end{verbatim}
\end{Usage}
%
\begin{Arguments}
\begin{ldescription}
\item[\code{data}] data frame ThermalT Stem Leaf Root Grain LAI
1 0.211 0.00733 0.00104 0.00704 0 0.00119 611 280.000
1.08019 0.95531 0.11618 0 1.62350 1221 560.000 5.91862
2.08684 1.42061 0 3.54748 1831 747.000 10.16707 2.36378
2.53192 0 4.01843 2442 969.000 15.08485 2.42849 3.57410 0
4.12843 3052 1080.000 18.56392 2.46765 4.32115 0 4.19501
3662 1136.000 20.87121 2.04021 4.82178 0 3.46836 4273
1452.000 22.05770 0.89954 5.20210 0 1.52921

\item[\code{MaizePhenoControl}] 
\end{ldescription}
\end{Arguments}
%
\begin{Value}
vector of biomass pools
\end{Value}
%
\begin{Author}\relax
Fernando E. Miguez
\end{Author}
\inputencoding{utf8}
\HeaderA{LayET}{Weather data}{LayET}
\keyword{datasets}{LayET}
%
\begin{Description}\relax
Layer data for evapo/transpiration. Simulated data to
show the result of the EvapoTrans function.
\end{Description}
%
\begin{Format}
data frame of dimensions 384 by 9.
\end{Format}
%
\begin{Details}\relax
lfClass: leaf class, 'sun' or 'shade'.

layer: layer in the canopy, 1 to 8.

hour: hour of the day, (0--23).

Rad: direct light.

Itot: total radiation.

Temp: air temperature, (Celsius).

RH: relative humidity, (0--1).

WindSpeed: wind speed, (m \eqn{s^{-1}}{}).

LAI: leaf area index.
\end{Details}
%
\begin{Source}\relax
simulated
\end{Source}
\inputencoding{utf8}
\HeaderA{lightME}{Simulates the light macro environment}{lightME}
\keyword{models}{lightME}
%
\begin{Description}\relax
Simulates light macro environment based on latitude, day
of the year. Other coefficients can be adjusted.
\end{Description}
%
\begin{Usage}
\begin{verbatim}
  lightME(lat = 40, DOY = 190, t.d = 12, t.sn = 12,
    atm.P = 1e+05, alpha = 0.85)
\end{verbatim}
\end{Usage}
%
\begin{Arguments}
\begin{ldescription}
\item[\code{lat}] the latitude, default is 40 (Urbana, IL,
U.S.).

\item[\code{DOY}] the day of the year (1--365), default 190.

\item[\code{t.d}] time of the day in hours (0--23), default 12.

\item[\code{t.sn}] time of solar noon, default 12.

\item[\code{atm.P}] atmospheric pressure, default 1e5 (kPa).

\item[\code{alpha}] atmospheric transmittance, default 0.85.
\end{ldescription}
\end{Arguments}
%
\begin{Value}
a \code{\LinkA{list}{list}} structure with components
\end{Value}
%
\begin{Examples}
\begin{ExampleCode}
## Direct and diffuse radiation for DOY 190 and hours 0 to 23

res <- lightME(t.d=0:23)

xyplot(I.dir + I.diff ~ 0:23 , data = res,
type='o',xlab='hour',ylab='Irradiance')

xyplot(propIdir + propIdiff ~ 0:23 , data = res,
type='o',xlab='hour',ylab='Irradiance proportion')
\end{ExampleCode}
\end{Examples}
\inputencoding{utf8}
\HeaderA{MaizeGro}{Simulation of Maize, Growth, LAI, Photosynthesis and phenology}{MaizeGro}
\keyword{models}{MaizeGro}
%
\begin{Description}\relax
It takes weather data as input (hourly timesteps) and
several parameters and it produces phenology,
photosynthesis, LAI, etc.
\end{Description}
%
\begin{Usage}
\begin{verbatim}
MaizeGro(WetDat, plant.day = NULL, emerge.day = NULL, harvest.day = NULL,
  plant.density = 7, timestep = 1, lat = 40, canopyControl = list(),
  MaizeSeneControl = list(), photoControl = list(),
  MaizePhenoControl = list(), MaizeCAllocControl = list(),
  laiControl = list(), soilControl = list(), MaizeNitroControl = list(),
  centuryControl = list())
\end{verbatim}
\end{Usage}
%
\begin{Arguments}
\begin{ldescription}
\item[\code{WetDat}] weather data as produced by the
\code{\LinkA{weach}{weach}} function.

\item[\code{plant.day}] Planting date (format 0-365)

\item[\code{emerge.day}] Emergence date (format 0-365)

\item[\code{harvest.day}] Harvest date (format 0-365)

\item[\code{plant.density}] Planting density (plants per meter
squared, default = 7)

\item[\code{timestep}] Simulation timestep, the default of 1
requires houlry weather data. A value of 3 would require
weather data every 3 hours.  This number should be a
divisor of 24.

\item[\code{lat}] latitude, default 40.

\item[\code{canopyControl}] List that controls aspects of the
canopy simulation. It should be supplied through the
\code{canopyParms} function.

\code{Sp} (specific leaf area) here the units are ha
\eqn{Mg^{-1}}{}.  If you have data in \eqn{m^2}{} of leaf per
kg of dry matter (e.g. 15) then divide by 10 before
inputting this coefficient.

\code{SpD} decrease of specific leaf area. Empirical
parameter. Default 0. example value (1.7e-3).

\code{nlayers} (number of layers of the canopy) Maximum
50. To increase the number of layers (more than 50) the
\code{C} source code needs to be changed slightly.

\code{kd} (extinction coefficient for diffuse light)
between 0 and 1.

\code{mResp} (maintenance respiration) a vector of length
2 with the first component for leaf and stem and the
second component for rhizome and root.

\item[\code{MaizeSeneControl}] List that controls aspects of
senescence simulation. It should be supplied through the
\code{MaizeSeneParms} function.

\code{senLeaf} Thermal time at which leaf senescence will
start.

\code{senStem} Thermal time at which stem senescence will
start.

\code{senRoot} Thermal time at which root senescence will
start.

\item[\code{photoControl}] List that controls aspects of
photosynthesis simulation. It should be supplied through
the \code{MaizePhotoParms} function.

\code{vmax} Vmax passed to the \code{\LinkA{c4photo}{c4photo}}
function.

\code{alpha} alpha parameter passed to the
\code{\LinkA{c4photo}{c4photo}} function.

\code{kparm} kparm parameter passed to the
\code{\LinkA{c4photo}{c4photo}} function.

\code{theta} theta parameter passed to the
\code{\LinkA{c4photo}{c4photo}} function.

\code{beta} beta parameter passed to the
\code{\LinkA{c4photo}{c4photo}} function.

\code{Rd} Rd parameter passed to the
\code{\LinkA{c4photo}{c4photo}} function.

\code{UPPERTEMP} UPPERTEMP parameter passed to the
\code{\LinkA{c4photo}{c4photo}} function.

\code{LOWERTEMP} LOWERTEMP parameter passed to the
\code{\LinkA{c4photo}{c4photo}} function.

\code{Catm} Catm parameter passed to the
\code{\LinkA{c4photo}{c4photo}} function.

\code{b0} b0 parameter passed to the
\code{\LinkA{c4photo}{c4photo}} function.

\code{b1} b1 parameter passed to the
\code{\LinkA{c4photo}{c4photo}} function.

\item[\code{MaizePhenoControl}] argument used to pass parameters
related to phenology characteristics 
\code{MaizePhenoControl} here\textasciitilde{}\textasciitilde{}

\item[\code{soilControl}] 
here\textasciitilde{}\textasciitilde{}

\item[\code{nitroControl}] 
here\textasciitilde{}\textasciitilde{}

\item[\code{centuryControl}] 
here\textasciitilde{}\textasciitilde{}
\end{ldescription}
\end{Arguments}
%
\begin{Details}\relax
The phenology follows the 'Corn Growth and Development'
Iowa State Publication. 
than the description above \textasciitilde{}\textasciitilde{}
\end{Details}
%
\begin{Value}
It currently returns a list with the following components

\begin{ldescription}
\item[\code{DayofYear}] Day of the year (0-365)

\item[\code{Hour}] Hour of the day (0-23)

\item[\code{TTTc}] Accumulated thermal time

\item[\code{PhenoStage}] Phenological stage of the crop

\item[\code{CanopyAssim}] Hourly canopy assimilation, (Mg
\eqn{ha^-1}{} ground \eqn{hr^-1}{}).

\item[\code{CanopyTrans}] Hourly canopy transpiration, (Mg
\eqn{ha^-1}{} ground \eqn{hr^-1}{}).

\item[\code{LAI}] Leaf Area Index
\end{ldescription}
\end{Value}
%
\begin{Author}\relax
Fernando E Miguez
\end{Author}
%
\begin{SeeAlso}\relax
\code{\LinkA{BioGro}{BioGro}} 
\code{\LinkA{help}{help}}, \textasciitilde{}\textasciitilde{}\textasciitilde{}
\end{SeeAlso}
%
\begin{Examples}
\begin{ExampleCode}
data(weather05)
res <- MaizeGro(weather05, plant.day = 110, emerge.day = 120, harvest.day=300,
                  MaizePhenoControl = MaizePhenoParms(R6 = 2000))
\end{ExampleCode}
\end{Examples}
\inputencoding{utf8}
\HeaderA{MCMCBioGro}{Simulated annealing and MCMC  function}{MCMCBioGro}
\aliasA{print.MCMCBioGro}{MCMCBioGro}{print.MCMCBioGro}
\keyword{optimize}{MCMCBioGro}
%
\begin{Description}\relax
Simulated Annealing and Markov chain Monte Carlo for
estimating parameters for Biomass Growth
\end{Description}
%
\begin{Usage}
\begin{verbatim}
  MCMCBioGro(niter = 10, niter2 = 10, phen = 6,
    iCoef = NULL, saTemp = 5, coolSamp = 20, scale = 0.5,
    WetDat, data, day1 = NULL, dayn = NULL, timestep = 1,
    lat = 40, iRhizome = 7, irtl = 1e-04,
    canopyControl = list(), seneControl = list(),
    photoControl = list(), phenoControl = list(),
    soilControl = list(), nitroControl = list(),
    centuryControl = list(), sd = c(0.02, 1e-06))
\end{verbatim}
\end{Usage}
%
\begin{Arguments}
\begin{ldescription}
\item[\code{niter}] number of iterations for the simulated
annealing portion of the optimization.

\item[\code{niter2}] number of iterations for the Markov chain
Monte Carlo portion of the optimization.

\item[\code{phen}] Phenological stage being optimized.

\item[\code{iCoef}] initial coefficients for dry biomass
partitioning.

\item[\code{saTemp}] simulated annealing temperature.

\item[\code{coolSamp}] number of cooling samples.

\item[\code{scale}] scale parameter to control the standard
deviations.

\item[\code{WetDat}] weather data.

\item[\code{data}] observed data.

\item[\code{day1}] first day of the growing season.

\item[\code{dayn}] last day of the growing season.

\item[\code{timestep}] Timestep see \code{\LinkA{BioGro}{BioGro}}.

\item[\code{lat}] latitude.

\item[\code{iRhizome}] initial rhizome biomass.

\item[\code{irtl}] See \code{\LinkA{BioGro}{BioGro}}.

\item[\code{canopyControl}] See \code{\LinkA{canopyParms}{canopyParms}}.

\item[\code{seneControl}] See \code{\LinkA{seneParms}{seneParms}}.

\item[\code{photoControl}] See \code{\LinkA{photoParms}{photoParms}}.

\item[\code{phenoControl}] See \code{\LinkA{phenoParms}{phenoParms}}.

\item[\code{soilControl}] See \code{\LinkA{soilParms}{soilParms}}.

\item[\code{nitroControl}] See \code{\LinkA{nitroParms}{nitroParms}}.

\item[\code{centuryControl}] See \code{\LinkA{centuryParms}{centuryParms}}.

\item[\code{sd}] standard deviations for the parameters to be
optimized. The first (0.02) is for the positive dry
biomass partitioning coefficients. The second (1e-6) is
for the negative dry biomass partitioning coefficients.
\end{ldescription}
\end{Arguments}
%
\begin{Details}\relax
This function atempts to implement the simulated
annealing method for estimating parameters of a generic
C4 crop growth.

This function implements a hybrid algorithm where the
first portion is simulated annealing and the second
portion is a Markov chain Monte Carlo. The user controls
the number of iterations in each portion of the chain
with niter and niter2.
\end{Details}
%
\begin{Value}
An object of class MCMCBioGro consisting of a list with
23 components.  The easiest way of accessing the
information is with the print and plot methods.
\end{Value}
%
\begin{Note}\relax
The automatic method for guessing the last day of the
growing season differs slightly from that in
\code{BioGro}. To prevent error due to a shorter
simulated growing season than the observed one the method
in \code{MCMCBioGro} adds one day to the last day of the
growing season. Although the upper limit is fixed at 330.
\end{Note}
%
\begin{Author}\relax
Fernando E. Miguez

Fernando E. Miguez
\end{Author}
%
\begin{SeeAlso}\relax
See Also as \code{\LinkA{BioGro}{BioGro}}, \code{\LinkA{OpBioGro}{OpBioGro}}
and \code{\LinkA{constrOpBioGro}{constrOpBioGro}}.
\end{SeeAlso}
%
\begin{Examples}
\begin{ExampleCode}
## Not run: 

data(weather05)

## Some coefficients
pheno.ll <- phenoParms(kLeaf1=0.48,kStem1=0.47,kRoot1=0.05,kRhizome1=-1e-4,
                       kLeaf2=0.14,kStem2=0.65,kRoot2=0.21, kRhizome2=-1e-4,
                       kLeaf3=0.01, kStem3=0.56, kRoot3=0.13, kRhizome3=0.3,
                       kLeaf4=0.01, kStem4=0.56, kRoot4=0.13, kRhizome4=0.3,
                       kLeaf5=0.01, kStem5=0.56, kRoot5=0.13, kRhizome5=0.3,
                       kLeaf6=0.01, kStem6=0.56, kRoot6=0.13, kRhizome6=0.3)

system.time(ans <- BioGro(weather05, phenoControl = pheno.ll))

ans.dat <- as.data.frame(unclass(ans)[1:11])
sel.rows <- seq(1,nrow(ans.dat),400)
simDat <- ans.dat[sel.rows,c('ThermalT','Stem','Leaf','Root','Rhizome','Grain','LAI')]
plot(ans,simDat)

## Residual sum of squares before the optimization

ans0 <- BioGro(weather05)
RssBioGro(simDat,ans0)


op1.mc <- MCMCBioGro(phen=1, niter=200,niter2=200,
                     WetDat=weather05,
                     data=simDat)


plot(op1.mc)

plot(op1.mc, plot.kind='trace', subset = nams %in%
\t\t\t\tc('kLeaf_1','kStem_1','kRoot_1'))


## End(Not run)
\end{ExampleCode}
\end{Examples}
\inputencoding{utf8}
\HeaderA{MCMCc4photo}{Markov chain Monte Carlo for C4 photosynthesis parameters}{MCMCc4photo}
\aliasA{print.MCMCc4photo}{MCMCc4photo}{print.MCMCc4photo}
\keyword{optimize}{MCMCc4photo}
%
\begin{Description}\relax
This function implement Markov chain Monte Carlo methods
for the C4 photosynthesis model of Collatz et al.  The
chain is constructed using a Gaussian random walk. This
is definitely a beta version of this function and more
testing and improvements are needed. The value of this
function is in the possibility of examining the empirical
posterior distribution (i.e. vectors) of the vmax and
alpha parameters.  Notice that you will get different
results each time you run it.
\end{Description}
%
\begin{Usage}
\begin{verbatim}
  MCMCc4photo(data, niter = 20000, ivmax = 39,
    ialpha = 0.04, ikparm = 0.7, itheta = 0.83,
    ibeta = 0.93, iRd = 0.8, Catm = 380, b0 = 0.08, b1 = 3,
    StomWS = 1, ws = c("gs", "vmax"), scale = 1,
    sds = c(1, 0.005), prior = c(39, 10, 0.04, 0.02),
    UPPERTEMP = 37.5, LOWERTEMP = 3)
\end{verbatim}
\end{Usage}
%
\begin{Arguments}
\begin{ldescription}
\item[\code{data}] observed assimilation data, which should be a
data frame or matrix.  The first column should be
observed net assimilation rate (\eqn{\mu mol \; m^{-2} \;
  }{}\eqn{
  s^{-1}}{}).  The
second column should be the observed quantum flux
(\eqn{\mu mol \; m^{-2} \; }{}\eqn{ s^{-1}}{}).  The third column should be observed
temperature of the leaf (Celsius).  The fourth column
should be the observed relative humidity in proportion
(e.g. 0.7).

\item[\code{niter}] number of iterations to run the chain for
(default = 20000).

\item[\code{ivmax}] initial value for Vcmax (default = 39).

\item[\code{ialpha}] initial value for alpha (default = 0.04).

\item[\code{ikparm}] initial value for kparm (default = 0.7).
Not optimized at the moment.

\item[\code{itheta}] initial value for theta (default = 0.83).
Not optimized at the moment.

\item[\code{ibeta}] initial value for beta (default = 0.93). Not
optimized at the moment.

\item[\code{iRd}] initial value for dark respiration (default =
0.8).

\item[\code{Catm}] see \code{\LinkA{c4photo}{c4photo}} function.

\item[\code{b0}] see \code{\LinkA{c4photo}{c4photo}} function.

\item[\code{b1}] see \code{\LinkA{c4photo}{c4photo}} function.

\item[\code{StomWS}] see \code{\LinkA{c4photo}{c4photo}} function.

\item[\code{ws}] see \code{\LinkA{c4photo}{c4photo}} function.

\item[\code{scale}] This scale parameter controls the size of
the standard deviations which generate the moves in the
chain.

\item[\code{sds}] Finer control for the standard deviations of
the prior normals. The first element is for vmax and the
second for alpha.

\item[\code{prior}] Vector of length 4 with first element prior
mean for vmax, second element prior standard deviation
for vmax, third element prior mean for alpha and fourth
element prior standard deviation for alpha.
\end{ldescription}
\end{Arguments}
%
\begin{Value}
an object of class \code{MCMCc4photo} with components
\end{Value}
%
\begin{References}\relax
Brooks, Stephen. (1998). Markov chain Monte Carlo and its
application. The Statistician. 47, Part 1, pp. 69-100.
\end{References}
%
\begin{Examples}
\begin{ExampleCode}
## Not run: 
## Using Beale, Bint and Long (1996)
data(obsBea)

## Different starting values
resB1 <- MCMCc4photo(obsBea, 100000, scale=1.5)
resB2 <- MCMCc4photo(obsBea, 100000, ivmax=25, ialpha=0.1, scale=1.5)
resB3 <- MCMCc4photo(obsBea, 100000, ivmax=45, ialpha=0.02, scale=1.5)

## Use the plot function to examine results
plot(resB1,resB2,resB3)
plot(resB1,resB2,resB3,plot.kind='density',burnin=1e4)


## End(Not run)
\end{ExampleCode}
\end{Examples}
\inputencoding{utf8}
\HeaderA{MCMCEc4photo}{Markov chain Monte Carlo for C4 photosynthesis parameters}{MCMCEc4photo}
\keyword{optimize}{MCMCEc4photo}
%
\begin{Description}\relax
This function attempts to implement Markov chain Monte
Carlo methods for models with no likelihoods. In this
case it is done for the von Caemmerer C4 photosynthesis
model.  The method implemented is based on a paper by
Marjoram et al. (2003).  The method is described in
Miguez (2007). The chain is constructed using a Gaussian
random walk. This is definitely a beta version of this
function and more testing and improvements are needed.
The value of this function is in the possibility of
examining the empirical posterior distribution (i.e.
vectors) of the Vcmax and alpha parameters. Notice that
you will get different results each time you run it.
\end{Description}
%
\begin{Usage}
\begin{verbatim}
  MCMCEc4photo(obsDat, niter = 30000, iCa = 380, iOa = 210,
    iVcmax = 60, iVpmax = 120, iVpr = 80, iJmax = 400,
    thresh = 0.01, scale = 1)
\end{verbatim}
\end{Usage}
%
\begin{Arguments}
\begin{ldescription}
\item[\code{obsDat}] observed assimilation data, which should be
a data frame or matrix.  The first column should be
observed net assimilation rate (\eqn{\mu}{} mol
\eqn{m^{-2}}{} \eqn{s^{-1}}{}).  The second column should be
the observed quantum flux (\eqn{\mu}{} mol \eqn{m^{-2}}{}
\eqn{s^{-1}}{}).  The third column should be observed
temperature of the leaf (Celsius).  The fourth column
should be the observed relative humidity in proportion
(e.g. 0.7).

\item[\code{niter}] number of iterations to run the chain for
(default = 30000).

\item[\code{iCa}] CO2 atmospheric concentration (ppm or
\eqn{\mu}{}bar).

\item[\code{iOa}] O2 atmospheric concentration (mbar).

\item[\code{iVcmax}] initial value for Vcmax (default = 60).

\item[\code{iVpmax}] initial value for Vpmax (default = 120).

\item[\code{iVpr}] initial value for Vpr (default = 80).

\item[\code{iJmax}] initial value for Jmax (default = 400).

\item[\code{thresh}] this is a threshold that determines the
``convergence'' of the initial burn-in period. The
convergence of the whole chain can be evaluated by
running the model with different starting values for
Vcmax and alpha. The chain should convergence, but for
this, runs with similar thresholds should be compared.
This threshold reflects the fact that for any given data
the model will not be able to simulate it perfectly so it
represents a compromise between computability and fit.

\item[\code{scale}] This scale parameter controls the size of
the standard deviations which generate the moves in the
chain. Decrease it to increase the acceptance rate and
viceversa.
\end{ldescription}
\end{Arguments}
%
\begin{Value}
a \code{\LinkA{list}{list}} structure with components
\end{Value}
%
\begin{References}\relax
P. Marjoram, J. Molitor, V. Plagnol, S. Tavare, Markov
chain monte carlo without likelihoods, PNAS 100 (26)
(2003) 15324--15328.

Miguez (2007) Miscanthus x giganteus production:
meta-analysis, field study and mathematical modeling. PhD
Thesis. University of Illinois at Urbana-Champaign.
\end{References}
%
\begin{Examples}
\begin{ExampleCode}
## Not run: 
## This is an example for the MCMCEc4photo
## evaluating the convergence of the chain
## Notice that if a parameter does not seem
## to converge this does not mean that the method
## doesn't work. Careful examination is needed
## in order to evaluate the validity of the results
data(obsNaid)
res1 <- MCMCEc4photo(obsNaid,100000,thresh=0.007)
res1

## Run it a few more times
## and test the stability of the method
res2 <- MCMCEc4photo(obsNaid,100000,thresh=0.007)
res3 <- MCMCEc4photo(obsNaid,100000,thresh=0.007)

## Now plot it
plot(res1,res2,res3)
plot(res1,res2,res3,type='density')

## End(Not run)
\end{ExampleCode}
\end{Examples}
\inputencoding{utf8}
\HeaderA{mOpc3photo}{Multiple optimization of assimilation (or stomatal conductance) curves.}{mOpc3photo}
\keyword{optimize}{mOpc3photo}
%
\begin{Description}\relax
It is a wrapper for Opc3photo which allows for
optimization of multiple runs of curves (A/Q or A/Ci).
\end{Description}
%
\begin{Usage}
\begin{verbatim}
  mOpc3photo(data, ID = NULL, iVcmax = 100, iJmax = 180,
    iRd = 1.1, op.level = 1, curve.kind = c("Ci", "Q"),
    verbose = FALSE, ...)
\end{verbatim}
\end{Usage}
%
\begin{Arguments}
\begin{ldescription}
\item[\code{data}] should be a \code{data.frame} or
\code{matrix} with x columns

col 1: should be an ID for the different runs col 2:
measured assimilation (CO2 uptake) col 3: Incomming PAR
(photosynthetic active radiation) col 4: Leaf temperature
col 5: Relative humidity col 6: Intercellular CO2 (for
A/Ci curves) col 7: Reference CO2 level

\item[\code{ID}] optional argument to include ids. should be of
length equal to the number of runs.

\item[\code{iVcmax}] Single value or vector of length equal to
number of runs to supply starting values for the
optimization of \code{vcmax}.

\item[\code{iJmax}] Single value or vector of length equal to
number of runs to supply starting values for the
optimization of \code{jmax}.

\item[\code{iRd}] Single value or vector of length equal to
number of runs to supply starting values for the
optimization of \code{Rd}.

\item[\code{op.level}] Level 1 will optimize \code{Vcmax} and
\code{Jmax} and level 2 will optimize \code{Vcmax},
\code{Jmax} and \code{Rd}.

\item[\code{curve.kind}] Whether an A/Ci curve is being
optimized or an A/Q curve.

\item[\code{verbose}] Whether to print information about
progress.

\item[\code{...}] Additional arguments to be passed to
\code{\LinkA{Opc3photo}{Opc3photo}}
\end{ldescription}
\end{Arguments}
%
\begin{Details}\relax
Include more details about the data.
\end{Details}
%
\begin{Value}
an object of class 'mOpc3photo'

if op.level equals 1 best Vcmax, Jmax and convergence

if op.level equals 2 best Vcmax, Jmax, Rd and convergence



\end{Value}
%
\begin{Author}\relax
Fernando E. Miguez
\end{Author}
%
\begin{SeeAlso}\relax
See also \code{\LinkA{Opc3photo}{Opc3photo}} 
as \code{\LinkA{help}{help}}, \textasciitilde{}\textasciitilde{}\textasciitilde{}
\end{SeeAlso}
%
\begin{Examples}
\begin{ExampleCode}
data(simAssim)
simAssim <- cbind(simAssim[,1:6],Catm=simAssim[,10])
simAssim <- simAssim[simAssim[,1] < 11,]

plotAC(simAssim, trt.col=1)

op.all <- mOpc3photo(simAssim, op.level=1,
                      verbose=TRUE)

plot(op.all)
plot(op.all, parm='jmax')
\end{ExampleCode}
\end{Examples}
\inputencoding{utf8}
\HeaderA{mOpc4photo}{Multiple optimization of C4 photosynthesis.}{mOpc4photo}
\aliasA{plot.mOpc4photo}{mOpc4photo}{plot.mOpc4photo}
\aliasA{print.mOpc4photo}{mOpc4photo}{print.mOpc4photo}
\keyword{optimize}{mOpc4photo}
%
\begin{Description}\relax
Wrapper function that allows for optimization of multiple
A/Ci or A/Q curves.
\end{Description}
%
\begin{Usage}
\begin{verbatim}
  mOpc4photo(data, ID = NULL, ivmax = 39, ialpha = 0.04,
    iRd = 0.8, iupperT = 37.5, ilowerT = 3, ikparm = 0.7,
    itheta = 0.83, ibeta = 0.93, curve.kind = c("Q", "Ci"),
    op.level = 1, op.ci = FALSE, verbose = FALSE, ...)
\end{verbatim}
\end{Usage}
%
\begin{Arguments}
\begin{ldescription}
\item[\code{data}] observed assimilation data, which should be a
data frame or matrix. The first column should contain a
run or id.  The second column should be observed net
assimilation rate (\eqn{\mu}{} mol \eqn{m^{-2}}{}
\eqn{s^{-1}}{}).  The third column should be the observed
quantum flux (\eqn{\mu}{} mol \eqn{m^{-2}}{} \eqn{s^{-1}}{}).
The fourth column should be observed temperature of the
leaf (Celsius).  The fifth column should be the observed
relative humidity in proportion (e.g. 0.7). An optional
sixth column can contain atmospheric CO2.

\item[\code{ID}] Optional vector with an alternative ID tothe
one used in data for runs. The length shoudl be equal to
the number of runs.

\item[\code{ivmax}] Initial value for vmax. It can be a single
value or a vector of length equal to the number of runs.

\item[\code{ialpha}] Initial value for alpha. It can be a single
value or a vector of length equal to the number of runs.

\item[\code{iRd}] Initial value for vmax. It can be a single
value or a vector of length equal to the number of runs.

\item[\code{ikparm}] Initial value for vmax. It can be a single
value or a vector of length equal to the number of runs.

\item[\code{itheta}] Initial value for vmax. It can be a single
value or a vector of length equal to the number of runs.

\item[\code{ibeta}] Initial value for vmax. It can be a single
value or a vector of length equal to the number of runs.

\item[\code{curve.kind}] If \code{'Q'} a type of response which
mainly depends on light will be assumed. Typically used
to optimized light response curves or diurnals. Use
\code{'Ci'} for A/Ci curves (stomatal conductance could
also be optimized).

\item[\code{op.level}] optimization level. If equal to 1
\code{vmax} and \code{alpha} will be optimized. If 2,
\code{vmax}, \code{alpha} and \code{Rd} will be
optimized. If 3, \code{vmax}, \code{alpha}, \code{theta}
and \code{Rd} will be optimized.

\item[\code{op.ci}] Whether to optimize intercellular CO2.
Default is FALSE as 'fast-measured' light curves do not
provide reliable values of Ci.

\item[\code{verbose}] Whether to display output about
convergence of each run.

\item[\code{...}] Used to supply additional arguments to
\code{Opc4photo}.
\end{ldescription}
\end{Arguments}
%
\begin{Details}\relax
There are printing and plotting methods for the object
created by this function. The plotting function has an
argument that it is used to dsiplay either vmax or alpha
(i.e. \code{parm=c('vmax','alpha')}). In both cases the
optimized value plus confidence intervals will be
displayed for each run.
\end{Details}
%
\begin{Value}
An object of class \code{mOpc4photo} 
value returned 


\end{Value}
%
\begin{Author}\relax
Fernando E. miguez
\end{Author}
%
\begin{SeeAlso}\relax
\code{\LinkA{Opc4photo}{Opc4photo}} \code{\LinkA{c4photo}{c4photo}}
\code{\LinkA{optim}{optim}} 
\code{\LinkA{help}{help}}, \textasciitilde{}\textasciitilde{}\textasciitilde{}
\end{SeeAlso}
%
\begin{Examples}
\begin{ExampleCode}
data(simAssim)
\end{ExampleCode}
\end{Examples}
\inputencoding{utf8}
\HeaderA{obsBea}{Miscanthus assimilation field data}{obsBea}
\keyword{datasets}{obsBea}
%
\begin{Description}\relax
assimilation as measured in Beale, Bint and Long (1996)
in Miscanthus.  The first column is the observed net
assimilation rate (\eqn{\mu}{} mol \eqn{m^{-2}}{}
\eqn{s^{-1}}{}).  The second column is the observed quantum
flux (\eqn{\mu}{} mol \eqn{m^{-2}}{} \eqn{s^{-1}}{}). The third
column is the temperature (Celsius).Relative humidity was
not reported and thus was assumed to be 0.7.
\end{Description}
%
\begin{Format}
data frame of dimensions 27 by 4.
\end{Format}
%
\begin{Source}\relax
C. V. Beale, D. A. Bint, S. P. Long, Leaf photosynthesis
in the C4-grass miscanthus x giganteus, growing in the
cool temperate climate of southern England, \emph{J. Exp.
Bot.} 47 (2) (1996) 267--273.
\end{Source}
\inputencoding{utf8}
\HeaderA{obsBeaC}{Complete Miscanthus assimilation field data}{obsBeaC}
\keyword{datasets}{obsBeaC}
%
\begin{Description}\relax
Assimilation and stomatal conductance as measured in
Beale, Bint and Long (1996) in Miscanthus (missing data
are also included).  The first column is the date, the
second the hour. Columns 3 and 4 are assimilation and
stomatal conductance respectively.
\end{Description}
%
\begin{Format}
data frame of dimensions 35 by 6.
\end{Format}
%
\begin{Details}\relax
The third column is the observed net assimilation rate
(\eqn{\mu}{} mol \eqn{m^{-2}}{} \eqn{s^{-1}}{}).

The fifth column is the observed quantum flux (\eqn{\mu}{}
mol \eqn{m^{-2}}{} \eqn{s^{-1}}{}).

The sixth column is the temperature (Celsius).
\end{Details}
%
\begin{Source}\relax
C. V. Beale, D. A. Bint, S. P. Long, Leaf photosynthesis
in the C4-grass miscanthus x giganteus, growing in the
cool temperate climate of southern England, \emph{J. Exp.
Bot.} 47 (2) (1996) 267--273.
\end{Source}
\inputencoding{utf8}
\HeaderA{obsNaid}{Miscanthus assimilation data}{obsNaid}
\keyword{datasets}{obsNaid}
%
\begin{Description}\relax
assimilation as measured in Naidu et al. (2003) in
Miscanthus. The first column is the observed net
assimilation rate (\eqn{\mu}{} mol \eqn{m^{-2}}{}
\eqn{s^{-1}}{}) The second column is the observed quantum
flux (\eqn{\mu}{} mol \eqn{m^{-2}}{} \eqn{s^{-1}}{}) The third
column is the temperature (Celsius). The fourth column is
the observed relative humidity in proportion (e.g. 0.7).
\end{Description}
%
\begin{Format}
data frame of dimensions 16 by 4.
\end{Format}
%
\begin{Source}\relax
S. L. Naidu, S. P. Moose, A. K. AL-Shoaibi, C. A. Raines,
S. P. Long, Cold Tolerance of C4 photosynthesis in
Miscanthus x giganteus: Adaptation in Amounts and
Sequence of C4 Photosynthetic Enzymes, \emph{Plant
Physiol.} 132 (3) (2003) 1688--1697.
\end{Source}
\inputencoding{utf8}
\HeaderA{OpBioGro}{Optimization of dry biomass partitioning coefficients.}{OpBioGro}
\aliasA{constrOpBioGro}{OpBioGro}{constrOpBioGro}
\aliasA{summary.OpBioGro}{OpBioGro}{summary.OpBioGro}
\keyword{optimize}{OpBioGro}
%
\begin{Description}\relax
Optimizes dry biomass partitioning coefficients using
constrained optimization (see below).
\end{Description}
%
\begin{Usage}
\begin{verbatim}
  OpBioGro(phen = 1, iCoef = NULL, WetDat, data,
    day1 = NULL, dayn = NULL, timestep = 1, lat = 40,
    iRhizome = 7, irtl = 1e-04, canopyControl = list(),
    seneControl = list(), photoControl = list(),
    phenoControl = list(), soilControl = list(),
    nitroControl = list(), centuryControl = list(),
    op.method = c("optim", "nlminb"), verbose = FALSE, ...)
\end{verbatim}
\end{Usage}
%
\begin{Arguments}
\begin{ldescription}
\item[\code{phen}] integer taking values 1 through 6 which
indicate the phenological stage being optimized. If all
of the phenological stages need to be optimized then use
zero (0).

\item[\code{iCoef}] initial vector of size 24 for the dry
biomass partitioning coefficients.

\item[\code{WetDat}] Weather data.

\item[\code{data}] observed data.

\item[\code{day1}] first day of the growing season.

\item[\code{dayn}] last day of the growing season.

\item[\code{timestep}] see \code{\LinkA{BioGro}{BioGro}}

\item[\code{lat}] see \code{\LinkA{BioGro}{BioGro}}

\item[\code{iRhizome}] see \code{\LinkA{BioGro}{BioGro}}

\item[\code{irtl}] see \code{\LinkA{BioGro}{BioGro}}

\item[\code{canopyControl}] see \code{\LinkA{BioGro}{BioGro}}

\item[\code{seneControl}] see \code{\LinkA{BioGro}{BioGro}}

\item[\code{photoControl}] see \code{\LinkA{BioGro}{BioGro}}

\item[\code{phenoControl}] see \code{\LinkA{BioGro}{BioGro}}

\item[\code{soilControl}] see \code{\LinkA{BioGro}{BioGro}}

\item[\code{nitroControl}] see \code{\LinkA{BioGro}{BioGro}}

\item[\code{centuryControl}] see \code{\LinkA{BioGro}{BioGro}}

\item[\code{op.method}] Optimization method. Whether to use
optim or nlminb

\item[\code{verbose}] Displays additional information,
originally used for debugging.

\item[\code{...}] additional arguments passed to
\code{\LinkA{optim}{optim}} or \code{\LinkA{constrOptim}{constrOptim}}.
\end{ldescription}
\end{Arguments}
%
\begin{Details}\relax
The optimization is done over the \code{\LinkA{BioGro}{BioGro}}
function. The \code{OpBioGro} function is a wrapper for
\code{optim} and the \code{constrOpBioGro} is a wrapper
for \code{constrOptim}.
\end{Details}
%
\begin{Value}
\code{\LinkA{list}{list}} of class \code{OpBioGro} with
components
\end{Value}
%
\begin{Section}{Warning}
This function has not had enough testing.
\end{Section}
%
\begin{References}\relax
no references yet.
\end{References}
%
\begin{SeeAlso}\relax
\code{\LinkA{BioGro}{BioGro}} \code{\LinkA{constrOptim}{constrOptim}}
\code{\LinkA{optim}{optim}} \code{\LinkA{c4photo}{c4photo}}
\end{SeeAlso}
%
\begin{Examples}
\begin{ExampleCode}
## Not run: 

data(weather05)

## Some coefficients
pheno.ll <- phenoParms(kLeaf1=0.48,kStem1=0.47,kRoot1=0.05,kRhizome1=-1e-4,
                       kLeaf2=0.14,kStem2=0.65,kRoot2=0.21, kRhizome2=-1e-4,
                       kLeaf3=0.01, kStem3=0.56, kRoot3=0.13, kRhizome3=0.3,
                       kLeaf4=0.01, kStem4=0.56, kRoot4=0.13, kRhizome4=0.3,
                       kLeaf5=0.01, kStem5=0.56, kRoot5=0.13, kRhizome5=0.3,
                       kLeaf6=0.01, kStem6=0.56, kRoot6=0.13, kRhizome6=0.3)

system.time(ans <- BioGro(weather05, phenoControl = pheno.ll))

ans.dat <- as.data.frame(unclass(ans)[1:11])
sel.rows <- seq(1,nrow(ans.dat),length.out=8)
simDat <- ans.dat[sel.rows,c('ThermalT','Stem','Leaf','Root','Rhizome','Grain','LAI')]
plot(ans,simDat)

## Residual sum of squares before the optimization

ans0 <- BioGro(weather05)
RssBioGro(simDat,ans0)

## This will optimize only the first phenological stage
idb <- valid_dbp(idbp(simDat))
op1 <- OpBioGro(phen=0, WetDat=weather05, data = simDat, iCoef=idb)
## or
cop1 <- constrOpBioGro(phen=0, WetDat=weather05, data = simDat)


## End(Not run)
\end{ExampleCode}
\end{Examples}
\inputencoding{utf8}
\HeaderA{Opc3photo}{Optimize parameters of the C3 photosynthesis model.}{Opc3photo}
\aliasA{plot.Opc3photo}{Opc3photo}{plot.Opc3photo}
\aliasA{predict.Opc3photo}{Opc3photo}{predict.Opc3photo}
\aliasA{print.Opc3photo}{Opc3photo}{print.Opc3photo}
\keyword{optimize}{Opc3photo}
%
\begin{Description}\relax
Applies the \code{optim} function to C3 photosynthesis.
\end{Description}
%
\begin{Usage}
\begin{verbatim}
  Opc3photo(data, ivcmax = 100, ijmax = 180, iRd = 1.1,
    Catm = 380, O2 = 210, ib0 = 0.08, ib1 = 9.58,
    itheta = 0.7, op.level = 1,
    op.method = c("optim", "nlminb"),
    response = c("Assim", "StomCond"), level = 0.95,
    hessian = TRUE, curve.kind = c("Ci", "Q"),
    op.ci = FALSE, ...)
\end{verbatim}
\end{Usage}
%
\begin{Arguments}
\begin{ldescription}
\item[\code{data}] should be a \code{data.frame} or
\code{matrix} with x columns

col 1: measured assimilation (CO2 uptake) col 2:
Incomming PAR (photosynthetic active radiation) col 3:
Leaf temperature col 4: Relative humidity col 5:
Intercellular CO2 (for A/Ci curves) col 6: Reference CO2
level

\item[\code{ivcmax}] Initial value for \code{vcmax}.

\item[\code{ijmax}] Initial value for \code{jmax}.

\item[\code{iRd}] Initial value for \code{Rd}.

\item[\code{Catm}] Reference CO2.

\item[\code{O2}] Reference level of O2.

\item[\code{ib0}] Initial value for the intercept to the
Ball-Berry model.

\item[\code{ib1}] Initial value for the slope to the Ball-Berry
model.

\item[\code{itheta}] Initial value for \code{theta}.

\item[\code{op.level}] Level 1 will optimize \code{Vcmax} and
\code{Jmax} and level 2 will optimize \code{Vcmax},
\code{Jmax} and \code{Rd}.

\item[\code{op.method}] optimization method. At the moment only
optim is implemented.

\item[\code{response}] \code{'Assim'} for assimilation and
\code{'StomCond'} for stomatal conductance.

\item[\code{level}] Confidence interval level.

\item[\code{hessian}] Whether the hessian should be computed

\item[\code{curve.kind}] Whether an A/Ci curve is being
optimized or an A/Q curve.

\item[\code{op.ci}] whether to optimize intercellular CO2.

\item[\code{...}] Additioanl arguments to be passed to
\code{\LinkA{optim}{optim}}.
\end{ldescription}
\end{Arguments}
%
\begin{Value}
An object of class \code{Opc3photo}.

The following components can be extracted:
\end{Value}
%
\begin{Note}\relax
\textasciitilde{}\textasciitilde{}further notes\textasciitilde{}\textasciitilde{} Additional notes about the assumptions.
\end{Note}
%
\begin{Author}\relax
Fernando E. Miguez
\end{Author}
%
\begin{SeeAlso}\relax
See Also \code{\LinkA{mOpc3photo}{mOpc3photo}}
\end{SeeAlso}
%
\begin{Examples}
\begin{ExampleCode}
## Load fabricated data
data(simA100)
## Look at it
head(simA100)

op <- Opc3photo(simA100[,1:5],Catm=simA100[,9], op.level = 2)

## If faced with a difficult problem
## This can give starting values
op100 <- Opc3photo(simA100[,1:5],Catm=simA100[,9],
                   op.level = 2, method='SANN',
                   hessian=FALSE)

op100 <- Opc3photo(simA100[,1:5],Catm = simA100[,9],
                   op.level = 2,
                   ivcmax = op100$bestVmax,
                   ijmax = op100$bestJmax,
                   iRd = op100$bestRd)
op100
\end{ExampleCode}
\end{Examples}
\inputencoding{utf8}
\HeaderA{Opc4photo}{Optimization of C4 photosynthesis parameters}{Opc4photo}
\aliasA{plot.Opc4photo}{Opc4photo}{plot.Opc4photo}
\aliasA{plotAQ}{Opc4photo}{plotAQ}
\aliasA{predict.Opc4photo}{Opc4photo}{predict.Opc4photo}
\aliasA{print.Opc4photo}{Opc4photo}{print.Opc4photo}
\aliasA{print.summary.Opc4photo}{Opc4photo}{print.summary.Opc4photo}
\aliasA{summary.Opc4photo}{Opc4photo}{summary.Opc4photo}
\keyword{optimize}{Opc4photo}
%
\begin{Description}\relax
Optimization method for the Collatz C4 photosynthesis
model.  At the moment Vcmax and alpha are optimized only.
\end{Description}
%
\begin{Usage}
\begin{verbatim}
  Opc4photo(data, ivmax = 39, ialpha = 0.04, iRd = 0.8,
    ikparm = 0.7, itheta = 0.83, ibeta = 0.93, Catm = 380,
    ib0 = 0.08, ib1 = 3, iStomWS = 1, ws = c("gs", "vmax"),
    iupperT = 37.5, ilowerT = 3,
    response = c("Assim", "StomCond"),
    curve.kind = c("Q", "Ci"), op.level = 1, level = 0.95,
    hessian = TRUE, op.ci = FALSE, ...)
\end{verbatim}
\end{Usage}
%
\begin{Arguments}
\begin{ldescription}
\item[\code{data}] observed assimilation data, which should be a
data frame or matrix.  The first column should be
observed net assimilation rate (\eqn{\mu}{} mol
\eqn{m^{-2}}{} \eqn{s^{-1}}{}).  The second column should be
the observed quantum flux (\eqn{\mu}{} mol \eqn{m^{-2}}{}
\eqn{s^{-1}}{}).  The third column should be observed
temperature of the leaf (Celsius).  The fourth column
should be the observed relative humidity in proportion
(e.g. 0.7). An optional fifth column can contain
intercellular CO2. The reference level of CO2 should be
supplied to the function using the \code{Catm} argument.

\item[\code{ivmax}] initial value for Vcmax (default = 39).

\item[\code{ialpha}] initial value for alpha (default = 0.04).

\item[\code{iRd}] initial value for dark respiration (default =
0.8).

\item[\code{ikparm}] initial value for k (default = 0.7).

\item[\code{itheta}] initial value for theta (default = 0.83).

\item[\code{ibeta}] initial value for beta (default = 0.93).

\item[\code{Catm}] Atmospheric CO2 in ppm (or
\eqn{\mu}{}mol/mol).

\item[\code{ib0}] initial value for the Ball-Berry intercept.

\item[\code{ib1}] initial value for the Ball-Berry slope.

\item[\code{iStomWS}] initial value for the stomata water stress
factor.

\item[\code{ws}] \code{ws} flag. See \code{\LinkA{c4photo}{c4photo}}.

\item[\code{response}] Use \code{'Assim'} if you want to
optimize assimilation data and use \code{'StomCond'} if
you want to optimize stomatal conductance data. The
parameters optimized will be different.

\item[\code{curve.kind}] If \code{'Q'} a type of response which
mainly depends on light will be assumed. Typically used
to optimized light response curves or diurnals. Use
\code{'Ci'} for A/Ci curves (stomatal conductance could
also be optimized).

\item[\code{op.level}] optimization level. If equal to 1
\code{vmax} and \code{alpha} will be optimized. If 2,
\code{vmax}, \code{alpha} and \code{Rd} will be
optimized. If 3, \code{vmax}, \code{alpha}, \code{theta}
and \code{Rd} will be optimized.

\item[\code{level}] level for the confidence intervals.

\item[\code{hessian}] Whether the hessian matrix should be
computed. See \code{\LinkA{optim}{optim}}.

\item[\code{op.ci}] Whether to optimize intercellular CO2.
Default is FALSE as 'fast-measured' light curves do not
provide reliable values of Ci.

\item[\code{list()}] Additional arguments passed to the
\code{\LinkA{optim}{optim}} in particular if a lower or upper
bound is desired this could be achieved by adding
\code{lower=c(0,0)} this will impose a lower bound on
\code{vmax} and \code{alpha} of zero so preventing
negative values from being returned. When the lower
argument is added the optimization method changes from
Nelder-Mead to BFGS.
\end{ldescription}
\end{Arguments}
%
\begin{Value}
An object of class \code{Opc4photo} a \code{\LinkA{list}{list}}
with components

If \code{op.level} 2 \code{bestRd} will also be supplied.
If \code{op.level} 3 \code{theta} and \code{bestRd} will
also be supplied.

If \code{op.level} 2 \code{ciRd} will also be supplied.
If \code{op.level} 3 \code{ciTheta} and \code{ciRd} will
also be supplied.
\end{Value}
%
\begin{SeeAlso}\relax
\code{\LinkA{c4photo}{c4photo}} \code{\LinkA{optim}{optim}}
\end{SeeAlso}
%
\begin{Examples}
\begin{ExampleCode}
data(aq)
## Select data for a single AQ optimization
aqd <- data.frame(aq[aq[,1] == 6,-c(1:2)],Catm=400)
res <- Opc4photo(aqd, Catm=aqd$Catm)
res

plot(res, plot.kind = 'OandF', type='o')
\end{ExampleCode}
\end{Examples}
\inputencoding{utf8}
\HeaderA{OpEC4photo}{Optimization of C4 photosynthesis parameters (von Caemmerer model)}{OpEC4photo}
\aliasA{summary.OpEC4photo}{OpEC4photo}{summary.OpEC4photo}
\keyword{optimize}{OpEC4photo}
%
\begin{Description}\relax

Optimization method for the von Caemmerer C4 photosynthesis model.

\end{Description}
%
\begin{Usage}
\begin{verbatim}
OpEC4photo(obsDat,iVcmax=60,iVpmax=120,iVpr=80,iJmax=400,co2=380,o2=210,level=0.95)
\end{verbatim}
\end{Usage}
%
\begin{Arguments}
\begin{ldescription}

\item[\code{obsDat}]  observed assimilation data, which should be a data
frame or matrix.
The first column should be observed net
assimilation rate  (\eqn{\mu}{} mol \eqn{m^{-2}}{} \eqn{s^{-1}}{}).
The second column should be the observed
quantum flux  (\eqn{\mu}{} mol \eqn{m^{-2}}{} \eqn{s^{-1}}{}).
The third column should be observed temperature of the leaf
(Celsius).
The fourth column should be the observed relative humidity
in proportion (e.g. 0.7).

\item[\code{iVcmax}] initial value for Vcmax (default = 60).
\item[\code{iVpmax}] initial value for Vpmax (default = 120).
\item[\code{iVpr}] initial value for Vpr (default = 80).
\item[\code{iJmax}] initial value for Jmax (default = 400).
\item[\code{co2}] atmospheric CO2 concentration (ppm), default = 380.
\item[\code{o2}] atmospheric O2 concentration (mmol/mol), default = 210.
\item[\code{level}] level for the confidence intervals.
\end{ldescription}
\end{Arguments}
%
\begin{Value}

an object of class \code{\LinkA{OpEC4photo}{OpEC4photo}}. Notice that these are
the new-style S4 classes. 
\end{Value}
%
\begin{Examples}
\begin{ExampleCode}
data(obsNaid)
## These data are from Naidu et al. (2003)
## in the correct format
res <- OpEC4photo(obsNaid)
## Other example using Beale, Bint and Long (1996)
data(obsBea)
resB <- OpEC4photo(obsBea)
\end{ExampleCode}
\end{Examples}
\inputencoding{utf8}
\HeaderA{OpMaizeGro}{Optimization of dry biomass partitioning coefficients.}{OpMaizeGro}
%
\begin{Description}\relax
Optimizes dry biomass partitioning coefficients using
constrained optimization (see below).
\end{Description}
%
\begin{Usage}
\begin{verbatim}
OpMaizeGro(phen = 1, iCoef = NULL, cTT, WetDat, data, plant.day = NULL,
  emerge.day = NULL, harvest.day = NULL, plant.density = 7,
  timestep = 1, lat = 40, canopyControl = list(),
  MaizeSeneControl = list(), photoControl = list(),
  MaizePhenoControl = list(), MaizeCAllocControl = list(),
  laiControl = list(), soilControl = list(), MaizeNitroControl = list(),
  centuryControl = list(), op.method = c("optim", "nlminb"),
  verbose = FALSE, ...)
\end{verbatim}
\end{Usage}
%
\begin{Arguments}
\begin{ldescription}
\item[\code{phen}] integer taking values 1 through 6 which
indicate the phenological stage being optimized. If all
of the phenological stages need to be optimized then use
zero (0).

\item[\code{iCoef}] initial vector of size 24 for the dry
biomass partitioning coefficients.

\item[\code{cTT}] 

\item[\code{WetDat}] Weather data.

\item[\code{data}] observed data.

\item[\code{plant.day}] 

\item[\code{emerge.day}] 

\item[\code{harvest.day}] 

\item[\code{plant.density}] 

\item[\code{canopyControl}] see \code{\LinkA{MaizeGro}{MaizeGro}}

\item[\code{seneControl}] see \code{\LinkA{MaizeGro}{MaizeGro}}

\item[\code{photoControl}] see \code{\LinkA{MaizeGro}{MaizeGro}}

\item[\code{phenoControl}] see \code{\LinkA{MaizeGro}{MaizeGro}}

\item[\code{soilControl}] see \code{\LinkA{MaizeGro}{MaizeGro}}

\item[\code{nitroControl}] see \code{\LinkA{MaizeGro}{MaizeGro}}

\item[\code{centuryControl}] see \code{\LinkA{MaizeGro}{MaizeGro}}

\item[\code{op.method}] Optimization method. Whether to use
optim or nlminb

\item[\code{verbose}] Displays additional information,
originally used for debugging.

\item[\code{...}] additional arguments passed to
\code{\LinkA{optim}{optim}} or \code{\LinkA{constrOptim}{constrOptim}}.
\end{ldescription}
\end{Arguments}
%
\begin{Details}\relax
The optimization is done over the \code{\LinkA{MaizeGro}{MaizeGro}}
function. The \code{OpMaizeGro} function is a wrapper for
\code{optim} and the \code{constrOpBioGro} is a wrapper for
\code{constrOptim}.
\end{Details}
\inputencoding{utf8}
\HeaderA{plot.BioGro}{Plotting function for BioGro objects}{plot.BioGro}
\keyword{hplot}{plot.BioGro}
%
\begin{Description}\relax
By default it plots stem, leaf, root, rhizome and LAI for
a \code{BioGro} object. Optionally, the observed data can
be plotted.
\end{Description}
%
\begin{Usage}
\begin{verbatim}
  plot.BioGro(x, obs = NULL, stem = TRUE, leaf = TRUE,
    root = TRUE, rhizome = TRUE, LAI = TRUE, grain = TRUE,
    xlab = NULL, ylab = NULL, pch = 21, lty = 1, lwd = 1,
    col = c("blue", "green", "red", "magenta", "black", "purple"),
    x1 = 0.1, y1 = 0.8, plot.kind = c("DB", "SW"), ...)
\end{verbatim}
\end{Usage}
%
\begin{Arguments}
\begin{ldescription}
\item[\code{x}] \code{\LinkA{BioGro}{BioGro}} object.

\item[\code{obs}] optional observed data object (format
following the \code{\LinkA{OpBioGro}{OpBioGro}} function .

\item[\code{stem}] whether to plot simulated stem (default =
TRUE).

\item[\code{leaf}] whether to plot simulated leaf (default =
TRUE).

\item[\code{root}] whether to plot simulated root (default =
TRUE).

\item[\code{rhizome}] whether to plot simulated rhizome (default
= TRUE).

\item[\code{grain}] whether to plot simulated grain (default =
TRUE).

\item[\code{LAI}] whether to plot simulated LAI (default =
TRUE).

\item[\code{pch}] point character.

\item[\code{lty}] line type.

\item[\code{lwd}] line width.

\item[\code{col}] Control of colors.

\item[\code{x1}] position of the legend. x coordinate (0-1).

\item[\code{y1}] position of the legend. y coordinate (0-1).

\item[\code{plot.kind}] DB plots dry biomass and SW plots soil
water.

\item[\code{...}] Optional arguments.
\end{ldescription}
\end{Arguments}
%
\begin{Details}\relax
This function uses internally
\code{\LinkA{xyplot}{xyplot}} in the 'lattice' package.
\end{Details}
%
\begin{SeeAlso}\relax
\code{\LinkA{BioGro}{BioGro}} \code{\LinkA{OpBioGro}{OpBioGro}}
\end{SeeAlso}
\inputencoding{utf8}
\HeaderA{plot.MaizeGro}{Plotting function for MaizeGro objects}{plot.MaizeGro}
\keyword{hplot}{plot.MaizeGro}
%
\begin{Description}\relax
By default it plots stem, leaf, root, rhizome and LAI for a
\code{MaizeGro} object. Optionally, the observed data can
be plotted.
\end{Description}
%
\begin{Usage}
\begin{verbatim}
## S3 method for class 'MaizeGro'
plot(x, obs = NULL, stem = TRUE, leaf = TRUE,
  root = TRUE, LAI = TRUE, grain = TRUE, xlab = NULL, ylab = NULL,
  pch = 21, lty = 1, lwd = 1, col = c("blue", "green", "red", "black",
  "purple"), x1 = 0.1, y1 = 0.8, plot.kind = c("DB", "SW", "LAI",
  "pheno"), ...)
\end{verbatim}
\end{Usage}
%
\begin{Arguments}
\begin{ldescription}
\item[\code{x}] \code{\LinkA{MaizeGro}{MaizeGro}} object.

\item[\code{obs}] optional observed data object (format
following the \code{\LinkA{OpMaizeGro}{OpMaizeGro}} function .

\item[\code{stem}] whether to plot simulated stem (default =
TRUE).

\item[\code{leaf}] whether to plot simulated leaf (default =
TRUE).

\item[\code{root}] whether to plot simulated root (default =
TRUE).

\item[\code{rhizome}] whether to plot simulated rhizome (default
= TRUE).

\item[\code{grain}] whether to plot simulated grain (default =
TRUE).

\item[\code{LAI}] whether to plot simulated LAI (default =
TRUE).

\item[\code{pch}] point character.

\item[\code{lty}] line type.

\item[\code{lwd}] line width.

\item[\code{col}] Control of colors.

\item[\code{x1}] position of the legend. x coordinate (0-1).

\item[\code{y1}] position of the legend. y coordinate (0-1).

\item[\code{plot.kind}] DB plots dry biomass and SW plots soil
water.

\item[\code{...}] Optional arguments.
\end{ldescription}
\end{Arguments}
%
\begin{Details}\relax
This function uses internally \code{\LinkA{xyplot}{xyplot}}
in the 'lattice' package.
\end{Details}
%
\begin{SeeAlso}\relax
\code{\LinkA{MaizeGro}{MaizeGro}} \code{\LinkA{OpMaizeGro}{OpMaizeGro}}
\end{SeeAlso}
\inputencoding{utf8}
\HeaderA{plot.MCMCBioGro}{ Plotting function fo the MCMCBioGro class}{plot.MCMCBioGro}
\keyword{hplot}{plot.MCMCBioGro}
%
\begin{Description}\relax
Powerful plotting function to make a variety of plots regarding the
\code{MCMCBioGro} class (output). It plots the residual sum of square
progression, observed vs. fitted, residuals vs. fitted, trace of the
coefficients and density.
\end{Description}
%
\begin{Usage}
\begin{verbatim}
## S3 method for class 'MCMCBioGro'
plot(x, x2 = NULL, x3 = NULL, plot.kind = c("rss", "OF", 
    "RF", "OFT", "trace", "density"), type = c("l", "p"), coef = 1, 
    cols = c("blue", "green", "red", "magenta", "black", "purple"), ...) 
\end{verbatim}
\end{Usage}
%
\begin{Arguments}
\begin{ldescription}
\item[\code{x}]  Object of class \code{MCMCBioGro} 
\item[\code{x2}]  Optional object of class \code{MCMCBioGro} 
\item[\code{x3}]  Optional object of class \code{MCMCBioGro}
\item[\code{plot.kind}]  Kind of plot. See details. 
\item[\code{type}]  Point of line as in \code{xyplot} 
\item[\code{coef}]  Whether to plot dry biomass partitioning coeficients (2)
or Vmax and alpha (1).  
\item[\code{cols}]  Colors. Modify if they don't suit you.
\item[\code{...}]  Additional arguments passed to the underlying
\code{xyplot} function. Some can be really useful. See details. 
\end{ldescription}
\end{Arguments}
%
\begin{Details}\relax
Kind of plots that can be produced

rss: Residual Sum of Squares progression.
OF: Observed vs. Fitted
RF: Residual vs. Fitted
OFT: Observed and Fitted with thermal time as the x-axis.
trace: trace for the parameters
density: density for the parameters








To choosing a subset of the 24 dry biomass partitioning coefficients
use the subset option as you would in the xyplot using nams as the
name. For example,
\code{plot(x,plot.kind="trace",subset=nams=="kLeaf\_1")} will select it
for the leaf at the first phenological stage.
\end{Details}
%
\begin{Value}
A \code{lattice} plot.
\end{Value}
%
\begin{SeeAlso}\relax
 \code{\LinkA{MCMCBioGro}{MCMCBioGro}} 
\end{SeeAlso}
%
\begin{Examples}
\begin{ExampleCode}
## See the MCMCBioGro function
\end{ExampleCode}
\end{Examples}
\inputencoding{utf8}
\HeaderA{plot.MCMCc4photo}{Plotting function for MCMCc4photo objects}{plot.MCMCc4photo}
\keyword{hplot}{plot.MCMCc4photo}
%
\begin{Description}\relax
By default it prints the trace of the four parameters
being estimated by the \code{\LinkA{MCMCc4photo}{MCMCc4photo}}
function. As an option the density can be plotted.
\end{Description}
%
\begin{Usage}
\begin{verbatim}
  plot.MCMCc4photo(x, x2 = NULL, x3 = NULL,
    plot.kind = c("trace", "density"), type = c("l", "p"),
    burnin = 1, cols = c("blue", "green", "purple"),
    prior = FALSE, pcol = "black", ...)
\end{verbatim}
\end{Usage}
%
\begin{Arguments}
\begin{ldescription}
\item[\code{x}] \code{\LinkA{MCMCc4photo}{MCMCc4photo}} object.

\item[\code{x2}] optional additional \code{\LinkA{MCMCc4photo}{MCMCc4photo}}
object.

\item[\code{x3}] optional additional \code{\LinkA{MCMCc4photo}{MCMCc4photo}}
object.

\item[\code{plot.kind}] 'trace' plots the iteration history and
'density' plots the kernel density.

\item[\code{type}] only the options for line and point are
offered.

\item[\code{burnin}] this will remove part of the chain that can
be considered burn-in period.  The plots will no include
this part.

\item[\code{cols}] Argument to pass colors to the line or points
being plotted.

\item[\code{prior}] Whether to plot the prior density. It only
works when x2 = NULL and x3 = NULL. Default is FALSE.

\item[\code{pcol}] Color used for plotting the prior density.

\item[\code{...}] Optional arguments.
\end{ldescription}
\end{Arguments}
%
\begin{Details}\relax
This function uses internally
\code{\LinkA{xyplot}{xyplot}},
\code{\LinkA{densityplot}{densityplot}} and
\code{\LinkA{panel.mathdensity}{panel.mathdensity}} both in the
'lattice' package.
\end{Details}
%
\begin{SeeAlso}\relax
\code{\LinkA{MCMCc4photo}{MCMCc4photo}}
\end{SeeAlso}
\inputencoding{utf8}
\HeaderA{plot.MCMCEc4photo}{Plottin function for MCMCEc4photo objects}{plot.MCMCEc4photo}
\keyword{hplot}{plot.MCMCEc4photo}
%
\begin{Description}\relax
By default it prints the trace of the four parameters
being estimated by the \code{\LinkA{MCMCEc4photo}{MCMCEc4photo}}
function. As an option the density can be plotted.
\end{Description}
%
\begin{Usage}
\begin{verbatim}
  plot.MCMCEc4photo(x, x2 = NULL, x3 = NULL,
    type = c("trace", "density"), ...)
\end{verbatim}
\end{Usage}
%
\begin{Arguments}
\begin{ldescription}
\item[\code{x}] \code{\LinkA{MCMCEc4photo}{MCMCEc4photo}} object.

\item[\code{x2}] optional additional \code{link\{MCMCEc4photo\}}
object.

\item[\code{x3}] optional additional \code{link\{MCMCEc4photo\}}
object.

\item[\code{type}] 'trace' plots the iteration history and
'density' plots the kernel density.

\item[\code{...}] Optional arguments.
\end{ldescription}
\end{Arguments}
%
\begin{Details}\relax
This function uses internally
\code{\LinkA{xyplot}{xyplot}} and
\code{\LinkA{densityplot}{densityplot}} both in the 'lattice'
package.
\end{Details}
%
\begin{SeeAlso}\relax
\code{\LinkA{MCMCEc4photo}{MCMCEc4photo}}
\end{SeeAlso}
\inputencoding{utf8}
\HeaderA{plot.mOpc3photo}{Plotting method}{plot.mOpc3photo}
%
\begin{Description}\relax
Plotting method
\end{Description}
%
\begin{Usage}
\begin{verbatim}
  plot.mOpc3photo(x, parm = c("vcmax", "jmax"), ...)
\end{verbatim}
\end{Usage}
\inputencoding{utf8}
\HeaderA{plot.mOpc4photo}{Plotting method}{plot.mOpc4photo}
%
\begin{Description}\relax
Plotting method
\end{Description}
%
\begin{Usage}
\begin{verbatim}
  plot.mOpc4photo(x, parm = c("vmax", "alpha"), ...)
\end{verbatim}
\end{Usage}
\inputencoding{utf8}
\HeaderA{plot.willowGro}{Plotting function for willowGro objects}{plot.willowGro}
\keyword{hplot}{plot.willowGro}
%
\begin{Description}\relax
By default it plots stem, leaf, root, rhizome and LAI for
a \code{willowGro} object. Optionally, the observed data
can be plotted.
\end{Description}
%
\begin{Usage}
\begin{verbatim}
  plot.willowGro(x, obs = NULL, stem = TRUE, leaf = TRUE,
    root = TRUE, rhizome = TRUE, LAI = TRUE, grain = TRUE,
    xlab = NULL, ylab = NULL, pch = 21, lty = 1, lwd = 1,
    col = c("blue", "green", "red", "magenta", "black", "purple"),
    x1 = 0.1, y1 = 0.8, plot.kind = c("DB", "SW"), ...)
\end{verbatim}
\end{Usage}
%
\begin{Arguments}
\begin{ldescription}
\item[\code{x}] \code{\LinkA{willowGro}{willowGro}} object.

\item[\code{obs}] optional observed data object (format
following the \code{\LinkA{OpwillowGro}{OpwillowGro}} function .

\item[\code{stem}] whether to plot simulated stem (default =
TRUE).

\item[\code{leaf}] whether to plot simulated leaf (default =
TRUE).

\item[\code{root}] whether to plot simulated root (default =
TRUE).

\item[\code{rhizome}] whether to plot simulated rhizome (default
= TRUE).

\item[\code{grain}] whether to plot simulated grain (default =
TRUE).

\item[\code{LAI}] whether to plot simulated LAI (default =
TRUE).

\item[\code{pch}] point character.

\item[\code{lty}] line type.

\item[\code{lwd}] line width.

\item[\code{col}] Control of colors.

\item[\code{x1}] position of the legend. x coordinate (0-1).

\item[\code{y1}] position of the legend. y coordinate (0-1).

\item[\code{plot.kind}] DB plots dry biomass and SW plots soil
water.

\item[\code{...}] Optional arguments.
\end{ldescription}
\end{Arguments}
%
\begin{Details}\relax
This function uses internally
\code{\LinkA{xyplot}{xyplot}} in the 'lattice' package.
\end{Details}
%
\begin{SeeAlso}\relax
\code{\LinkA{willowGro}{willowGro}} \code{\LinkA{OpwillowGro}{OpwillowGro}}
\end{SeeAlso}
\inputencoding{utf8}
\HeaderA{plotAC}{plot A/Ci curve}{plotAC}
\keyword{hplot}{plotAC}
%
\begin{Description}\relax
Function to plot A/Ci curves
\end{Description}
%
\begin{Usage}
\begin{verbatim}
  plotAC(data, fittd, id.col = 1, trt.col = 2,
    ylab = "CO2 Uptake", xlab = "Ci", by = c("trt", "ID"),
    type = c("p", "smooth"))
\end{verbatim}
\end{Usage}
%
\begin{Arguments}
\begin{ldescription}
\item[\code{data}] Input data in the format needed for the
\code{\LinkA{mOpc4photo}{mOpc4photo}}; assumed to have the following
structure col 1: trt col 2 (optional): other treatment
factor col 2: Assimilation col 3: Quantum flux col 4:
Temperature col 5: Relative humidity col 6: Intercellular
CO2 col 7: Reference CO2

\item[\code{fittd}] Optional fitted values.

\item[\code{id.col}] Specify which column has the ids. Default
is col 1.

\item[\code{trt.col}] Specify which column has the treatments.
Default is col 2. If no treatment is specified then use
1.

\item[\code{ylab}] Label for the y-axis.

\item[\code{xlab}] Label for the x-axis.

\item[\code{by}] Whether to plot by id or by treatment.

\item[\code{type}] this argument is passed to the
\code{\LinkA{xyplot}{xyplot}}. It changes the plotting symbols
behavior.
\end{ldescription}
\end{Arguments}
%
\begin{Details}\relax
A small helper function that can be used to easily plot
multiple A/Ci curves
\end{Details}
%
\begin{Value}
NULL, creates plot
\end{Value}
%
\begin{Author}\relax
Fernando E. Miguez
\end{Author}
%
\begin{SeeAlso}\relax
See Also \code{\LinkA{xyplot}{xyplot}}.
\end{SeeAlso}
%
\begin{Examples}
\begin{ExampleCode}
data(aci)
plotAC(aci, trt.col=1)
\end{ExampleCode}
\end{Examples}
\inputencoding{utf8}
\HeaderA{plotAQ}{plot A/Q curve}{plotAQ}
%
\begin{Description}\relax
Function to plot A/Q curves
\end{Description}
%
\begin{Usage}
\begin{verbatim}
  plotAQ(data, fittd, id.col = 1, trt.col = 2,
    ylab = "CO2 Uptake", xlab = "Quantum flux",
    by = c("trt", "ID"), type = "o", ...)
\end{verbatim}
\end{Usage}
%
\begin{Arguments}
\begin{ldescription}
\item[\code{data}] is assumed to have the following structure
col 1: trt col 2 (optional): other treatment factor col
2: Assimilation col 3: Quantum flux col 4: Temperature
col 5: Relative humidity col 6 (optional): Reference CO2

\item[\code{fittd}] 

\item[\code{id.col}] 

\item[\code{trt.col}] 

\item[\code{ylab}] 

\item[\code{xlab}] 

\item[\code{by}] 

\item[\code{type}] 

\item[\code{...}] 
\end{ldescription}
\end{Arguments}
%
\begin{Value}
NULL, creates plot
\end{Value}
%
\begin{Author}\relax
Fernando E. Miguez
\end{Author}
\inputencoding{utf8}
\HeaderA{predict.Opc3photo}{Predict method}{predict.Opc3photo}
%
\begin{Description}\relax
Predict method
\end{Description}
%
\begin{Usage}
\begin{verbatim}
  predict.Opc3photo(object, newdata, ...)
\end{verbatim}
\end{Usage}
\inputencoding{utf8}
\HeaderA{predict.Opc4photo}{Predict method}{predict.Opc4photo}
%
\begin{Description}\relax
Predict method
\end{Description}
%
\begin{Usage}
\begin{verbatim}
  predict.Opc4photo(object, newdata, ...)
\end{verbatim}
\end{Usage}
\inputencoding{utf8}
\HeaderA{print.BioGro}{printing method for BioGro}{print.BioGro}
%
\begin{Description}\relax
printing method for BioGro
\end{Description}
%
\begin{Usage}
\begin{verbatim}
## S3 method for class 'BioGro'
print(x, level = 1, ...)
\end{verbatim}
\end{Usage}
%
\begin{Arguments}
\begin{ldescription}
\item[\code{x}] 
\end{ldescription}
\end{Arguments}
\inputencoding{utf8}
\HeaderA{print.MaizeGro}{printing method for MaizeGro}{print.MaizeGro}
%
\begin{Description}\relax
printing method for MaizeGro
\end{Description}
%
\begin{Usage}
\begin{verbatim}
## S3 method for class 'MaizeGro'
print(x, level = 1, ...)
\end{verbatim}
\end{Usage}
%
\begin{Arguments}
\begin{ldescription}
\item[\code{x}] 
\end{ldescription}
\end{Arguments}
\inputencoding{utf8}
\HeaderA{print.MCMCBioGro}{printing method for MCMCBioGro}{print.MCMCBioGro}
%
\begin{Description}\relax
printing method for MCMCBioGro
\end{Description}
%
\begin{Usage}
\begin{verbatim}
## S3 method for class 'MCMCBioGro'
print(x, ...)
\end{verbatim}
\end{Usage}
%
\begin{Arguments}
\begin{ldescription}
\item[\code{x}] 
\end{ldescription}
\end{Arguments}
\inputencoding{utf8}
\HeaderA{print.MCMCc4photo}{Function for printing the MCMCc4photo objects}{print.MCMCc4photo}
%
\begin{Description}\relax
Function for printing the MCMCc4photo objects
\end{Description}
%
\begin{Usage}
\begin{verbatim}
## S3 method for class 'MCMCc4photo'
print(x, burnin = 1, level = 0.95, digits = 1, ...)
\end{verbatim}
\end{Usage}
\inputencoding{utf8}
\HeaderA{print.MCMCEc4photo}{Print an MCMCEc4photo object}{print.MCMCEc4photo}
\keyword{optimize}{print.MCMCEc4photo}
%
\begin{Description}\relax
This functions doesn't just print the object components,
but it also computes quantiles according to the
\code{level} argument below and a correlation matrix as
well as a summary for each parameter.
\end{Description}
%
\begin{Usage}
\begin{verbatim}
  print.MCMCEc4photo(x, level = 0.95, ...)
\end{verbatim}
\end{Usage}
%
\begin{Arguments}
\begin{ldescription}
\item[\code{x}] \code{\LinkA{MCMCEc4photo}{MCMCEc4photo}} object

\item[\code{level}] specified \code{level} for the Highest
Posterior Density region.

\item[\code{...}] Optional arguments
\end{ldescription}
\end{Arguments}
%
\begin{Details}\relax
The Highest Posterior Density region is calculated using
the \code{\LinkA{quantile}{quantile}} function.  The correlation
matrix is computed using the \code{\LinkA{cor}{cor}} function.
The summaries for each parameter are computed using the
\code{\LinkA{summary}{summary}} function.
\end{Details}
%
\begin{SeeAlso}\relax
\code{\LinkA{MCMCEc4photo}{MCMCEc4photo}}
\end{SeeAlso}
\inputencoding{utf8}
\HeaderA{print.mOpc3photo}{Printing method}{print.mOpc3photo}
%
\begin{Description}\relax
Printing method
\end{Description}
%
\begin{Usage}
\begin{verbatim}
  print.mOpc3photo(x, ...)
\end{verbatim}
\end{Usage}
\inputencoding{utf8}
\HeaderA{print.mOpc4photo}{Printing method}{print.mOpc4photo}
%
\begin{Description}\relax
Printing method
\end{Description}
%
\begin{Usage}
\begin{verbatim}
  print.mOpc4photo(x, ...)
\end{verbatim}
\end{Usage}
\inputencoding{utf8}
\HeaderA{print.OpBioGro}{ Print an OpBioGro object}{print.OpBioGro}
\keyword{optimize}{print.OpBioGro}
%
\begin{Description}\relax
Print an object generated by OpBioGro
\end{Description}
%
\begin{Usage}
\begin{verbatim}
## S3 method for class 'OpBioGro'
print(x, ...)
\end{verbatim}
\end{Usage}
%
\begin{Arguments}
\begin{ldescription}
\item[\code{x}] 
Object returned from BioGro optimization
(Collatz model).

\item[\code{...}] 
Optional arguments. 

\end{ldescription}
\end{Arguments}
%
\begin{SeeAlso}\relax
  \code{\LinkA{OpBioGro}{OpBioGro}}
\end{SeeAlso}
\inputencoding{utf8}
\HeaderA{print.Opc3photo}{Display methods for Opc4photo and OpEC4photo}{print.Opc3photo}
%
\begin{Description}\relax
Display methods for Opc4photo and OpEC4photo
\end{Description}
%
\begin{Usage}
\begin{verbatim}
  print.Opc3photo(x, digits = 2, ...)
\end{verbatim}
\end{Usage}
\inputencoding{utf8}
\HeaderA{print.Opc4photo}{Display methods for Opc4photo and OpEC4photo}{print.Opc4photo}
%
\begin{Description}\relax
Display methods for Opc4photo and OpEC4photo
\end{Description}
%
\begin{Usage}
\begin{verbatim}
  print.Opc4photo(x, digits = 2, ...)
\end{verbatim}
\end{Usage}
\inputencoding{utf8}
\HeaderA{print.OpEC4photo}{ Print an OpEC4photo object}{print.OpEC4photo}
\keyword{optimize}{print.OpEC4photo}
%
\begin{Description}\relax
Print an object generated by OpEC4photo
\end{Description}
%
\begin{Usage}
\begin{verbatim}
## S3 method for class 'OpEC4photo'
print(x, ...)
\end{verbatim}
\end{Usage}
%
\begin{Arguments}
\begin{ldescription}
\item[\code{x}] 
Object returned from C4 photosynthesis optimization
(von Caemmerer model).

\item[\code{...}] 
Optional arguments. 

\end{ldescription}
\end{Arguments}
%
\begin{SeeAlso}\relax
  \code{\LinkA{OpEC4photo}{OpEC4photo}}
\end{SeeAlso}
\inputencoding{utf8}
\HeaderA{Rmiscanmod}{RUE-based model to calculate miscanthus growth and yield.}{Rmiscanmod}
\keyword{models}{Rmiscanmod}
%
\begin{Description}\relax
Simple model to calculate crop growth and yield based on
MISCANMOD (see references).
\end{Description}
%
\begin{Usage}
\begin{verbatim}
  Rmiscanmod(data, RUE = 2.4, LER = 0.01, Tb = 10,
    k = 0.67, LAIdrd = 0.8, LAIStop = 1.8, RUEdrd = 1.3,
    RUEStop = 2.5, SMDdrd = -30, SMDStop = -120,
    FieldC = 45, iWatCont = 45, a = 6682.2, b = -0.33,
    soildepth = 0.6)
\end{verbatim}
\end{Usage}
%
\begin{Arguments}
\begin{ldescription}
\item[\code{data}] data.frame or matrix described in details.

\item[\code{RUE}] Radiation use efficiency (g/MJ).

\item[\code{LER}] Leaf expansion rate LAI/GDD.

\item[\code{Tb}] Base Temperature (Celsius).

\item[\code{k}] extinction coefficient of light in the canopy.

\item[\code{LAIdrd}] Leaf Area Index 'down regulation decline'.

\item[\code{LAIStop}] Leaf Area Index 'down regulation decline'
threshold .

\item[\code{RUEdrd}] Radiation Use Efficieny 'down regulation
decline'.

\item[\code{RUEStop}] Radiation Use Efficieny 'down regulation
decline' threshold.

\item[\code{SMDdrd}] Soil Moisture Deficit 'down regulation
decline'.

\item[\code{SMDStop}] Soil Moisture Deficit 'down regulation
decline' threshold.

\item[\code{FieldC}] Soil field capacity.

\item[\code{iWatCont}] Initial water content.

\item[\code{a}] Soil parameter.

\item[\code{b}] Soil parameter.

\item[\code{soildepth}] Soil depth.
\end{ldescription}
\end{Arguments}
%
\begin{Details}\relax
The data.frame or matrix should contain

column 1: year column 2: month column 3: day column 4: JD
column 5: max T (Celsius) column 6: min T (Celsius)
column 7: PPFD or solar radiation divided by 2 (MJ/m2)
column 8: Potential evaporation column 9: precip (mm)
\end{Details}
%
\begin{Value}
returns a list
\end{Value}
%
\begin{References}\relax
Clifton-Brown, J. C.; Neilson, B.; Lewandowski, I. and
Jones, M. B. The modelled productivity of Miscanthus x
giganteus (GREEF et DEU) in Ireland. Industrial Crops and
Products, 2000, 12, 97-109.

Clifton-brown, J. C.; Stampfl, P. F. and Jones, M. B.
Miscanthus biomass production for energy in Europe and
its potential contribution to decreasing fossil fuel
carbon emissions. Global Change Biology, 2004, 10,
509-518.
\end{References}
%
\begin{Examples}
\begin{ExampleCode}
## Need to get an example data set and then run it
## Not run: 
data(WD1979)

res <- Rmiscanmod(WD1979)

## convert to Mg/ha

Yld <- res$Yield / 100

xyplot(Yld ~ 1:365 ,
       xlab='doy',
       ylab='Dry biomass (Mg/ha)')

## although the default value for Field Capacity is 45
## a more reasonable value is closer to 27



## End(Not run)
\end{ExampleCode}
\end{Examples}
\inputencoding{utf8}
\HeaderA{rootDist}{Returns a value for rootDist based on the arguments layers, rootDepth, depthsp, and rfl. This value then factors into several equations in the primary function soilML.}{rootDist}
%
\begin{Description}\relax
Returns a value for rootDist based on the arguments layers,
rootDepth, depthsp, and rfl. This value then factors into
several equations in the primary function soilML.
\end{Description}
%
\begin{Usage}
\begin{verbatim}
rootDist(layers, rootDepth, depthsp, rfl)
\end{verbatim}
\end{Usage}
\inputencoding{utf8}
\HeaderA{RsqC4photo}{R-squared for C4 photosynthesis simulation (Collatz model)}{RsqC4photo}
\keyword{univar}{RsqC4photo}
%
\begin{Description}\relax
This is an auxiliary function which is made available in
case it is useful. It calculates the R-squared based on
observed assimilation (or stomatal conductance) data and
coefficients for the Collatz C4 photosynthesis model. The
only coefficients being considered are Vcmax and alpha as
described in the Collatz paper. At the moment it does not
optimize k; this will be added soon.  Notice that to be
able to optimize k A/Ci type data are needed.
\end{Description}
%
\begin{Usage}
\begin{verbatim}
  RsqC4photo(data, vmax = 39, alph = 0.04, kparm = 0.7,
    theta = 0.83, beta = 0.93, Rd = 0.8, iupperT = 37.5,
    ilowerT = 3, Catm = 380, b0 = 0.08, b1 = 3, StomWS = 1,
    response = c("Assim", "StomCond"))
\end{verbatim}
\end{Usage}
%
\begin{Arguments}
\begin{ldescription}
\item[\code{data}] observed assimilation data, which should be a
data frame or matrix.  The first column should be
observed net assimilation rate (\eqn{\mu}{} mol
\eqn{m^{-2}}{} \eqn{s^{-1}}{}).  The second column should be
the observed quantum flux (\eqn{\mu}{} mol \eqn{m^{-2}}{}
\eqn{s^{-1}}{}).  The third column should be observed
temperature of the leaf (Celsius).  The fourth column
should be the observed relative humidity in proportion
(e.g. 0.7).

\item[\code{vmax}] Vcmax (default = 39); for more details see
the \code{\LinkA{c4photo}{c4photo}} function.

\item[\code{alph}] alpha as in Collatz (default = 0.04); for
more details see the \code{\LinkA{c4photo}{c4photo}} function.

\item[\code{kparm}] k as in Collatz (default = 0.70); for more
details see the \code{\LinkA{c4photo}{c4photo}} function.

\item[\code{theta}] theta as in Collatz (default = 0.83); for
more details see the \code{\LinkA{c4photo}{c4photo}} function.

\item[\code{beta}] beta as in Collatz (default = 0.93); for more
details see the \code{\LinkA{c4photo}{c4photo}} function.

\item[\code{Rd}] Rd as in Collatz (default = 0.8); for more
details see the \code{\LinkA{c4photo}{c4photo}} function.

\item[\code{StomWS}] StomWS as in Collatz (default = 1); for
more details see the \code{\LinkA{c4photo}{c4photo}} function.

\item[\code{Catm}] Atmospheric CO2 in ppm (or
\eqn{\mu}{}mol/mol).

\item[\code{b0}] Intercept for the Ball-Berry model.

\item[\code{b1}] Slope for the Ball-Berry model.

\item[\code{response}] Use \code{'Assim'} if you want an
\eqn{R^2}{} for assimilation data and use \code{'StomCond'}
if you want an \eqn{R^2}{} for stomatal conductance data.
\end{ldescription}
\end{Arguments}
%
\begin{Value}
a \code{\LinkA{numeric}{numeric}} object

It simply returns the \eqn{R^2}{} value for the given data
and coefficients.
\end{Value}
%
\begin{Examples}
\begin{ExampleCode}
data(obsNaid)
## These data are from Naidu et al. (2003)
## in the correct format
res <- RsqC4photo(obsNaid)
## Other example using Beale, Bint and Long (1996)
data(obsBea)
resB <- RsqC4photo(obsBea)
\end{ExampleCode}
\end{Examples}
\inputencoding{utf8}
\HeaderA{RsqEC4photo}{R-squared for C4 photosynthesis simulation (von Caemmerer model)}{RsqEC4photo}
\keyword{univar}{RsqEC4photo}
%
\begin{Description}\relax
This is an auxiliary function which is made available in case it is
useful. It calculates the R-squared based on observed assimilation (or
stomatal conductance) data and coefficients for the von Caemmerer C4
photosynthesis model. The only coefficients being considered are
Vcmax, Vpmax, Vpr and Jmax. 

\end{Description}
%
\begin{Usage}
\begin{verbatim}
RsqEC4photo(obsDat, iVcmax = 60, iVpmax = 120, iVpr = 80, iJmax = 400, 
    co2 = 380, o2 = 210, type = c("Assim", "StomCond"))
\end{verbatim}
\end{Usage}
%
\begin{Arguments}
\begin{ldescription}

\item[\code{obsDat}]  observed assimilation data, which should be a data
frame or matrix.
The first column should be observed net
assimilation rate  (\eqn{\mu}{} mol \eqn{m^{-2}}{} \eqn{s^{-1}}{}).
The second column should be the observed
quantum flux  (\eqn{\mu}{} mol \eqn{m^{-2}}{} \eqn{s^{-1}}{}).
The third column should be observed temperature of the leaf
(Celsius).
The fourth column should be the observed relative humidity
in proportion (e.g. 0.7).

\item[\code{iVcmax}] Maximum rubisco activity
(\eqn{\mu}{} mol \eqn{m^{-2}}{} \eqn{s^{-1}}{}).
\item[\code{iVpmax}] Maximum PEP carboxylase activity  (\eqn{\mu}{} mol
\eqn{m^{-2}}{} \eqn{s^{-1}}{}).
\item[\code{iVpr}] PEP regeneration rate
(\eqn{\mu}{} mol \eqn{m^{-2}}{} \eqn{s^{-1}}{}).
\item[\code{iJmax}] Maximal electron transport rate
(\eqn{\mu}{}mol electrons \eqn{m^{-2}}{} \eqn{s^{-1}}{})..
\item[\code{co2}] atmospheric carbon dioxide concentration
(ppm or \eqn{\mu}{}bar) (default = 380).
\item[\code{o2}] atmospheric oxygen concentration (mbar) (default = 210).
\item[\code{type}] Use \code{"Assim"} if you want an \eqn{R^2}{} for assimilation
data and use \code{"StomCond"} if you want an \eqn{R^2}{} for
stomatal conductance data.
\end{ldescription}
\end{Arguments}
%
\begin{Value}

a \code{\LinkA{numeric}{numeric}} object

It simply returns the \eqn{R^2}{} value for the given data and coefficients.
\end{Value}
%
\begin{Examples}
\begin{ExampleCode}
data(obsNaid)
obs <- obsNaid
## These data are from Naidu et al. (2003)
## in the correct format
res <- RsqEC4photo(obs)
## Other example using Beale, Bint and Long (1996)
data(obsBea)
obsD <- obsBea
resB <- RsqEC4photo(obsD)
\end{ExampleCode}
\end{Examples}
\inputencoding{utf8}
\HeaderA{RssBioGro}{Residual sum of squares for BioGro.}{RssBioGro}
\keyword{models}{RssBioGro}
%
\begin{Description}\relax
Computes residual sum of squares for the
\code{\LinkA{BioGro}{BioGro}} function.
\end{Description}
%
\begin{Usage}
\begin{verbatim}
  RssBioGro(obs, sim)
\end{verbatim}
\end{Usage}
%
\begin{Arguments}
\begin{ldescription}
\item[\code{obs}] Observed data.

\item[\code{sim}] Simulated data.
\end{ldescription}
\end{Arguments}
%
\begin{Value}
Atomic vector with the residual sum of squares.
\end{Value}
%
\begin{Author}\relax
Fernando E. Miguez
\end{Author}
%
\begin{SeeAlso}\relax
See Also \code{\LinkA{BioGro}{BioGro}}.
\end{SeeAlso}
%
\begin{Examples}
\begin{ExampleCode}
## A simple example
data(annualDB)
data(EngWea94i)
res <- BioGro(EngWea94i)
RssBioGro(annualDB,res)
\end{ExampleCode}
\end{Examples}
\inputencoding{utf8}
\HeaderA{RssMaizeGro}{Very simple function to compare the distance between simulated and observed data for the BioGro function Need to add an argument such as pc.sigmas "plant component sigmas" If variability of the plant component is known}{RssMaizeGro}
%
\begin{Description}\relax
Very simple function to compare the distance between
simulated and observed data for the BioGro function Need to
add an argument such as pc.sigmas "plant component sigmas"
If variability of the plant component is known
\end{Description}
%
\begin{Usage}
\begin{verbatim}
RssMaizeGro(obs, sim)
\end{verbatim}
\end{Usage}
\inputencoding{utf8}
\HeaderA{RUEmod}{Radiation use efficiency based model}{RUEmod}
\keyword{models}{RUEmod}
%
\begin{Description}\relax
Simulates leaf area index, biomass and light interception.
Based on the Monteith (1973) equations, adapted for Miscanthus
by Clifton Brown et al.(2001) (see references). 
\end{Description}
%
\begin{Usage}
\begin{verbatim}
RUEmod(Rad, T.a, doy.s = 91, doy.f = 227, lai.c = 0.0102, 
    rue = 2.4, T.b = 10, k = 0.68) 
\end{verbatim}
\end{Usage}
%
\begin{Arguments}
\begin{ldescription}

\item[\code{Rad}] Daily solar radiation (MJ \eqn{ha^{-2}}{}).
\item[\code{T.a}] Daily average temperature (Fahrenheit).
\item[\code{doy.s}] first day of the growing season, default 91.
\item[\code{doy.f}] last day of the growing season, default 227.
\item[\code{lai.c}] linear relationship between growing degree days and leaf area index.
\item[\code{rue}] radiation use efficiency, default 2.4.
\item[\code{T.b}] base temperature for calculating growing degree days, default 10.
\item[\code{k}] light extinction coefficient, default 0.68.
\end{ldescription}
\end{Arguments}
%
\begin{Value}
a \code{\LinkA{list}{list}} structure with components
\begin{ldescription}
\item[\code{doy}] day of the year.
\item[\code{lai.cum}] cumulative leaf area index.
\item[\code{AG.cum}] cumulative above ground dry biomass (Mg \eqn{ha^{-1}}{}).
\item[\code{AGDD}] cumulative growing degree days.
\item[\code{Int.e}] Intercepted solar radiation.
\end{ldescription}
\end{Value}
%
\begin{Note}\relax
This empirical model is useful but it has limitations. 
\end{Note}
%
\begin{References}\relax
Monteith (1973) \emph{Principles of Environmental Physics}.
Edward Arnold, London. UK.

Clifton-Brown, J.C., Long, S.P. and Jorgensen, U. with
contributions from S.A. Humphries, Schwarz, K.-U. and
Schwarz, H. (2001) \emph{Miscanthus Productivity}. Ch 4. In:
Miscanthus for Energy and Fibre. Edited by:
Jones, Michael B. and Walsh, Mary. James \& James (Science Publishers),
London, UK.
\end{References}
%
\begin{SeeAlso}\relax
\code{\LinkA{RUEmodMY}{RUEmodMY}}
\end{SeeAlso}
%
\begin{Examples}
\begin{ExampleCode}

## See RUEmodMY

\end{ExampleCode}
\end{Examples}
\inputencoding{utf8}
\HeaderA{RUEmodMY}{Radiation use efficiency based model}{RUEmodMY}
\keyword{models}{RUEmodMY}
%
\begin{Description}\relax
Same as \code{\LinkA{RUEmod}{RUEmod}} but it handles multiple
years.
\end{Description}
%
\begin{Usage}
\begin{verbatim}
  RUEmodMY(weatherdatafile, doy.s = 91, doy.f = 227, ...)
\end{verbatim}
\end{Usage}
%
\begin{Arguments}
\begin{ldescription}
\item[\code{weatherdatafile}] weather data file (see example).

\item[\code{doy.s}] first day of the growing season, default
91.

\item[\code{doy.f}] last day of the growing season, default
227.

\item[\code{...}] additional arguments to be passed to the
\code{\LinkA{RUEmod}{RUEmod}} function.
\end{ldescription}
\end{Arguments}
%
\begin{Value}
a \code{\LinkA{data.frame}{data.frame}} structure with components
\end{Value}
%
\begin{Examples}
\begin{ExampleCode}
## weather data from Champaign, IL
data(cmiWet)
tmp1 <- RUEmodMY(cmiWet)

xyplot(AG.cum ~ doy | factor(year), type='l', data = tmp1,
       lwd=2,
       ylab=expression(paste('dry biomass (Mg ',ha^-1,')')),
       xlab='DOY')
\end{ExampleCode}
\end{Examples}
\inputencoding{utf8}
\HeaderA{showSoilType}{the function that deinfes the soil types}{showSoilType}
%
\begin{Description}\relax
the function that deinfes the soil types
\end{Description}
%
\begin{Usage}
\begin{verbatim}
showSoilType(soiltype)
\end{verbatim}
\end{Usage}
%
\begin{Arguments}
\begin{ldescription}
\item[\code{soiltype}] an integer from 0 to 10 that coresopnds
to a type of soil
\end{ldescription}
\end{Arguments}
\inputencoding{utf8}
\HeaderA{simDat2}{Simulated biomass data.}{simDat2}
\keyword{datasets}{simDat2}
%
\begin{Description}\relax
Simulated data produced by \code{\LinkA{BioGro}{BioGro}}.
\end{Description}
%
\begin{Format}
A data frame with 5 observations on the following 6 variables.
\begin{description}
 \item[list('TT')] a numeric vector\item[list('Stem')] a
numeric vector\item[list('Leaf')] a numeric vector\item[list('Root')] a
numeric vector\item[list('Rhiz')] a numeric vector\item[list('LAI')] a
numeric vector
\end{description}
\end{Format}
%
\begin{Details}\relax
\textasciitilde{}\textasciitilde{} If necessary, more details than the description above
\textasciitilde{}\textasciitilde{}
\end{Details}
%
\begin{Source}\relax
\textasciitilde{}\textasciitilde{} reference to a publication or URL from which the data
were obtained \textasciitilde{}\textasciitilde{}
\end{Source}
%
\begin{References}\relax
\textasciitilde{}\textasciitilde{} possibly secondary sources and usages \textasciitilde{}\textasciitilde{}
\end{References}
%
\begin{Examples}
\begin{ExampleCode}
data(simDat2)
## maybe str(simDat2) ; plot(simDat2) ...
\end{ExampleCode}
\end{Examples}
\inputencoding{utf8}
\HeaderA{SoilEvapo}{Soil Evaporation}{SoilEvapo}
\keyword{models}{SoilEvapo}
%
\begin{Description}\relax
Calculates soil evaporation
\end{Description}
%
\begin{Usage}
\begin{verbatim}
  SoilEvapo(LAI, k, AirTemp, IRad, awc, FieldC, WiltP,
    winds, RelH)
\end{verbatim}
\end{Usage}
%
\begin{Arguments}
\begin{ldescription}
\item[\code{LAI}] Leaf Area Index.

\item[\code{k}] \textasciitilde{}\textasciitilde{}Describe \code{k} here\textasciitilde{}\textasciitilde{}

\item[\code{AirTemp}] Air temperature.

\item[\code{IRad}] Incident radiation.

\item[\code{awc}] Available water content.

\item[\code{FieldC}] Field capacity.

\item[\code{WiltP}] Wilting point.

\item[\code{winds}] Wind speed.

\item[\code{RelH}] Relative humidty.
\end{ldescription}
\end{Arguments}
%
\begin{Details}\relax
The style of the code is \code{C} like because this is a
prototype for the underlying \code{C} (like so many other
functions in this package). I leave it here for future
development.
\end{Details}
%
\begin{Value}
Returns a single value of soil Evaporation in Mg H20 per
hectare.
\end{Value}
%
\begin{Author}\relax
Fernando Miguez
\end{Author}
%
\begin{SeeAlso}\relax
Source code :)
\end{SeeAlso}
%
\begin{Examples}
\begin{ExampleCode}
SoilEvapo(LAI=3,k=0.68,AirTemp=20,IRad=1000,awc=0.3,FieldC=0.4,WiltP=0.2,winds=3,RelH=0.8)
\end{ExampleCode}
\end{Examples}
\inputencoding{utf8}
\HeaderA{soilML}{soil multi-layered}{soilML}
\aliasA{rootDist}{soilML}{rootDist}
\keyword{models}{soilML}
%
\begin{Description}\relax
Simulates soil water content for a layered soil.
\end{Description}
%
\begin{Usage}
\begin{verbatim}
  soilML(precipt, CanopyT, cws, soilDepth, FieldC, WiltP,
    phi1 = 0.01, phi2 = 10,
    wsFun = c("linear", "logistic", "exp", "none"), rootDB,
    soilLayers = 3, LAI, k, AirTemp, IRad, winds, RelH,
    soilType = 10, hydrDist = 0, rfl = 0.3)
\end{verbatim}
\end{Usage}
%
\begin{Arguments}
\begin{ldescription}
\item[\code{precipt}] Precipitation (mm).

\item[\code{CanopyT}] Canopy transpiration.

\item[\code{cws}] Current water status. Vector of length equal
to soilLayers.

\item[\code{soilDepth}] Rooting depth.

\item[\code{FieldC}] Field capacity.

\item[\code{WiltP}] Wilting point.

\item[\code{phi1}] See \code{\LinkA{wtrstr}{wtrstr}}.

\item[\code{phi2}] See \code{\LinkA{wtrstr}{wtrstr}}.

\item[\code{wsFun}] See \code{\LinkA{wtrstr}{wtrstr}}.

\item[\code{rootDB}] Root biomass (Mg/ha).

\item[\code{soilLayers}] Integer used to specify the number of
soil layers.

\item[\code{LAI}] Leaf area index.

\item[\code{k}] Light extinction coefficient.

\item[\code{AirTemp}] Air temperature (Celsius).

\item[\code{IRad}] Direct irradiance (\eqn{\mu}{} \eqn{m^-2}{}
\eqn{s^-1}{}).

\item[\code{winds}] Wind speed (m/s).

\item[\code{RelH}] Relative humidity (0-1).

\item[\code{soilType}] See \code{\LinkA{showSoilType}{showSoilType}}.

\item[\code{hydrDist}] Zero or otherwise positive integer. Zero
does not calculate hydraulic distribution, otherwise
does.

\item[\code{rfl}] Root factor lambda. A Poisson distribution is
used to simulate the distribution of roots in the soil
profile and this parameter can be used to change the
lambda parameter of the Poisson.
\end{ldescription}
\end{Arguments}
%
\begin{Value}
optiontocalculaterootdept

rootfrontvelocity

dap

matrix with 8 (if hydrDist=0) or 12 (if hydrDist > 0).
\end{Value}
%
\begin{Author}\relax
Fernando E. Miguez
\end{Author}
%
\begin{SeeAlso}\relax
See Also \code{\LinkA{wtrstr}{wtrstr}}.
\end{SeeAlso}
%
\begin{Examples}
\begin{ExampleCode}
layers <- 5
ans <- soilML(precipt=2, CanopyT=2, cws = rep(0.3,layers),
soilDepth=1.5, FieldC=0.33, WiltP=0.13, rootDB=2, soilLayers=layers,
LAI=3, k=0.68, AirTemp=25,IRad=500, winds=2, RelH=0.8, soilType=6,
hydrDist=1)
ans
\end{ExampleCode}
\end{Examples}
\inputencoding{utf8}
\HeaderA{SoilType}{takes an interger from 0 to 10 that coresponds to a specifically defined soil type  and returns the composition of the soil in a list.}{SoilType}
%
\begin{Description}\relax
takes an interger from 0 to 10 that coresponds to a
specifically defined soil type and returns the composition
of the soil in a list.
\end{Description}
%
\begin{Usage}
\begin{verbatim}
SoilType(soiltype)
\end{verbatim}
\end{Usage}
%
\begin{Arguments}
\begin{ldescription}
\item[\code{soiltype}] an integer from 0 to 10 that coresopnds
to a type of soil
\end{ldescription}
\end{Arguments}
\inputencoding{utf8}
\HeaderA{summary.OpBioGro}{This function will implement simple calculations of predicted and residuals for the OpBioGro function}{summary.OpBioGro}
%
\begin{Description}\relax
This function will implement simple calculations of
predicted and residuals for the OpBioGro function
\end{Description}
%
\begin{Usage}
\begin{verbatim}
## S3 method for class 'OpBioGro'
summary(object, ...)
\end{verbatim}
\end{Usage}
%
\begin{Arguments}
\begin{ldescription}
\item[\code{object}] 
\end{ldescription}
\end{Arguments}
\inputencoding{utf8}
\HeaderA{summary.Opc4photo}{This function will implement simple calculations of predicted and residuals for the Opc4photo function}{summary.Opc4photo}
%
\begin{Description}\relax
This function will implement simple calculations of
predicted and residuals for the Opc4photo function
\end{Description}
%
\begin{Usage}
\begin{verbatim}
  summary.Opc4photo(object, ...)
\end{verbatim}
\end{Usage}
\inputencoding{utf8}
\HeaderA{summary.OpEC4photo}{This function will implement simple calculations of predicted and residuals for the OpEC4photo function}{summary.OpEC4photo}
%
\begin{Description}\relax
This function will implement simple calculations of
predicted and residuals for the OpEC4photo function
\end{Description}
%
\begin{Usage}
\begin{verbatim}
## S3 method for class 'OpEC4photo'
summary(object, ...)
\end{verbatim}
\end{Usage}
\inputencoding{utf8}
\HeaderA{sunML}{Sunlit shaded multi-layer model}{sunML}
\keyword{models}{sunML}
%
\begin{Description}\relax
Simulates the light microenvironment in the canopy based
on the sunlit-shade model and the multiple layers.
\end{Description}
%
\begin{Usage}
\begin{verbatim}
  sunML(Idir, Idiff, LAI = 8, nlayers = 8, cos.theta = 0.5,
    kd = 0.7, chi.l = 1, heightf = 3)
\end{verbatim}
\end{Usage}
%
\begin{Arguments}
\begin{ldescription}
\item[\code{I.dir}] direct light (quantum flux), (\eqn{\mu mol
  \; m^{-2} \; s^{-1}}{}).

\item[\code{I.diff}] indirect light (diffuse), (\eqn{\mu mol \;
  m^{-2} \; s^{-1}}{}).

\item[\code{LAI}] leaf area index, default 8.

\item[\code{nlayers}] number of layers in which the canopy is
partitioned, default 8.

\item[\code{kd}] extinction coefficient for diffuse light.

\item[\code{chi.l}] The ratio of horizontal:vertical projected
area of leaves in the canopy segment.

\item[\code{cos.theta}] cosine of \eqn{\theta}{}, solar
zenith angle.
\end{ldescription}
\end{Arguments}
%
\begin{Value}
a \code{\LinkA{list}{list}} structure with components

Vectors size equal to the number of layers.
\end{Value}
%
\begin{Examples}
\begin{ExampleCode}
## Not run: 
res2 <- sunML(1500,200,3,10)

xyplot(Fsun + Fshade ~ c(1:10), data=res2,
       ylab='LAI',
       xlab='layer',
       type='l',lwd=2,col=c('blue','green'),
       lty=c(1,2),
       key=list(text=list(c('Direct','Diffuse')),lty=c(1,2),
                cex=1.2,lwd=2,lines=TRUE,x=0.7,y=0.5,
                col=c('blue','green')))

## End(Not run)
\end{ExampleCode}
\end{Examples}
\inputencoding{utf8}
\HeaderA{TempToDdryA}{Returns a value for TempToDdryA}{TempToDdryA}
%
\begin{Description}\relax
Takes a value for Temp as defined by the SoilEvapo function
and returns a value for DdryA which factors into variable
PsychParam which in turn helps define the Evaporation.
\end{Description}
%
\begin{Usage}
\begin{verbatim}
TempToDdryA(Temp)
\end{verbatim}
\end{Usage}
%
\begin{Arguments}
\begin{ldescription}
\item[\code{Temp}] Temperature
\end{ldescription}
\end{Arguments}
\inputencoding{utf8}
\HeaderA{TempToLHV}{Returns a value for TempToLHV}{TempToLHV}
%
\begin{Description}\relax
Takes a value for Temp as defined by the SoilEvapo function
and returns a value for LHV which factors into variable
PsychParam which in turn helps define the Evaporation.
\end{Description}
%
\begin{Usage}
\begin{verbatim}
TempToLHV(Temp)
\end{verbatim}
\end{Usage}
%
\begin{Arguments}
\begin{ldescription}
\item[\code{Temp}] Temperature
\end{ldescription}
\end{Arguments}
\inputencoding{utf8}
\HeaderA{TempToSFS}{Returns a value for TempToSlopeFS}{TempToSFS}
%
\begin{Description}\relax
Takes a value for Temp as defined by the SoilEvapo function
and returns a value for SlopeFS which helps define the
Evaporation.
\end{Description}
%
\begin{Usage}
\begin{verbatim}
TempToSFS(Temp)
\end{verbatim}
\end{Usage}
%
\begin{Arguments}
\begin{ldescription}
\item[\code{Temp}] Temperature
\end{ldescription}
\end{Arguments}
\inputencoding{utf8}
\HeaderA{TempToSWVC}{Returns a value for TempToSWC}{TempToSWVC}
%
\begin{Description}\relax
Takes a value for Temp as defined by the SoilEvapo function
and returns a value for SWVC which factors into variable
DeltaPVa which in turn helps define the Evaporation.
\end{Description}
%
\begin{Usage}
\begin{verbatim}
TempToSWVC(Temp)
\end{verbatim}
\end{Usage}
%
\begin{Arguments}
\begin{ldescription}
\item[\code{Temp}] Temperature
\end{ldescription}
\end{Arguments}
\inputencoding{utf8}
\HeaderA{valid\_dbp}{Validate dry biomass partitioning coefficients}{valid.Rul.dbp}
\keyword{utilities}{valid\_dbp}
%
\begin{Description}\relax
It attempts to check the requirements of the dry biomass
partitioning coefficients.
\end{Description}
%
\begin{Usage}
\begin{verbatim}
  valid_dbp(x, tol = 0.001)
\end{verbatim}
\end{Usage}
%
\begin{Arguments}
\begin{ldescription}
\item[\code{x}] Vector of length equal to 25 containing the dry
biomass partitioning coefficients for the 6 phenological
stages as for example produced by
\code{\LinkA{phenoParms}{phenoParms}}.

\item[\code{tol}] Numerical tolerance passed to the
\code{\LinkA{all.equal}{all.equal}} function.
\end{ldescription}
\end{Arguments}
%
\begin{Value}
It will return the vector of coefficients unchanged if no
errors are detected.
\end{Value}
%
\begin{Author}\relax
Fernando E. Miguez
\end{Author}
%
\begin{SeeAlso}\relax
\code{\LinkA{BioGro}{BioGro}}
\end{SeeAlso}
%
\begin{Examples}
\begin{ExampleCode}
xx <- as.vector(unlist(phenoParms())[7:31])
valid_dbp(xx)
\end{ExampleCode}
\end{Examples}
\inputencoding{utf8}
\HeaderA{WD1979}{Randomly picked dataset from the Illinois area from 1979}{WD1979}
\keyword{datasets}{WD1979}
%
\begin{Description}\relax
Data from the Illinois area from one point from the 32km
grid from NOAA from 1979. the purpuse is to illustrate
the \code{\LinkA{Rmiscanmod}{Rmiscanmod}} function.
\end{Description}
%
\begin{Format}
A data frame with 365 observations on the following 9 variables.
\begin{description}
 \item[list('year')] year\item[list('month')] month (not really
needed)\item[list('day')] day of the month (not really needed)
\item[list('JD')] day of the year (1-365)\item[list('maxTemp')] maximum
temperature (Celsius)\item[list('minTemp')] minimum temperature
(Celsius)\item[list('SolarR')] solar radiation (MJ/m2)
\item[list('PotEv')] potential evaporation (kg/m2). Approx. mm.
\item[list('precip')] precipitation (kg/m2). Approx. mm.
\end{description}
\end{Format}
%
\begin{Source}\relax
\url{http://www.noaa.gov/}
\end{Source}
%
\begin{Examples}
\begin{ExampleCode}
data(WD1979)
summary(WD1979)
\end{ExampleCode}
\end{Examples}
\inputencoding{utf8}
\HeaderA{weach}{Simulates the hourly conditions from daily}{weach}
\keyword{datagen}{weach}
%
\begin{Description}\relax
Manipulates weather data in the format obtained from WARM
(see link below) and returns the format and units needed
for most functions in this package. This function should
be used for one year at a time.  It returns hourly (or
sub-daily) weather information.
\end{Description}
%
\begin{Usage}
\begin{verbatim}
  weach(X, lati, ts = 1,
    temp.units = c("Fahrenheit", "Celsius"),
    rh.units = c("percent", "fraction"),
    ws.units = c("mph", "mps"), pp.units = c("in", "mm"),
    ...)
\end{verbatim}
\end{Usage}
%
\begin{Arguments}
\begin{ldescription}
\item[\code{X}] a matrix (or data frame) containing weather
information.  The input format is strict but it is meant
to be used with the data usually obtained from weather
stations in Illinois. The data frame should have 11
columns (see details).

\item[\code{lati}] latitude at the specific location

\item[\code{ts}] timestep for the simulation of sub-daily data
from daily. For example a value of 3 would return data
every 3 hours. Only divisors of 24 work (i.e. 1,2,3,4,
etc.).

\item[\code{temp.units}] Option to specify the units in which
the temperature is entered. Default is Farenheit.

\item[\code{rh.units}] Option to specify the units in which the
relative humidity is entered. Default is percent.

\item[\code{ws.units}] Option to specify the units in which the
wind speed is entered. Default is miles per hour.

\item[\code{pp.units}] Option to specify the units in which the
precipitation is entered. Default is inches.

\item[\code{list()}] additional arguments to be passed to
\code{\LinkA{lightME}{lightME}}
\end{ldescription}
\end{Arguments}
%
\begin{Details}\relax
This function was originally used to transform daily data
to hourly data. Some flexibility has been added so that
other units can be used. The input data used originally
looked as follows. \begin{enumerate}


\bsl{}itemcol 1year \bsl{}itemcol 2day of the year (1--365). Does
not consider leap years. \bsl{}itemcol 3total daily solar
radiation (MJ/m\textasciicircum{}2). \bsl{}itemcol 4maximum temperature
(Fahrenheit). \bsl{}itemcol 5minimum temperature (Fahrenheit).
\bsl{}itemcol 6average temperature (Fahrenheit). \bsl{}itemcol
7maximum relative humidity (\%). \bsl{}itemcol 8minimum
relative humidity (\%). \bsl{}itemcol 9average relative
humidity (\%). \bsl{}itemcol 10average wind speed (miles per
hour). \bsl{}itemcol 11precipitation (inches). 
\end{enumerate}


All the units above are the defaults but they can be
changed as part of the arguments.
\end{Details}
%
\begin{Value}
a \code{\LinkA{matrix}{matrix}} returning hourly (or sub-daily)
weather data. Dimensions 8760 (if hourly) by 8.
\end{Value}
%
\begin{Examples}
\begin{ExampleCode}
data(cmi0506)
tmp1 <- cmi0506[cmi0506$year == 2005,]
wet05 <- weach(tmp1,40)

# Return data every 3 hours
wet05.3 <- weach(tmp1,40,ts=3)
\end{ExampleCode}
\end{Examples}
\inputencoding{utf8}
\HeaderA{weach365}{if the function CheckLeapYear returns 0 this function will be used}{weach365}
%
\begin{Description}\relax
if the function CheckLeapYear returns 0 this function will
be used
\end{Description}
%
\begin{Usage}
\begin{verbatim}
weach365(X, lati, ts = 1, temp.units = c("Farenheit", "Celsius"),
  rh.units = c("percent", "fraction"), ws.units = c("mph", "mps"),
  pp.units = c("in", "mm"), ...)
\end{verbatim}
\end{Usage}
%
\begin{Arguments}
\begin{ldescription}
\item[\code{X}] a matrix (or data frame) containing weather
information.  The input format is strict but it is meant
to be used with the data usually obtained from weather
stations in Illinois. The data frame should have 11
columns (see details).

\item[\code{lati}] latitude at the specific location

\item[\code{ts}] timestep for the simulation of sub-daily data
from daily. For example a value of 3 would return data
every 3 hours. Only divisors of 24 work (i.e. 1,2,3,4,
etc.).

\item[\code{temp.units}] Option to specify the units in which
the temperature is entered. Default is Farenheit.

\item[\code{rh.units}] Option to specify the units in which the
relative humidity is entered. Default is percent.

\item[\code{ws.units}] Option to specify the units in which the
wind speed is entered. Default is miles per hour.

\item[\code{pp.units}] Option to specify the units in which the
precipitation is entered. Default is inches.

\item[\code{list()}] additional arguments to be passed to
\code{\LinkA{lightME}{lightME}}
\end{ldescription}
\end{Arguments}
\inputencoding{utf8}
\HeaderA{weach366}{if the function CheckLeapYear returns 1 this function will be used}{weach366}
%
\begin{Description}\relax
if the function CheckLeapYear returns 1 this function will
be used
\end{Description}
%
\begin{Usage}
\begin{verbatim}
weach366(X, lati, ts = 1, temp.units = c("Farenheit", "Celsius"),
  rh.units = c("percent", "fraction"), ws.units = c("mph", "mps"),
  pp.units = c("in", "mm"), ...)
\end{verbatim}
\end{Usage}
%
\begin{Arguments}
\begin{ldescription}
\item[\code{X}] a matrix (or data frame) containing weather
information.  The input format is strict but it is meant
to be used with the data usually obtained from weather
stations in Illinois. The data frame should have 11
columns (see details).

\item[\code{lati}] latitude at the specific location

\item[\code{ts}] timestep for the simulation of sub-daily data
from daily. For example a value of 3 would return data
every 3 hours. Only divisors of 24 work (i.e. 1,2,3,4,
etc.).

\item[\code{temp.units}] Option to specify the units in which
the temperature is entered. Default is Farenheit.

\item[\code{rh.units}] Option to specify the units in which the
relative humidity is entered. Default is percent.

\item[\code{ws.units}] Option to specify the units in which the
wind speed is entered. Default is miles per hour.

\item[\code{pp.units}] Option to specify the units in which the
precipitation is entered. Default is inches.

\item[\code{list()}] additional arguments to be passed to
\code{\LinkA{lightME}{lightME}}
\end{ldescription}
\end{Arguments}
\inputencoding{utf8}
\HeaderA{weachDT}{weachDT}{weachDT}
%
\begin{Description}\relax
Simple, Fast Daily to Hourly Climate Downscaling
\end{Description}
%
\begin{Usage}
\begin{verbatim}
  weachDT(X, lati)
\end{verbatim}
\end{Usage}
%
\begin{Arguments}
\begin{ldescription}
\item[\code{X}] data table with climate variables

\item[\code{lati}] latitude (for calculating solar radiation)
\end{ldescription}
\end{Arguments}
%
\begin{Details}\relax
Based on weach family of functions but 5x faster than
weachNEW, and requiring metric units (temperature in
celsius, windspeed in kph, precip in mm, relative
humidity as fraction)
\end{Details}
%
\begin{Value}
weather file for input to BioGro and related crop growth
functions
\end{Value}
%
\begin{Author}\relax
David LeBauer
\end{Author}
\inputencoding{utf8}
\HeaderA{weachNEW}{algorithm that decides weather weach365 or weach366 will be used based on the output of CheckLeapYear}{weachNEW}
%
\begin{Description}\relax
algorithm that decides weather weach365 or weach366 will be
used based on the output of CheckLeapYear
\end{Description}
%
\begin{Usage}
\begin{verbatim}
weachNEW(X, lati, ts = 1, temp.units = c("Farenheit", "Celsius"),
  rh.units = c("percent", "fraction"), ws.units = c("mph", "mps"),
  pp.units = c("in", "mm"), ...)
\end{verbatim}
\end{Usage}
%
\begin{Arguments}
\begin{ldescription}
\item[\code{X}] a matrix (or data frame) containing weather
information.  The input format is strict but it is meant
to be used with the data usually obtained from weather
stations in Illinois. The data frame should have 11
columns (see details).

\item[\code{lati}] latitude at the specific location

\item[\code{ts}] timestep for the simulation of sub-daily data
from daily. For example a value of 3 would return data
every 3 hours. Only divisors of 24 work (i.e. 1,2,3,4,
etc.).

\item[\code{temp.units}] Option to specify the units in which
the temperature is entered. Default is Farenheit.

\item[\code{rh.units}] Option to specify the units in which the
relative humidity is entered. Default is percent.

\item[\code{ws.units}] Option to specify the units in which the
wind speed is entered. Default is miles per hour.

\item[\code{pp.units}] Option to specify the units in which the
precipitation is entered. Default is inches.

\item[\code{list()}] additional arguments to be passed to
\code{\LinkA{lightME}{lightME}}
\end{ldescription}
\end{Arguments}
\inputencoding{utf8}
\HeaderA{weach\_imn}{Weather change Iowa Mesonet}{weach.Rul.imn}
\keyword{datagen}{weach\_imn}
%
\begin{Description}\relax
Manipulates weather data in the format obtained from Iowa
Mesonet (see link below) and returns the format and units
needed for most functions in this package. This function
should be used for one year at a time.  It takes and
returns hourly weather information only.
\end{Description}
%
\begin{Usage}
\begin{verbatim}
weach_imn(data, lati, ts = 1, temp.units = c("Fahrenheit", "Celsius"),
  rh.units = c("percent", "fraction"), ws.units = c("mph", "mps"),
  pp.units = c("in", "mm"), ...)
\end{verbatim}
\end{Usage}
%
\begin{Arguments}
\begin{ldescription}
\item[\code{data}] data as obtained from the Iowa Mesonet (see
details)

\item[\code{lati}] Latitude, not used at the moment

\item[\code{ts}] Time step, not used at the moment

\item[\code{temp.units}] Temperature units

\item[\code{rh.units}] Relative humidity units

\item[\code{ws.units}] wind speed units

\item[\code{pp.units}] precipitation units

\item[\code{...}] 
\end{ldescription}
\end{Arguments}
%
\begin{Details}\relax
This function should be used to transform data from the
Iowa Mesonet at hourly intervals from here:

http://mesonet.agron.iastate.edu/agclimate/hist/hourlyRequest.php

When selecting to download variables: Air Temperature
(Fahrenheit) Solar Radiation (kilocalories per meter
squared) Precipitation (inches) Relative humidity (percent)
Wind Speed (mph)

You can read the data directly as it is downloaded making
sure you skip the first 6 lines (This includes the title
row).

The imported data frame should have 9 columns with:

\begin{enumerate}


\item site ID \item site name \item date in format
"year-month-day", e.g. '2010-3-25' \item hour in format
"hour:minute", e.g. '15:00' \item temperature
(Fahrenheit) \item solar radiation (kilocalories per meter
squared) \item precipitation (inches) \item relative
humidity (\%). \item wind speed (mph) 
\end{enumerate}



above\textasciitilde{}\textasciitilde{}
\end{Details}
%
\begin{Value}
It will return a data frame in the same format as the
\code{\LinkA{weach}{weach}} function.
\end{Value}
%
\begin{Author}\relax
Fernando E. Miguez
\end{Author}
%
\begin{References}\relax
Iowa Mesonet http://mesonet.agron.iastate.edu/index.phtml

\end{References}
%
\begin{SeeAlso}\relax
\code{\LinkA{weach}{weach}} 
\code{\LinkA{help}{help}}, \textasciitilde{}\textasciitilde{}\textasciitilde{}
\end{SeeAlso}
%
\begin{Examples}
\begin{ExampleCode}
## Read an example data set from my website
url <- "http://www.agron.iastate.edu/miguezlab/teaching/CropSoilModel/ames_2010-iowamesonet.txt"
ames.wea <- read.table(url, skip = 6)
ames.wea2 <- weach_imn(ames.wea)
\end{ExampleCode}
\end{Examples}
\inputencoding{utf8}
\HeaderA{weather05}{Weather data}{weather05}
\aliasA{weather04}{weather05}{weather04}
\keyword{datasets}{weather05}
%
\begin{Description}\relax
Weather data as produced by the \code{\LinkA{weach}{weach}}
function.  These are for 2004 and 2005.
\end{Description}
%
\begin{Format}
data frame of dimensions 8760 by 7.
\end{Format}
%
\begin{Source}\relax
simulated (based on Champaign, Illinois conditions).
\end{Source}
\inputencoding{utf8}
\HeaderA{weather06}{Weather data}{weather06}
\keyword{datasets}{weather06}
%
\begin{Description}\relax

\textasciitilde{}\textasciitilde{}
\end{Description}
%
\begin{Format}
A data frame with 8760 observations on the following 8 variables.
\begin{description}
 \item[list('year')] a numeric vector\item[list('doy')] a
numeric vector\item[list('hour')] a numeric vector
\item[list('SolarR')] a numeric vector\item[list('Temp')] a numeric
vector\item[list('RH')] a numeric vector\item[list('WS')] a numeric
vector\item[list('precip')] a numeric vector
\end{description}
\end{Format}
%
\begin{Details}\relax

above \textasciitilde{}\textasciitilde{}
\end{Details}
%
\begin{Source}\relax

data were obtained \textasciitilde{}\textasciitilde{}
\end{Source}
%
\begin{Examples}
\begin{ExampleCode}
data(weather06)
## maybe str(weather06) ; plot(weather06) ...
\end{ExampleCode}
\end{Examples}
\inputencoding{utf8}
\HeaderA{willowCent}{willowmass crops growth simulation}{willowCent}
\aliasA{canopyParms}{willowCent}{canopyParms}
\aliasA{centuryParms}{willowCent}{centuryParms}
\aliasA{nitroParms}{willowCent}{nitroParms}
\aliasA{phenoParms}{willowCent}{phenoParms}
\aliasA{photoParms}{willowCent}{photoParms}
\aliasA{print.willowGro}{willowCent}{print.willowGro}
\aliasA{seneParms}{willowCent}{seneParms}
\aliasA{showSoilType}{willowCent}{showSoilType}
\aliasA{soilParms}{willowCent}{soilParms}
\aliasA{SoilType}{willowCent}{SoilType}
\aliasA{willowGro}{willowCent}{willowGro}
\keyword{models}{willowCent}
%
\begin{Description}\relax
Simulates dry biomass growth during an entire growing
season.  It represents an integration of the
photosynthesis function \code{\LinkA{c3photo}{c3photo}}, canopy
evapo/transpiration \code{\LinkA{CanA}{CanA}}, the multilayer
canopy model \code{\LinkA{sunML}{sunML}} and a dry biomass
partitioning calendar and senescence. It also considers,
carbon and nitrogen cycles and water and nitrogen
limitations.
\end{Description}
%
\begin{Usage}
\begin{verbatim}
  willowCent(WetDat, day1 = 120, dayn = 300, timestep = 1,
    iRhizome = 1, lat = 40, iPlant = 1, irtl = 1e-04,
    canopyControl = list(), seneControl = list(),
    photoControl = list(), willowphenoControl = list(),
    soilControl = list(), nitroControl = list(),
    iPlantControl = list(), centuryControl = list())
\end{verbatim}
\end{Usage}
%
\begin{Arguments}
\begin{ldescription}
\item[\code{WetDat}] weather data as produced by the
\code{\LinkA{weach}{weach}} function.

\item[\code{day1}] first day of the growing season, (1--365).

\item[\code{dayn}] last day of the growing season, (1--365, but
larger than \code{day1}). See details.

\item[\code{timestep}] Simulation timestep, the default of 1
requires houlry weather data. A value of 3 would require
weather data every 3 hours.  This number should be a
divisor of 24.

\item[\code{lat}] latitude, default 40.

\item[\code{iRhizome}] initial dry biomass of the Rhizome (Mg
\eqn{ha^{-1}}{}).

\item[\code{irtl}] Initial rhizome proportion that becomes leaf.
This should not typically be changed, but it can be used
to indirectly control the effect of planting density.

\item[\code{canopyControl}] List that controls aspects of the
canopy simulation. It should be supplied through the
\code{canopyParms} function.

\code{Sp} (specific leaf area) here the units are ha
\eqn{Mg^{-1}}{}.  If you have data in \eqn{m^2}{} of leaf per
kg of dry matter (e.g. 15) then divide by 10 before
inputting this coefficient.

\code{nlayers} (number of layers of the canopy) Maximum
50. To increase the number of layers (more than 50) the
\code{C} source code needs to be changed slightly.

\code{kd} (extinction coefficient for diffuse light)
between 0 and 1.

\code{mResp} (maintenance respiration) a vector of length
2 with the first component for leaf and stem and the
second component for rhizome and root.

\item[\code{seneControl}] List that controls aspects of
senescence simulation. It should be supplied through the
\code{seneParms} function.

\code{senLeaf} Thermal time at which leaf senescence will
start.

\code{senStem} Thermal time at which stem senescence will
start.

\code{senRoot} Thermal time at which root senescence will
start.

\code{senRhizome} Thermal time at which rhizome
senescence will start.

\item[\code{photoControl}] List that controls aspects of
photosynthesis simulation. It should be supplied through
the \code{photoParms} function.

\code{vmax} Vmax passed to the \code{\LinkA{c3photo}{c3photo}}
function.

\code{alpha} alpha parameter passed to the
\code{\LinkA{c3photo}{c3photo}} function.

\code{theta} theta parameter passed to the
\code{\LinkA{c3photo}{c3photo}} function.

\code{beta} beta parameter passed to the
\code{\LinkA{c3photo}{c3photo}} function.

\code{Rd} Rd parameter passed to the
\code{\LinkA{c3photo}{c3photo}} function.

\code{Catm} Catm parameter passed to the
\code{\LinkA{c3photo}{c3photo}} function.

\code{b0} b0 parameter passed to the
\code{\LinkA{c3photo}{c3photo}} function.

\code{b1} b1 parameter passed to the
\code{\LinkA{c3photo}{c3photo}} function.

\item[\code{phenoControl}] List that controls aspects of the
crop phenology. It should be supplied through the
\code{phenoParms} function.

\code{tp1-tp6} thermal times which determine the time
elapsed between phenological stages. Between 0 and tp1 is
the juvenile stage. etc.

\code{kLeaf1-6} proportion of the carbon that is
allocated to leaf for phenological stages 1 through 6.

\code{kStem1-6} proportion of the carbon that is
allocated to stem for phenological stages 1 through 6.

\code{kRoot1-6} proportion of the carbon that is
allocated to root for phenological stages 1 through 6.

\code{kRhizome1-6} proportion of the carbon that is
allocated to rhizome for phenological stages 1 through 6.

\code{kGrain1-6} proportion of the carbon that is
allocated to grain for phenological stages 1 through 6.
At the moment only the last stage (i.e. 6 or
post-flowering) is allowed to be larger than zero. An
error will be returned if kGrain1-5 are different from
zero.

\item[\code{soilControl}] List that controls aspects of the soil
environment. It should be supplied through the
\code{soilParms} function.

\code{FieldC} Field capacity. This can be used to
override the defaults possible from the soil types (see
\code{\LinkA{showSoilType}{showSoilType}}).

\code{WiltP} Wilting point.  This can be used to override
the defaults possible from the soil types (see
\code{\LinkA{showSoilType}{showSoilType}}).

\code{phi1} Parameter which controls the spread of the
logistic function. See \code{\LinkA{wtrstr}{wtrstr}} for more
details.

\code{phi2} Parameter which controls the reduction of the
leaf area growth due to water stress. See
\code{\LinkA{wtrstr}{wtrstr}} for more details.

\code{soilDepth} Maximum depth of the soil that the roots
have access to (i.e. rooting depth).

\code{iWatCont} Initial water content of the soil the
first day of the growing season. It can be a single value
or a vector for the number of layers specified.

\code{soilType} Soil type, default is 6 (a more typical
soil would be 3). To see details use the function
\code{\LinkA{showSoilType}{showSoilType}}.

\code{soilLayer} Integer between 1 and 50. The default is
5. If only one soil layer is used the behavior can be
quite different.

\code{soilDepths} Intervals for the soil layers.

\code{wsFun} one of 'logistic','linear','exp' or 'none'.
Controls the method for the relationship between soil
water content and water stress factor.

\code{scsf} stomatal conductance sensitivity factor
(default = 1). This is an empirical coefficient that
needs to be adjusted for different species.

\code{rfl} Root factor lambda. A Poisson distribution is
used to simulate the distribution of roots in the soil
profile and this parameter can be used to change the
lambda parameter of the Poisson.

\code{rsec} Radiation soil evaporation coefficient.
Empirical coefficient used in the incidence of direct
radiation on soil evaporation.

\code{rsdf} Root soil depth factor. Empirical coefficient
used in calculating the depth of roots as a function of
root biomass.

\item[\code{nitroControl}] List that controls aspects of the
nitrogen environment. It should be supplied through the
\code{nitrolParms} function.

\code{iLeafN} initial value of leaf nitrogen (g m-2).

\code{kLN} coefficient of decrease in leaf nitrogen
during the growing season. The equation is LN = iLeafN *
(Stem + Leaf)\textasciicircum{}-kLN .

\code{Vmax.b1} slope which determines the effect of leaf
nitrogen on Vmax.

\code{alpha.b1} slope which controls the effect of leaf
nitrogen on alpha.

\item[\code{centuryControl}] List that controls aspects of the
Century model for carbon and nitrogen dynamics in the
soil. It should be supplied through the
\code{centuryParms} function.

\code{SC1-9} Soil carbon pools in the soil.  SC1:
Structural surface litter.  SC2: Metabolic surface
litter.  SC3: Structural root litter.  SC4: Metabolic
root litter.  SC5: Surface microbe.  SC6: Soil microbe.
SC7: Slow carbon.  SC8: Passive carbon.  SC9: Leached
carbon.

\code{LeafL.Ln} Leaf litter lignin content.

\code{StemL.Ln} Stem litter lignin content.

\code{RootL.Ln} Root litter lignin content.

\code{RhizomeL.Ln} Rhizome litter lignin content.

\code{LeafL.N} Leaf litter nitrogen content.

\code{StemL.N} Stem litter nitrogen content.

\code{RootL.N} Root litter nitrogen content.

\code{RhizomeL.N} Rhizome litter nitrogen content.

\code{Nfert} Nitrogen from a fertilizer source.

\code{iMinN} Initial value for the mineral nitrogen pool.

\code{Litter} Initial values of litter (leaf, stem, root,
rhizome).

\code{timestep} currently either week (default) or day.
\end{ldescription}
\end{Arguments}
%
\begin{Value}
a \code{\LinkA{list}{list}} structure with components
\end{Value}
%
\begin{Examples}
\begin{ExampleCode}
## Not run: 
data(weather05)

res0 <- willowGro(weather05)

plot(res0)

## Looking at the soil model

res1 <- willowGro(weather05, soilControl = soilParms(soilLayers = 6))
plot(res1, plot.kind='SW') ## Without hydraulic distribution
res2 <- willowGro(weather05, soilControl = soilParms(soilLayers = 6, hydrDist=TRUE))
plot(res2, plot.kind='SW') ## With hydraulic distribution


## Example of user defined soil parameters.
## The effect of phi2 on yield and soil water content

ll.0 <- soilParms(FieldC=0.37,WiltP=0.2,phi2=1)
ll.1 <- soilParms(FieldC=0.37,WiltP=0.2,phi2=2)
ll.2 <- soilParms(FieldC=0.37,WiltP=0.2,phi2=3)
ll.3 <- soilParms(FieldC=0.37,WiltP=0.2,phi2=4)

ans.0 <- willowGro(weather05,soilControl=ll.0)
ans.1 <- willowGro(weather05,soilControl=ll.1)
ans.2 <- willowGro(weather05,soilControl=ll.2)
ans.3 <-willowGro(weather05,soilControl=ll.3)

xyplot(ans.0$SoilWatCont +
       ans.1$SoilWatCont +
       ans.2$SoilWatCont +
       ans.3$SoilWatCont ~ ans.0$DayofYear,
       type='l',
       ylab='Soil water Content (fraction)',
       xlab='DOY')

## Compare LAI

xyplot(ans.0$LAI +
       ans.1$LAI +
       ans.2$LAI +
       ans.3$LAI ~ ans.0$DayofYear,
       type='l',
       ylab='Leaf Area Index',
       xlab='DOY')




## End(Not run)
\end{ExampleCode}
\end{Examples}
\inputencoding{utf8}
\HeaderA{willowGro}{willowmass crops growth simulation}{willowGro}
\aliasA{canopyParms}{willowGro}{canopyParms}
\aliasA{centuryParms}{willowGro}{centuryParms}
\aliasA{nitroParms}{willowGro}{nitroParms}
\aliasA{phenoParms}{willowGro}{phenoParms}
\aliasA{photoParms}{willowGro}{photoParms}
\aliasA{print.willowGro}{willowGro}{print.willowGro}
\aliasA{seneParms}{willowGro}{seneParms}
\aliasA{showSoilType}{willowGro}{showSoilType}
\aliasA{soilParms}{willowGro}{soilParms}
\aliasA{SoilType}{willowGro}{SoilType}
\keyword{models}{willowGro}
%
\begin{Description}\relax
Simulates dry biomass growth during an entire growing
season.  It represents an integration of the
photosynthesis function \code{\LinkA{c3photo}{c3photo}}, canopy
evapo/transpiration \code{\LinkA{CanA}{CanA}}, the multilayer
canopy model \code{\LinkA{sunML}{sunML}} and a dry biomass
partitioning calendar and senescence. It also considers,
carbon and nitrogen cycles and water and nitrogen
limitations.
\end{Description}
%
\begin{Usage}
\begin{verbatim}
  willowGro(WetDat, day1 = NULL, dayn = NULL, timestep = 1,
    iRhizome = 1, lat = 40, iPlant = 1, irtl = 1e-04,
    canopyControl = list(), seneControl = list(),
    photoControl = list(), willowphenoControl = list(),
    soilControl = list(), nitroControl = list(),
    iPlantControl = list(), centuryControl = list())
\end{verbatim}
\end{Usage}
%
\begin{Arguments}
\begin{ldescription}
\item[\code{WetDat}] weather data as produced by the
\code{\LinkA{weach}{weach}} function.

\item[\code{day1}] first day of the growing season, (1--365).

\item[\code{dayn}] last day of the growing season, (1--365, but
larger than \code{day1}). See details.

\item[\code{timestep}] Simulation timestep, the default of 1
requires houlry weather data. A value of 3 would require
weather data every 3 hours.  This number should be a
divisor of 24.

\item[\code{lat}] latitude, default 40.

\item[\code{iRhizome}] initial dry biomass of the Rhizome (Mg
\eqn{ha^{-1}}{}).

\item[\code{irtl}] Initial rhizome proportion that becomes leaf.
This should not typically be changed, but it can be used
to indirectly control the effect of planting density.

\item[\code{canopyControl}] List that controls aspects of the
canopy simulation. It should be supplied through the
\code{canopyParms} function.

\code{Sp} (specific leaf area) here the units are ha
\eqn{Mg^{-1}}{}.  If you have data in \eqn{m^2}{} of leaf per
kg of dry matter (e.g. 15) then divide by 10 before
inputting this coefficient.

\code{nlayers} (number of layers of the canopy) Maximum
50. To increase the number of layers (more than 50) the
\code{C} source code needs to be changed slightly.

\code{kd} (extinction coefficient for diffuse light)
between 0 and 1.

\code{mResp} (maintenance respiration) a vector of length
2 with the first component for leaf and stem and the
second component for rhizome and root.

\item[\code{seneControl}] List that controls aspects of
senescence simulation. It should be supplied through the
\code{seneParms} function.

\code{senLeaf} Thermal time at which leaf senescence will
start.

\code{senStem} Thermal time at which stem senescence will
start.

\code{senRoot} Thermal time at which root senescence will
start.

\code{senRhizome} Thermal time at which rhizome
senescence will start.

\item[\code{photoControl}] List that controls aspects of
photosynthesis simulation. It should be supplied through
the \code{photoParms} function.

\code{vmax} Vmax passed to the \code{\LinkA{c3photo}{c3photo}}
function.

\code{alpha} alpha parameter passed to the
\code{\LinkA{c3photo}{c3photo}} function.

\code{theta} theta parameter passed to the
\code{\LinkA{c3photo}{c3photo}} function.

\code{beta} beta parameter passed to the
\code{\LinkA{c3photo}{c3photo}} function.

\code{Rd} Rd parameter passed to the
\code{\LinkA{c3photo}{c3photo}} function.

\code{Catm} Catm parameter passed to the
\code{\LinkA{c3photo}{c3photo}} function.

\code{b0} b0 parameter passed to the
\code{\LinkA{c3photo}{c3photo}} function.

\code{b1} b1 parameter passed to the
\code{\LinkA{c3photo}{c3photo}} function.

\item[\code{phenoControl}] List that controls aspects of the
crop phenology. It should be supplied through the
\code{phenoParms} function.

\code{tp1-tp6} thermal times which determine the time
elapsed between phenological stages. Between 0 and tp1 is
the juvenile stage. etc.

\code{kLeaf1-6} proportion of the carbon that is
allocated to leaf for phenological stages 1 through 6.

\code{kStem1-6} proportion of the carbon that is
allocated to stem for phenological stages 1 through 6.

\code{kRoot1-6} proportion of the carbon that is
allocated to root for phenological stages 1 through 6.

\code{kRhizome1-6} proportion of the carbon that is
allocated to rhizome for phenological stages 1 through 6.

\code{kGrain1-6} proportion of the carbon that is
allocated to grain for phenological stages 1 through 6.
At the moment only the last stage (i.e. 6 or
post-flowering) is allowed to be larger than zero. An
error will be returned if kGrain1-5 are different from
zero.

\item[\code{soilControl}] List that controls aspects of the soil
environment. It should be supplied through the
\code{soilParms} function.

\code{FieldC} Field capacity. This can be used to
override the defaults possible from the soil types (see
\code{\LinkA{showSoilType}{showSoilType}}).

\code{WiltP} Wilting point.  This can be used to override
the defaults possible from the soil types (see
\code{\LinkA{showSoilType}{showSoilType}}).

\code{phi1} Parameter which controls the spread of the
logistic function. See \code{\LinkA{wtrstr}{wtrstr}} for more
details.

\code{phi2} Parameter which controls the reduction of the
leaf area growth due to water stress. See
\code{\LinkA{wtrstr}{wtrstr}} for more details.

\code{soilDepth} Maximum depth of the soil that the roots
have access to (i.e. rooting depth).

\code{iWatCont} Initial water content of the soil the
first day of the growing season. It can be a single value
or a vector for the number of layers specified.

\code{soilType} Soil type, default is 6 (a more typical
soil would be 3). To see details use the function
\code{\LinkA{showSoilType}{showSoilType}}.

\code{soilLayer} Integer between 1 and 50. The default is
5. If only one soil layer is used the behavior can be
quite different.

\code{soilDepths} Intervals for the soil layers.

\code{wsFun} one of 'logistic','linear','exp' or 'none'.
Controls the method for the relationship between soil
water content and water stress factor.

\code{scsf} stomatal conductance sensitivity factor
(default = 1). This is an empirical coefficient that
needs to be adjusted for different species.

\code{rfl} Root factor lambda. A Poisson distribution is
used to simulate the distribution of roots in the soil
profile and this parameter can be used to change the
lambda parameter of the Poisson.

\code{rsec} Radiation soil evaporation coefficient.
Empirical coefficient used in the incidence of direct
radiation on soil evaporation.

\code{rsdf} Root soil depth factor. Empirical coefficient
used in calculating the depth of roots as a function of
root biomass.

\item[\code{nitroControl}] List that controls aspects of the
nitrogen environment. It should be supplied through the
\code{nitrolParms} function.

\code{iLeafN} initial value of leaf nitrogen (g m-2).

\code{kLN} coefficient of decrease in leaf nitrogen
during the growing season. The equation is LN = iLeafN *
(Stem + Leaf)\textasciicircum{}-kLN .

\code{Vmax.b1} slope which determines the effect of leaf
nitrogen on Vmax.

\code{alpha.b1} slope which controls the effect of leaf
nitrogen on alpha.

\item[\code{centuryControl}] List that controls aspects of the
Century model for carbon and nitrogen dynamics in the
soil. It should be supplied through the
\code{centuryParms} function.

\code{SC1-9} Soil carbon pools in the soil.  SC1:
Structural surface litter.  SC2: Metabolic surface
litter.  SC3: Structural root litter.  SC4: Metabolic
root litter.  SC5: Surface microbe.  SC6: Soil microbe.
SC7: Slow carbon.  SC8: Passive carbon.  SC9: Leached
carbon.

\code{LeafL.Ln} Leaf litter lignin content.

\code{StemL.Ln} Stem litter lignin content.

\code{RootL.Ln} Root litter lignin content.

\code{RhizomeL.Ln} Rhizome litter lignin content.

\code{LeafL.N} Leaf litter nitrogen content.

\code{StemL.N} Stem litter nitrogen content.

\code{RootL.N} Root litter nitrogen content.

\code{RhizomeL.N} Rhizome litter nitrogen content.

\code{Nfert} Nitrogen from a fertilizer source.

\code{iMinN} Initial value for the mineral nitrogen pool.

\code{Litter} Initial values of litter (leaf, stem, root,
rhizome).

\code{timestep} currently either week (default) or day.
\end{ldescription}
\end{Arguments}
%
\begin{Value}
a \code{\LinkA{list}{list}} structure with components
\end{Value}
%
\begin{Examples}
\begin{ExampleCode}
## Not run: 
data(weather05)

res0 <- willowGro(weather05)

plot(res0)

## Looking at the soil model

res1 <- willowGro(weather05, soilControl = soilParms(soilLayers = 6))
plot(res1, plot.kind='SW') ## Without hydraulic distribution
res2 <- willowGro(weather05, soilControl = soilParms(soilLayers = 6, hydrDist=TRUE))
plot(res2, plot.kind='SW') ## With hydraulic distribution


## Example of user defined soil parameters.
## The effect of phi2 on yield and soil water content

ll.0 <- soilParms(FieldC=0.37,WiltP=0.2,phi2=1)
ll.1 <- soilParms(FieldC=0.37,WiltP=0.2,phi2=2)
ll.2 <- soilParms(FieldC=0.37,WiltP=0.2,phi2=3)
ll.3 <- soilParms(FieldC=0.37,WiltP=0.2,phi2=4)

ans.0 <- willowGro(weather05,soilControl=ll.0)
ans.1 <- willowGro(weather05,soilControl=ll.1)
ans.2 <- willowGro(weather05,soilControl=ll.2)
ans.3 <-willowGro(weather05,soilControl=ll.3)

xyplot(ans.0$SoilWatCont +
       ans.1$SoilWatCont +
       ans.2$SoilWatCont +
       ans.3$SoilWatCont ~ ans.0$DayofYear,
       type='l',
       ylab='Soil water Content (fraction)',
       xlab='DOY')

## Compare LAI

xyplot(ans.0$LAI +
       ans.1$LAI +
       ans.2$LAI +
       ans.3$LAI ~ ans.0$DayofYear,
       type='l',
       ylab='Leaf Area Index',
       xlab='DOY')




## End(Not run)
\end{ExampleCode}
\end{Examples}
\inputencoding{utf8}
\HeaderA{wsRcoef}{R coefficient for water stress}{wsRcoef}
%
\begin{Description}\relax
determines whether the argument wsFun is linear, logistic,
exponential, or something else and calculates a value for
wsPhoto based on that.
\end{Description}
%
\begin{Usage}
\begin{verbatim}
wsRcoef(aw, fieldc, wiltp, phi1, phi2, wsFun = c("linear", "logistic", "exp",
  "none"))
\end{verbatim}
\end{Usage}
%
\begin{Arguments}
\begin{ldescription}
\item[\code{aw}] available water

\item[\code{fieldc}] Field capacity of the soil (fraction).

\item[\code{wiltp}] Wilting point of the soil (fraction).

\item[\code{phi1}] coefficient which controls the spread of the
logistic function.

\item[\code{phi2}] coefficient which controls the effect on leaf
area expansion.

\item[\code{wsFun}] option to control which method is used for
the water stress function.
\end{ldescription}
\end{Arguments}
\inputencoding{utf8}
\HeaderA{wtrstr}{Simple function to illustrate soil water content effect on plant water stress.}{wtrstr}
\aliasA{wsRcoef}{wtrstr}{wsRcoef}
\keyword{models}{wtrstr}
%
\begin{Description}\relax
This is a very simple function which implements the
'bucket' model for soil water content and it calculates a
coefficient of plant water stress.
\end{Description}
%
\begin{Usage}
\begin{verbatim}
  wtrstr(precipt, evapo, cws, soildepth, fieldc, wiltp,
    phi1 = 0.01, phi2 = 10,
    wsFun = c("linear", "logistic", "exp", "none"))
\end{verbatim}
\end{Usage}
%
\begin{Arguments}
\begin{ldescription}
\item[\code{precipt}] Precipitation (mm).

\item[\code{evapo}] Evaporation (Mg H2O ha-1 hr-1).

\item[\code{cws}] current water content (fraction).

\item[\code{soildepth}] Soil depth, typically 1m.

\item[\code{fieldc}] Field capacity of the soil (fraction).

\item[\code{wiltp}] Wilting point of the soil (fraction).

\item[\code{phi1}] coefficient which controls the spread of the
logistic function.

\item[\code{phi2}] coefficient which controls the effect on leaf
area expansion.

\item[\code{wsFun}] option to control which method is used for
the water stress function.
\end{ldescription}
\end{Arguments}
%
\begin{Details}\relax
This is a very simple function and the details can be
seen in the code.
\end{Details}
%
\begin{Value}
A list with components:
\end{Value}
%
\begin{SeeAlso}\relax
\code{\LinkA{wsRcoef}{wsRcoef}}
\end{SeeAlso}
%
\begin{Examples}
\begin{ExampleCode}
## Looking at the three possible models for the effect of soil moisture on water
## stress

aws <- seq(0,0.4,0.001)
wats.L <- numeric(length(aws)) # linear
wats.Log <- numeric(length(aws)) # logistic
wats.exp <- numeric(length(aws)) # exp
wats.none <- numeric(length(aws)) # none
for(i in 1:length(aws)){
wats.L[i] <- wtrstr(1,1,aws[i],0.5,0.37,0.2,2e-2,4)$wsPhoto
wats.Log[i] <- wtrstr(1,1,aws[i],0.5,0.37,0.2,2e-2,4,wsFun='logistic')$wsPhoto
wats.exp[i] <- wtrstr(1,1,aws[i],0.5,0.37,0.2,2e-2,4, wsFun='exp')$wsPhoto
wats.none[i] <- wtrstr(1,1,aws[i],0.5,0.37,0.2,2e-2,4, wsFun='none')$wsPhoto
}

xyplot(wats.L + wats.Log + wats.exp  + wats.none~ aws,
       col=c('blue','green','purple','red'),
       type = 'l',
       xlab='Soil Water',
       ylab='Stress Coefficient',
       key = list(text=list(c('linear','logistic','exp', 'none')),
       col=c('blue','green','purple','red'), lines = TRUE) )

## This function is sensitive to the soil depth parameter

SDepth <- seq(0.05,2,0.05)

wats <- numeric(length(SDepth))

for(i in 1:length(SDepth)){
wats[i] <- wtrstr(1,1,0.3,SDepth[i],0.37,0.2,2e-2,3)$wsPhoto
}

xyplot(wats ~ SDepth, ylab='Water Stress Coef',
       xlab='Soil depth')

## Difference between the effect on assimilation and leaf expansion rate

aws <- seq(0,0.4,0.001)
wats.P <- numeric(length(aws))
wats.L <- numeric(length(aws))
for(i in 1:length(aws)){
wats.P[i] <- wtrstr(1,1,aws[i],0.5,0.37,0.2,2e-2,4)$wsPhoto
wats.L[i] <- wtrstr(1,1,aws[i],0.5,0.37,0.2,2e-2,4)$wsSpleaf
}

xyplot(wats.P + wats.L ~ aws,
       xlab='Soil Water',
       ylab='Stress Coefficient')


## An example for wsRcoef
## The scale parameter makes a big difference

aws <- seq(0.2,0.4,0.001)
wats.1 <- wsRcoef(aw=aws,fieldc=0.37,wiltp=0.2,phi1=1e-2,phi2=1, wsFun='logistic')$wsPhoto
wats.2 <- wsRcoef(aw=aws,fieldc=0.37,wiltp=0.2,phi1=2e-2,phi2=1, wsFun='logistic')$wsPhoto
wats.3 <- wsRcoef(aw=aws,fieldc=0.37,wiltp=0.2,phi1=3e-2,phi2=1, wsFun='logistic')$wsPhoto

xyplot(wats.1 + wats.2 + wats.3 ~ aws,type='l',
       col=c('blue','red','green'),
       ylab='Water Stress Coef',
       xlab='SoilWater Content',
       key=list(text=list(c('phi1 = 1e-2','phi1 = 2e-2','phi1 = 3e-2')),
         lines=TRUE,col=c('blue','red','green')))
\end{ExampleCode}
\end{Examples}
\printindex{}
\end{document}
